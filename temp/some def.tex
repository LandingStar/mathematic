\documentclass{book} 
\usepackage{graphicx} % Required for inserting images
\usepackage{mathrsfs}
\usepackage{amssymb}
\usepackage{amsmath}
\usepackage{indentfirst}
\usepackage{color}
\usepackage{hyperref}
\usepackage{xypic}
\usepackage{bbm}
\hypersetup{hidelinks,
	colorlinks=true,
	allcolors=black,
	pdfstartview=Fit,
	breaklinks=true
}
\newcommand{\abs}[1]{\left\lvert #1 \right\rvert} 
\newcommand{\leftbracket}{[}
\newcommand{\rightbracket}{]}
\begin{document}
\section{Def}
We call integral operator on S any $\mathbb{R}$-linear mapping $I:S\rightarrow\mathbb{R}$ that satisfies the following conditions:
\begin{itemize}
    \item [(1)]If $f\in S$ is such that $\forall \omega\in \Omega,f(\omega)\geq0$ then $I(f)\geq0$
    \item [(2)]If $(f_n)_{n\in \mathbb{N}}$ is  a decreasing sequence of elements in $S$ such that $\forall \omega\in \Omega\ \lim\limits_{n\rightarrow+\infty}f_n(\omega)=0$ then $$\lim\limits_{n\rightarrow+\infty}I(f_n)=0$$
    ($\forall \omega\in \Omega,n\in \mathbb{N},f_n(\omega)\geq f_{n+1}(\omega)$)
\end{itemize}
\section{Def}
Let $\Omega$ be a set. We call semialgebra on $\Omega$ any $\mathcal{C}\subseteqq\wp(\Omega)$ that verifies:
\begin{itemize}
    \item $\varnothing\in \mathcal{C}$
    \item $\forall(A,B)\in \mathcal{C}^2,A\cap B\in \mathcal{C}$
    \item $\forall(A,B)\in \mathcal{C}^2,\exists(C_i)_{i=1}^n$ a finite family of elements in $\mathcal{C}$ such that $B\setminus A=\bigsqcup\limits_{i=1}^n C_i$
\end{itemize}
\section{Def}
Let $\mathcal{C}$ be a semialgebra on $\Omega$. The set
$$\{A\in \wp(\Omega)\mid\exists n\in \mathbb{N},\exists(A_i)_{i=1}^n\in\mathcal{C}^n,A=\bigsqcup\limits_{i=1}^n A_i\}$$
is called the algebra generated by $\mathcal{C}$
\section{Def}
Let $\mathcal{C}\subseteq \wp(\Omega)$. We denote by $\sigma(\mathcal{C})$ the intersection of all $\sigma$-algebras on $\Omega$ containing $\mathcal{C}$. It's the smallest $\sigma$-algebra containing $\mathcal{C}$
\section{Def}
Let $f:X\rightarrow Y$ be a mapping of sets.\begin{itemize}
    \item For any $\mathcal{C}_Y\subseteq\wp(Y)$ we denote by $$f^{-1}(\mathcal{C}_Y):=\{f^{-1}(B)\mid B\in \mathcal{C}_Y\}$$
    \item For any $\mathcal{C}_X\subseteq\wp(X)$ we denote by $$f_*(\mathcal{C}_X):=\{B\subseteq Y\mid f^{-1}(B)\in \mathcal{C}_X\}$$
\end{itemize}
\section{Def}
Let $(X,\mathcal{G}_X)$ and $(Y,g_Y)$ be measurable spaces, $f:X\rightarrow Y$ be a mapping. If $f^{-1}(g_Y)\subseteq \mathcal{G}_X$ or equivalently $g_Y\subseteq f_8(\mathcal{G}_X)$ (or $\forall B\in g_Y,f^{-1}(B)\in \mathcal{G}_X$) then we say that $f$ is measurable.
\section{Def}
Let $\Omega$ be a set $((E_i,\mathcal{E}))_{i\in I}$ be a family of measurable spaces. $f=(f_i)_{i\in I}$ where $f_i:\Omega\rightarrow E_i$ is a mapping. We denote by $\sigma(f)$ the $\sigma$-algebra $\sigma(\bigcup\limits_{i\in I} f_i^{-1}(\mathcal{E}_i))$ It's the smallest $\sigma$-algebra on $\Omega$ making all $f_i$ measurable.

If $I_\mu$ is an integral operator, we say that $\mu$ is $\sigma$-additive.
\section{Def}
If $\exists(A_n)_{n\in \mathbb{N}}$ such that $\Omega=\bigcup\limits_{n\in \mathbb{N}}A_n$ and $\mu(A_n)<+\infty$ then $\mu$ is said to be $\sigma$-finite.
\section{Def}
We fix a measure space $(\Omega,\mathcal{G},\mu)$ the set of measurable mappings $f:\Omega\rightarrow\mathbb{R}$ such that
$$\norm{f}_{L^p}:=(\int_\Omega\abs{f(\omega)}^p\mu(dx))^{\frac{1}p}<+\infty$$
\end{document}