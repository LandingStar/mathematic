\documentclass{book} 
\usepackage{graphicx} % Required for inserting images
\usepackage{mathrsfs}
\usepackage{amssymb}
\usepackage{amsmath}
\usepackage{indentfirst}
\usepackage{color}
\usepackage{hyperref}
\usepackage{xypic}
\usepackage{bbm}
\usepackage{xeCJK}
\usepackage{dutchcal}
\usepackage{pgfplots}
\hypersetup{hidelinks,
	colorlinks=true,
	allcolors=black,
	pdfstartview=Fit,
	breaklinks=true
}
\newcommand{\abs}[1]{\left\lvert #1 \right\rvert} 
\newcommand{\norm}[1]{\left\lVert #1 \right\rVert}
\newcommand{\leftbracket}{[}
\newcommand{\rightbracket}{]}
\newcommand{\inprod}[2]{\left<#1,#2\right>}
\begin{document}
\tableofcontents
\chapter{Countable sets}
\section{Notation}
$\mathbb N=\mathbb N\setminus \{0\}$
\section{Def}
$S$ is \textbf{infinitely countable} if $\exists S\rightarrow \mathbb N$ bijection, \textbf{countable} if $S$ is finite or inf-countable
\subsection*{Remark}
\begin{itemize}
    \item for sequence $<S_n>_{n\in \mathbb N}$$$\begin{aligned}
        \mathbb N &\rightarrow S\\ n&\mapsto S_n
    \end{aligned}$$
    \item if $S\neq \varnothing$ then TFAE:
    \begin{itemize}
        \item $S$ is countable
        \item $\exists$ surjection $\mathbb N\rightarrow S$
        \item $\exists$ injection $S\rightarrow \mathbb N$
    \end{itemize}
    \item $\mathbb Q$ is inf-countable
    \item if $m\in \mathbb N_0$$S_1,\cdots,S_m$ are countable. Then $\prod\limits_{j=1}^m S_j$ is countable.
\end{itemize}
\section{Cantor Theorem}$\mathbb N$ is not equinumberous with $\mathbb{\wp(\mathbb N)}$
\subsection*{Proof}
$\wp(\mathbb N)\cong \{0,1\}^{\mathbb N}$ if $A\in \wp(\mathbb N)$ then $$\begin{aligned}
    \mathbbm{1}_A:&\mathbb{N}&\rightarrow&\{0,1\}\\ & n&\mapsto &\left\{\begin{aligned}
        1&\text{ if }n\in A\\ 0&\text{ if }n\notin A
    \end{aligned}\right.
\end{aligned}$$
the identify of $A$:
$$\begin{aligned}
    \wp(\mathbb N) &\rightarrow\{0,1\}^{\mathbb{N}}\\A &\mapsto \mathbbm 1_A
\end{aligned}$$
is a bijection$$\{0,1\}^{\mathbb N}=\mathcal{F}(\mathbb N;\{0,1\})$$
\subsecction*{Remark}
A,B be sets. $\mathcal{F}(A;B)$ is the set of all functions from A to B.
\subsection*{Proof}
Assume that $\exists$ bijection
$$\begin{align}
    \mathbb N &\rightarrow \wp(\mathbb N)\\ n&\mapsto f_n
\end{align}$$
Define $$\begin{aligned}
    f:&\mathbb N &\rightarrow &\{0,1\}\\ &n&\mapsto &\left\{\begin{aligned}
        0 &\text{ if }f_n(n)=1\\ 1&\text{ if }f_n(n)=0
    \end{aligned}\right.
\end{aligned}$$
$f\in \mathcal F(\mathbb N;\{0,1\})$ thus $\exists m\in \mathbb N$ s.t. $f=f_m$. Then $f_m(m)$ broken.
\chapter{Number Series}
\section{Def}
$\sum\limits_{n=0}^{+\infty}a_n$ is \textbf{commutatively convergent} (CC) if for each permutation $\phi$ of $\mathbb N$ the series $\sum\limits_{n=0}^{+\infty}a_{\phi(n)}$ converges. 
\subsectioin*{Remark}
A.C. is \textbf{absolutely convergent}.

C. is \textbf{convergent}.
Let $\varphi:\mathbb N\rightarrow \mathbb N$ be a bijection.
\begin{itemize}
    \item if $\sum\limits_{n=0}^{+\infty}a_n$ is A.C. then $\sum\limits_{n=0}^{+\infty}a_n$ C.
    \item $\sum\limits_{n=0}^{+\infty}\frac{(-1)^n}n$ C. but not A.C. or C.C.
\end{itemize}
\section{Riemann Theorem}
Let $\sum\limits_{n=0}^{+\infty}a_n$ be a convergent series in $\mathbb R$ TFAE:
\begin{itemize}
    \item $\sum\limits_{n=0}^{+\infty}a_n$ is not A.C.
    \item $\forall s\in \mathbb R\ \exists$ permutation of $\mathbb N$ s.t.$$\sum\limits_{n=0}^{+\infty}a_{\phi(n)}=s$$
    \item $\forall s\in \mathbb R\cup\{-\infty,+\infty\}\ \exists$permutation of $\mathbb N$ s.t.$$\sum\limits_{n=0}^{+\infty}a_{\phi(n)}=s$$ 
\end{itemize}
\chapter{Kurzneil-Henstock integral}
\section{Def}
\textbf{Cell} is a non-degenerated interval
\section{Nested cell theorem}
If $<I_n>_{n\in \mathbb N}$ is a decreasing sequence ($I_{n+1}\subseteq I_n$) of compact cells s.t.$$\lim\limits_{N\rightarrow +\infty} diam I_n=0$$ then $\exists x\in \mathbb R$$$\bigcap\limits_{n\in \mathbb N}I_n=\{x\}$$
\section{Exercises}
Every cell is uncountable.
\section{Def}
Two cells are \textbf{non-overlapping} if their intersection either empty or a singleton.
\section{Exercises}If $I_1,I_2,I_3$ are pairwise non-overlapping, then $$I_1\cap I_2\cup I_3= \varnothing$$ 
\section{Lemma}
If $I$ is a compact cell and $N\in \mathbb N_0$  are pairwise non-overlapping cells s.t. $\bigcup\limits_{n=1}^N I_n=I$ then renumbering them if necessary, we may get:
$$\begin{aligned}
    &\min I=\min I_1\\ &\max I_n=\min I_{n+1}\\ &\max I_N=\max I
\end{aligned}$$
\section{Def}
A \textbf{partial division} $\Delta$ of $I$ is a finite set consisting of non-overlapping compact sub-cells of $I$. If
$$\bigcup \Delta=I$$
it's called a \textbf{division} of $I$
\section{Lemma}If $\Delta$ is a partial division of $I$, then there exists a partial $\Delta'$ of $I$ s.t. $\Delta\cap\Delta'$ is a division of $I$
\section{Def}
A \textbf{gauge} on $I$ is a function$$\delta:I\rightarrow\mathbb R$$such that $\forall x\in I\ \delta(x)>0$
\subsection*{Remark}
If $\delta_1,\cdots,\delta_N$ are gauges on $I$ then $$\delta(x)=\min\{\delta_1(x),\cdots,\delta_N(x)\}$$
is also a gauge.
\section{Def}
A \textbf{partial P-division} of a compact cell $I$, is a finite $\Pi$ of pairs $(J,x)$ s.t.
\begin{itemize}
    \item $J\subseteq I$
    \item $J$ is a compact cell
    \item $x\in J$
    \item $\forall (J_1,x),(J_2,x_2)\in \Pi$ if $J_1\neq J_2$ then $J_1,J_2$ are non-overlapping
\end{itemize} 
$x$ is cal tag of the pair.
\section{Def}
Given a partial P-division $\Pi$ of I define
$$body(\Pi)=\bigcup\{J:(J,x)\in\Pi\}$$
A \textbf{P-division} $\Pi$ of $I$ is a partial P-division s.t. $body(\Pi)=I$
\section{Lemmas}
\begin{itemize}
    \item If $\Pi_1,\cdots,\Pi_N$ are partial P-divisions of I s.t. for each $n,m\in \{1,\cdots,N\}, n\neq m $ $body \Pi_n$ and $body \Pi_m$ are either disjoint or their intersection is a singleton, then $\bigcup\limits_{n=1}^N\Pi_n$ is a partial P-division of I.
    \item If $\Pi$ is a partial P-division of I and $\xi\in I$ then there're at most 2 $(J,x)\in \Pi$ s.t. $x=\xi$
\end{itemize}
\section{Def}Let $\delta$ be a gauge on I and $\Pi$ a (partial) P-division of I, we say that $\Pi$ is $\delta$-finite if$$\forall(J,x)\in \Pi\quad J\subseteq[x-\delta(x),x+\delta(x)]$$
\section{Def}
If $f:I\rightarrow\mathbb R$ and $\Pi$ is a (partial) P-division then the \textbf{Riemann sum} is defined as$$S(\Pi,f):=\sum\limits_{(J,x)\in \Pi}f(x)\abs{J}$$
\section{Def}
Let $f:I\rightarrow\mathbb R$ $f$ is \textbf{KH-integrable} on I if $\exists r\in \mathbb R, \forall \epsilon>0\exists$ gauge $\delta$ on I $\forall \delta$-finite P-division $\Pi$ of I$$\abs{S(\Pi,f)-r}<\epsilon$$
\section{Prop}$r$ is unique
\subsection*{Proof}
Assume that $r_1$ and $r_2$. Fix $\epsilon>0$. For $i=1,2$, there's a gauge $\delta_i$ on I s.t. if $\Pi$ is a $\delta_i$-finite P-division of I then $$\abs{S(\Pi,f)-r_i}<\epsilon$$
$$\begin{aligned}
    \abs{r_1-r_2}&=\abs{r_1-S(\Pi,f)+S(\Pi,f)-r_2}\\ &\leq\abs{r_1-S(\Pi,f)}+\abs{S(\Pi,f)-r_2}\\ &<2\epsilon
\end{aligned}$$
Let $\delta(x)=\min\{\delta_1(x),\delta_2(x)\}$ then $\delta$ is a gauge on I. If $\Pi$ is $\delta$-finite then it's $\delta_1$-finite and $\delta_2$-finite.
\section{Cousin Theorem}
I be a compact cell and $\delta$ a gauge on I. Then there exists a $\delta$-finite then P-division of I.
subsection*{Proof}
assume there's no. Then divide I into $I_l,I_r$ by middle. Then either $I_l,I_r$ has no $\delta$-finite division. Then we get a decreasing sequence $(I_n)_{n\in \mathbb N}$ by keeping dividing. According to nested theorem, get their intersection a singleton $x$. Notice that $x$ is a point of I, for $N\in \mathbb N$ big enough
$$diam I_N=2^{-N}\cdot diam I<\delta(x)$$then $\Pi=\{(I_N,x)\}$ is a $\delta$-finite P-division of $I_N$.
\section{Notation}
$$r=\int_If=\int_If(x)\text{d}x$$
if $I=[a,b]$
$$r=\int_a^bf=\int_a^bf(x)\text{d}x$$
\end{document}