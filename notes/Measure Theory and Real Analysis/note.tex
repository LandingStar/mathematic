\documentclass{book} 
\usepackage{graphicx} % Required for inserting images
\usepackage{mathrsfs}
\usepackage{amssymb}
\usepackage{amsmath}
\usepackage{indentfirst}
\usepackage{color}
\usepackage{hyperref}
\usepackage{xypic}
\usepackage{bbm}
\usepackage{xeCJK}
\usepackage{dutchcal}
\usepackage{pgfplots}
\hypersetup{hidelinks,
	colorlinks=true,
	allcolors=black,
	pdfstartview=Fit,
	breaklinks=true
}
\newcommand{\abs}[1]{\left\lvert #1 \right\rvert} 
\newcommand{\norm}[1]{\left\lVert #1 \right\rVert}
\newcommand{\leftbracket}{[}
\newcommand{\rightbracket}{]}
\newcommand{\inprod}[2]{\left<#1,#2\right>}
\newcommand{\fpartial}[3][]{\frac{\partial^{#1} #2}{\partial #3^{#1}}}
\begin{document}
\tableofcontents
\chapter{Countable sets}
\section{Notation}
$\mathbb N=\mathbb N\setminus \{0\}$
\section{Def}
$S$ is \textbf{infinitely countable} if $\exists S\rightarrow \mathbb N$ bijection, \textbf{countable} if $S$ is finite or inf-countable
\subsection*{Remark}
\begin{itemize}
    \item for sequence $<S_n>_{n\in \mathbb N}$$$\begin{aligned}
        \mathbb N &\rightarrow S\\ n&\mapsto S_n
    \end{aligned}$$
    \item if $S\neq \varnothing$ then TFAE:
    \begin{itemize}
        \item $S$ is countable
        \item $\exists$ surjection $\mathbb N\rightarrow S$
        \item $\exists$ injection $S\rightarrow \mathbb N$
    \end{itemize}
    \item $\mathbb Q$ is inf-countable
    \item if $m\in \mathbb N_0$$S_1,\cdots,S_m$ are countable. Then $\prod\limits_{j=1}^m S_j$ is countable.
\end{itemize}
\section{Cantor Theorem}$\mathbb N$ is not equinumberous with $\mathbb{\wp(\mathbb N)}$
\subsection*{Proof}
$\wp(\mathbb N)\cong \{0,1\}^{\mathbb N}$ if $A\in \wp(\mathbb N)$ then $$\begin{aligned}
    \mathbbm{1}_A:&\mathbb{N}&\rightarrow&\{0,1\}\\ & n&\mapsto &\left\{\begin{aligned}
        1&\text{ if }n\in A\\ 0&\text{ if }n\notin A
    \end{aligned}\right.
\end{aligned}$$
the identify of $A$:
$$\begin{aligned}
    \wp(\mathbb N) &\rightarrow\{0,1\}^{\mathbb{N}}\\A &\mapsto \mathbbm 1_A
\end{aligned}$$
is a bijection$$\{0,1\}^{\mathbb N}=\mathcal{F}(\mathbb N;\{0,1\})$$
\subsection*{Remark}
A,B be sets. $\mathcal{F}(A;B)$ is the set of all functions from A to B.
\subsection*{Proof}
Assume that $\exists$ bijection
$$\begin{aligned}
    \mathbb N &\rightarrow \wp(\mathbb N)\\ n&\mapsto f_n
\end{aligned}$$
Define $$\begin{aligned}
    f:&\mathbb N &\rightarrow &\{0,1\}\\ &n&\mapsto &\left\{\begin{aligned}
        0 &\text{ if }f_n(n)=1\\ 1&\text{ if }f_n(n)=0
    \end{aligned}\right.
\end{aligned}$$
$f\in \mathcal F(\mathbb N;\{0,1\})$ thus $\exists m\in \mathbb N$ s.t. $f=f_m$. Then $f_m(m)$ broken.
\chapter{Number Series}
\section{Def}
$\sum\limits_{n=0}^{+\infty}a_n$ is \textbf{commutatively convergent} (CC) if for each permutation $\phi$ of $\mathbb N$ the series $\sum\limits_{n=0}^{+\infty}a_{\phi(n)}$ converges. 
\subsection*{Remark}
A.C. is \textbf{absolutely convergent}.

C. is \textbf{convergent}.
Let $\varphi:\mathbb N\rightarrow \mathbb N$ be a bijection.
\begin{itemize}
    \item if $\sum\limits_{n=0}^{+\infty}a_n$ is A.C. then $\sum\limits_{n=0}^{+\infty}a_n$ C.
    \item $\sum\limits_{n=0}^{+\infty}\frac{(-1)^n}n$ C. but not A.C. or C.C.
\end{itemize}
\section{Riemann Theorem}
Let $\sum\limits_{n=0}^{+\infty}a_n$ be a convergent series in $\mathbb R$ TFAE:
\begin{itemize}
    \item $\sum\limits_{n=0}^{+\infty}a_n$ is not A.C.
    \item $\forall s\in \mathbb R\ \exists$ permutation of $\mathbb N$ s.t.$$\sum\limits_{n=0}^{+\infty}a_{\phi(n)}=s$$
    \item $\forall s\in \mathbb R\cup\{-\infty,+\infty\}\ \exists$permutation of $\mathbb N$ s.t.$$\sum\limits_{n=0}^{+\infty}a_{\phi(n)}=s$$ 
\end{itemize}
\chapter{Kurzneil-Henstock integral}
\section{Def}
\textbf{Cell} is a non-degenerated interval
\section{Nested cell theorem}
If $<I_n>_{n\in \mathbb N}$ is a decreasing sequence ($I_{n+1}\subseteq I_n$) of compact cells s.t.$$\lim\limits_{N\rightarrow +\infty} diam I_n=0$$ then $\exists x\in \mathbb R$$$\bigcap\limits_{n\in \mathbb N}I_n=\{x\}$$
\section{Exercises}
Every cell is uncountable.
\section{Def}
Two cells are \textbf{non-overlapping} if their intersection either empty or a singleton.
\section{Exercises}If $I_1,I_2,I_3$ are pairwise non-overlapping, then $$I_1\cap I_2\cup I_3= \varnothing$$ 
\section{Lemma}
If $I$ is a compact cell and $N\in \mathbb N_0$  are pairwise non-overlapping cells s.t. $\bigcup\limits_{n=1}^N I_n=I$ then renumbering them if necessary, we may get:
$$\begin{aligned}
    &\min I=\min I_1\\ &\max I_n=\min I_{n+1}\\ &\max I_N=\max I
\end{aligned}$$
\section{Def}
A \textbf{partial division} $\Delta$ of $I$ is a finite set consisting of non-overlapping compact sub-cells of $I$. If
$$\bigcup \Delta=I$$
it's called a \textbf{division} of $I$
\section{Lemma}If $\Delta$ is a partial division of $I$, then there exists a partial $\Delta'$ of $I$ s.t. $\Delta\cap\Delta'$ is a division of $I$
\section{Def}
A \textbf{gauge} on $I$ is a function$$\delta:I\rightarrow\mathbb R$$such that $\forall x\in I\ \delta(x)>0$
\subsection*{Remark}
If $\delta_1,\cdots,\delta_N$ are gauges on $I$ then $$\delta(x)=\min\{\delta_1(x),\cdots,\delta_N(x)\}$$
is also a gauge.
\section{Def}
A \textbf{partial P-division} of a compact cell $I$, is a finite $\Pi$ of pairs $(J,x)$ s.t.
\begin{itemize}
    \item $J\subseteq I$
    \item $J$ is a compact cell
    \item $x\in J$
    \item $\forall (J_1,x),(J_2,x_2)\in \Pi$ if $J_1\neq J_2$ then $J_1,J_2$ are non-overlapping
\end{itemize} 
$x$ is cal tag of the pair.
\section{Def}
Given a partial P-division $\Pi$ of I define
$$body(\Pi)=\bigcup\{J:(J,x)\in\Pi\}$$
A \textbf{P-division} $\Pi$ of $I$ is a partial P-division s.t. $body(\Pi)=I$
\section{Lemmas}
\begin{itemize}
    \item If $\Pi_1,\cdots,\Pi_N$ are partial P-divisions of I s.t. for each $n,m\in \{1,\cdots,N\}, n\neq m $ $body \Pi_n$ and $body \Pi_m$ are either disjoint or their intersection is a singleton, then $\bigcup\limits_{n=1}^N\Pi_n$ is a partial P-division of I.
    \item If $\Pi$ is a partial P-division of I and $\xi\in I$ then there're at most 2 $(J,x)\in \Pi$ s.t. $x=\xi$
\end{itemize}
\section{Def}Let $\delta$ be a gauge on I and $\Pi$ a (partial) P-division of I, we say that $\Pi$ is $\delta$-finite if$$\forall(J,x)\in \Pi\quad J\subseteq[x-\delta(x),x+\delta(x)]$$
\section{Def}
If $f:I\rightarrow\mathbb R$ and $\Pi$ is a (partial) P-division then the \textbf{Riemann sum} is defined as$$S(\Pi,f):=\sum\limits_{(J,x)\in \Pi}f(x)\abs{J}$$
\section{Def}
Let $f:I\rightarrow\mathbb R$ $f$ is \textbf{KH-integrable} on I if $\exists r\in \mathbb R, \forall \epsilon>0\exists$ gauge $\delta$ on I $\forall \delta$-finite P-division $\Pi$ of I$$\abs{S(\Pi,f)-r}<\epsilon$$
\section{Prop}$r$ is unique
\subsection*{Proof}
Assume that $r_1$ and $r_2$. Fix $\epsilon>0$. For $i=1,2$, there's a gauge $\delta_i$ on I s.t. if $\Pi$ is a $\delta_i$-finite P-division of I then $$\abs{S(\Pi,f)-r_i}<\epsilon$$
$$\begin{aligned}
    \abs{r_1-r_2}&=\abs{r_1-S(\Pi,f)+S(\Pi,f)-r_2}\\ &\leq\abs{r_1-S(\Pi,f)}+\abs{S(\Pi,f)-r_2}\\ &<2\epsilon
\end{aligned}$$
Let $\delta(x)=\min\{\delta_1(x),\delta_2(x)\}$ then $\delta$ is a gauge on I. If $\Pi$ is $\delta$-finite then it's $\delta_1$-finite and $\delta_2$-finite.
\section{Cousin Theorem}
I be a compact cell and $\delta$ a gauge on I. Then there exists a $\delta$-finite then P-division of I.
\subsection*{Proof}
assume there's no. Then divide I into $I_l,I_r$ by middle. Then either $I_l,I_r$ has no $\delta$-finite division. Then we get a decreasing sequence $(I_n)_{n\in \mathbb N}$ by keeping dividing. According to nested theorem, get their intersection a singleton $x$. Notice that $x$ is a point of I, for $N\in \mathbb N$ big enough
$$diam I_N=2^{-N}\cdot diam I<\delta(x)$$then $\Pi=\{(I_N,x)\}$ is a $\delta$-finite P-division of $I_N$.
\section{Notation}
$$r=\int_If=\int_If(x)\text{d}x$$
if $I=[a,b]$
$$r=\int_a^bf=\int_a^bf(x)\text{d}x$$
\section{Prop of Riemann Sum}
\begin{itemize}
    \item [linearity]$\forall\Pi$(partial)P-division, $\forall f_1,f_2:I\rightarrow \mathbb R,\ \forall \alpha\in \mathbb R$$$S(\Pi,\alpha f_1+f_2)=\alpha S(\Pi,f_1)+S(\Pi,f_2)$$
    \item [monotonicity]$$f_1\leq f_2\ \Rightarrow\ S(\Pi,f_1)\leq S(\Pi,f_2)$$
    \item [additivity]if $\Pi_1,\Pi_2$ are partial P-division of I and $(body \Pi_1)\cap(body\Pi_2)$ is either empty or a finite set, then $\forall f$$$S(\Pi_1\cup\Pi_2)=S(\Pi_1,f)+S(\Pi_2,f)$$
\end{itemize}
\section{Prop of KH-integral}
I a compact cell
\section{Prop: Constant functions}
If $f:I\rightarrow \mathbb R$ is constant then $f\in KH(I)$ and $\int_If=y\cdot\abs{I}.$ ($y$ is the constant value of $f$)
\subsection*{Proof}
$\forall \Pi$ P-division of I$$S(\Pi,f)=\sum\limits_{(J,x)\in \Pi}f(x)\abs{J}=y\sum\limits_{(J,x)\in \Pi}\abs{J}=y\abs{I}$$
\section{Theorem} $KH(I)$ is a vector space and $KH(I)\rightarrow\mathbb R,f\mapsto \int_If$ is linear and monotone.
\subsection*{Proof}
$0,\mathbbm 1_I\in KH(I)$

If $f_1,f_2\in KH(I)$ and $\alpha\in \mathbb R$, we want to show that $\alpha f_1+f_2\in KH(I)$ and $$\int_I(\alpha f_1+f_2)=\alpha\int_If_1+\int_If_2$$
Let $\epsilon>0$, $\delta_1$ be a gauge on I, $\frac{\epsilon}{2(\abs \alpha+1)}$-adapted to $f_1$ and $\delta_2$ $\frac{\epsilon}2$-adapted. Def $$\delta=\min\{\delta_1,\delta_2\}$$
Let $\Pi$ be a $\delta$-finite P-division of I
$$
\begin{aligned}
    \abs{S(\Pi,\alpha S(\Pi,f_1)+S(\Pi,f_2))-(\alpha\int_If_1+\int_If_2)}&=\abs{\alpha S(\Pi,f_1)+S(\Pi,f_2)-(\alpha\int_If_1+\int_If_2)}\\
    &\leq\abs{\alpha}\abs{S(\Pi,f_1)-\int_If_1}+\abs{S(\Pi,f_2)-\int_If_2}\\
    &\leq\abs{\alpha}\frac{\epsilon}{2(\abs \alpha+1)}+\frac{\epsilon}2\leq \epsilon
\end{aligned}$$
\section{Cauchy criterion}
Let $f:I\rightarrow \mathbb R$, TFAE:
\begin{itemize}
    \item $f\in KH(I)$
    \item $\forall \epsilon>0\ \exists \text{gauge}\ \delta$ on I s.t. $\forall \Pi, \Pi$ is $\delta$-finite P-division of I
    $$\abs{S(\Pi,f)-\int_If}<\epsilon$$
\end{itemize}
\subsection*{Proof}
\subsubsection{$1\Rightarrow2$}
trivial
\subsubsection{$2\Rightarrow1$}
For each $n\in \mathbb N_0$, we apply hypothesis (2) with $\epsilon=\frac{1}n$ and we obtain a gauge $\delta_n$, define
$$\hat{\delta_n}=\min\limits_{i=1}^n\delta_i$$
choose $\Pi_n$ a  $\hat{\delta_n}$-finite

Let $r_n:=S(\Pi_n,f)$. We show that $\left<r_n\right>_{n\in \mathbb N_0}$ is a Cauchy sequence. Let $0<p<q\in \mathbb N_0$
$$\abs{r_p-r_q}=\abs{S(\Pi_p,f)-S(\Pi_q,f)}<\frac{1}p$$
Name $r:=\lim\limits_{n\rightarrow +\infty}r_n$, now we show that $f$ is KH-integrable with $\int_If=r$. Let $\epsilon>0$, choose $n_0\in \mathbb N_0$ large enough for $\frac{1}{n_0}<\epsilon$. We claim that $\hat{\delta_n}$ is a gauge with integrability of $f$. $\forall \Pi\hat{\delta_n}$-finite, for each $n\geq n_0$, we have:
$$\begin{aligned}
    \abs{S(\Pi,f)-r}&\leq\abs{S(\Pi,f)-S(\Pi_n,f)}+\abs{S(\Pi_n,f)-r}\\
    &\leq\frac{1}{n_0}+\abs{r_n-r}\\
    &\leq \epsilon+\epsilon
\end{aligned}$$
\section{Example: Dirichlet function}
$$f:\mathbb R\rightarrow\mathbb R:\mathbbm{1}_{\mathbb Q}$$
Let $I$ be a compact cell, we want to show
$$f\mid_I\in KH(I)\quad\int_If\mid_i=0$$
We deal with $S(\Pi,\mathbbm 1_{\mathbb Q})=\sum\limits_{(J,x)\in \Pi,x\in \mathbb Q}\abs{J}$. For $\mathbb Q$ countable:
$$\exists q:\mathbb N\stackrel{q}{\cong}I\cap \mathbb Q$$
Let $\epsilon>0$, we define $\delta$ on $I\cap\mathbb Q$ as follows:\begin{itemize}
    \item If $x\in I\cap \mathbb Q$, then $x=q(n)$ for some $n$ and let $\delta(x)=\frac{\epsilon}{2^n}$
    \item If $x\in I\setminus\mathbb Q$, then define $\delta(x)=1$
\end{itemize}
Let $\Pi$ be $\delta$-finite,
$$\begin{aligned}
    S(\Pi,\mathbbm{1}_{\mathbb Q})&=\sum\limits_{(J,x)\in \Pi,x\in \mathbb Q}\abs{J}\\
    &\leq\sum\limits_{n=0}^\infty\sum\limits_{(J,x)\in \Pi,x\in \mathbb Q}\abs J\\
    &\leq \sum\limits_{n=0}^\infty2\cdot2\cdot\frac{\epsilon}{2^n}=8\epsilon
\end{aligned}$$
\section*{Exercises}
If $\mathbbm 1_{\mathbb Q\cap I}$ Riemann integrable?
\section{Theorem}
Let $f\in KH(I),g:I\rightarrow\mathbb R$ s.t. $\{f\neq g\}$ is countable. Then $g\in KH(I)$ and $\int_Ig=\int_If$
\section{Theorem: subordinate P-division}
Let I be a compact cell and $\Delta$ be a division of I. There exists a gauge $\delta$ on I satisfying the following properties:

$\forall \delta$-finite P-division $\Pi$ of I,
\begin{itemize}
    \item $\forall K\in \Delta,\ \exists$P-division $\Pi_K$ of K
    \item There exists a P-division $\tilde\Pi$ of I s.t.\begin{itemize}
        \item [A]$\tilde{\Pi}=\bigcup\limits_{K\in \Delta}\Pi_K$
        \item [B]$\forall f:I\rightarrow\mathbb R$$$S(\Pi,f)=S(\tilde\Pi,f)=\sum\limits_{K\in \Delta}S(\Pi_K,f_K)$$
        \item [C]For every gauge $\eta$ on I, if $\Pi$ is $\eta$-finite, then each $\Pi_K$ is $\eta\mid_K$-finite, $K\in \Delta$
    \end{itemize}
\end{itemize}
\subsection*{Proof}
$$\delta(x)=\begin{cases}
    dist(x,F)&\text{if }x\notin F\\
    dist(x,F\setminus\{x\})\text{if }x\in F
\end{cases}$$
\section{Finite-additivity}
Let $\{I_1,\cdots,I_N\}$ be a division of a compact cell I and $f:I\rightarrow \mathbb R$ TFAE
\begin{itemize}
    \item $f\in KH(I)$
    \item $f\mid_{I_n}\in KH(I_n),\forall n\in\{1,\cdots,N\}$, In this case, $$\int_If=\sum\limits_{I_n}f\mid_{I_n}$$
\end{itemize}
\subsection*{Proof}
\subsubsection{$1\Rightarrow 2$}
Let $J\subseteq I$ be a compact cell and assume $$I=J_1\sqcup J\sqcup J_2\quad(J_1\leq J\leq J_2)$$
We want to show that $f\mid_J\in KH(J)$. Apply Cauchy criterion for this. Let $\epsilon>0$ We need to find a gauge $\delta_0$ on J s.t. $\Pi_0$ is $delta_0$-finite P-division, then $$\abs{S(\Pi_0,f\mid_J)-S(\Pi_0,f\mid_J)}<\epsilon$$
For $\epsilon>0$, $\exists \delta$ gauge on I $\epsilon$-adapted to $f$. We define:
\begin{itemize}
    \item $\delta_1=\delta\mid_{J_1}$ then $\Pi_1$ is $\delta_1$-finite
    \item $\delta_0=\delta\mid_J$ then $\Pi_0$ is $\delta_0$-finite
    \item $\delta_2=\delta\mid_{J_2}$ then $\Pi_2$ is $\delta_2$-finite
\end{itemize}
so 
$$S(\Pi,f)=S(\Pi_1,f\mid_{J_1})+S(\Pi_0,f\mid_J)+S(\Pi_2,f\mid_{J_2})$$
\subsubsection{$2\Rightarrow 1$}trivial
\section{Theorem}
If $f\in KH(I)$ and $J\subseteq I$ is a compact cell, then $f\mid_J\in KH(I)$ and $$\int_Jf\mid_J=\int_I\mathbbm 1_J\cdot f$$
\section{Def: step function}
$f:I\rightarrow\mathbb R$ is a \textbf{step function} if there exists a division $\Delta$ of I s.t. $\forall J\in \Delta,f\mid_{\mathring J}$ is constant.
\section{Theorem}
Every step function on I is JH-integrable.
\section{Theorem}
If $(f_n)$ a sequence in $KH(I)$ that converges uniformly on I to $f:I\rightarrow\mathbb R$, then $f\in KH(I)$
\section{Def:regulated function}
A \textbf{regulated function} $f:I\rightarrow\mathbb R$ is a function which is a limit of a sequence of step functions.
\section{Corollary}Every regulated function on I is KH-integrable.
\section{Prop}
\begin{itemize}
    \item Every continuous function $f:I\rightarrow\mathbb R$ is regulated
    \item Every monotone function $f:I\rightarrow\mathbb R$ is regulated.
\end{itemize}
\chapter{Fundamental theorem of calculus}
\section{Theorem}
If $F:I\rightarrow\mathbb R$ is diff. (differentiable) everywhere, then $F'\in KH(I)$ and $$\int_IF'=F(\max I)-F(\min I)$$
\subsection{Lemma}
\label{Lemma 2.52}
If $f$ is diff. at $x\in I$ then$\forall\epsilon$ $\exists\delta>0$ s.t. $\forall y\leq x\leq z$, $y,z\in I$, $\max\{\abs{y-x},\abs{x-z}\}<\delta$, then$$\abs{F(z)-F(y)-F'(x)(z-y)}<\epsilon\abs{z-y}$$
\subsubsection*{Proof of lemma}$$\begin{aligned}
    &\abs{F(z)-F(x)+F(x)-F(y)-F'(x)(z-x+x-y)}\\\leq& \epsilon\abs{z-x}+\epsilon\abs{y-x}\\ =&\epsilon\abs{y-z}
\end{aligned}
$$
\subsection*{Proof}
Let $\epsilon>0$, $\forall x\in I$, there exists $\delta(x)>0$ s.t. $\forall$ compact cell $J\subseteq I$, with $x\in J\subseteq[x-\delta(x),x+\delta(x)]$
$$\abs{F(\max J)-F(\min J)-F'(x)\abs J}<\epsilon\abs J$$
If $\Pi$ is a $\delta$-finite P-division of I. We want to show $$\abs{S(\Pi,F')-F(\max I)+F(\min I)}<\epsilon\abs I$$
Basically$$S(\Pi,F')=\sum\limits_{(J,x)\in \Pi}F'(x)\abs J$$
$$F(\max I)-F(\min I)=\sum\limits_{(J,x)\in \Pi}\left(F(\max J)-F(\min J)\right)$$
$$\begin{aligned}
    &\abs{S(\Pi,F')-F(\max I)+F(\min I)}\\
    \leq &\sum\limits_{(J,x)\in \Pi}\abs{F'(x)\abs J-F(\max J)+F(\min J)}\\
    \leq &\epsilon\abs I
\end{aligned}$$
\chapter{Change of variables}
\section{Theorem: change of variable}
$$I\stackrel{\phi}{\longrightarrow}\tilde I\stackrel{f}\longrightarrow\mathbb R$$
I and $\tilde I$ be compact cells, $\phi:I\leftrightarrow \tilde I$ be a (monotone) bijection which is diff. everywhere on I. If $f\in KH(\tilde I)$ then $(f\circ\phi)\abs{\phi'}\in KH(I)$ and $$\int_I(f\circ\phi)\abs{\phi'}=\int_{\tilde I}f\abs{f'}$$
\subsection*{Proof}
Let $\epsilon>0$, exists a gauge $\tilde\delta$ on $\tilde I$ s.t. if $\tilde\Pi$ is a $\tilde\delta$-finite P-division, then $$\abs{S(\tilde\Pi,f)-\int_{\tilde I}f\abs{f}}<\epsilon$$
If $\Pi$ is any P-division, then we can associate with if $\tilde\Pi=\{(\phi(J),\phi(x))\mid(J,x)\in \Pi\}$ which is a P-division of $\tilde I$

Since $\phi$ is uniformly continuous on I, there exists $\eta:\rightbracket0,+\infty\leftbracket\rightarrow\rightbracket0,+\infty\leftbracket$ s.t. $\forall\delta>0,\forall x,y\in I$ have $$\abs{x-y}\leq\eta(\delta)\ \Rightarrow\ \abs{\phi(x)-\phi(y)}\leq\delta$$
Only a different interpretation of uniformly continuous. We define a gauge $\delta_1$ on I:
$$\delta_1(x)=\eta\circ\tilde\delta\circ\phi(x)$$
\subsection*{Remark}
If $\Pi$ is a $\delta_1$-finite P-division of I then $\tilde\Pi$ is a $\tilde\delta$-finite P-division of $\tilde I$
$$J=[y,z]\subseteq[x-\delta_1(x),x+\delta_1(x)]$$
$$\max\{\abs{y-x},\abs{x-z}\}\leq \delta_1(x)=\eta\circ\tilde\delta\circ\phi(x)$$
$$\max\{\abs{\phi(y)-\phi(x)},\abs{\phi(x)-\phi(z)}\}\leq\tilde\delta\circ\phi(x)$$

Given $x\in I$, we define $\epsilon(x)=\frac{\epsilon}{1+\abs{f\circ\phi(x)}}$. By lemma\ref{Lemma 2.52}, there exists a $\delta_2(x)>0$ s.t. if $J=[y,z]\subseteq[x-\delta_2(x),x+\delta_2(x)]\subseteq I$ contains $x$ and then 
$$\begin{aligned}
    \abs{\abs{\phi(J)}-\abs{\phi'(x)}\cdot\abs J} &= \abs{\abs{\phi(y)-\phi(z)}-\abs{\phi'(x)}\cdot\abs{z-y}}\\
    &=\abs{\phi(z)-\phi(y)-\phi'(x)(z-y)}\\
    &<\epsilon(x)\abs{z-y}=\epsilon(x)\abs J
\end{aligned} $$
Define a gauge $\delta$ on I by $\delta=\min\{\delta_1,\delta_2\}$. If $\Pi $ is a $\delta$-finite P-division of I then 
$$\begin{aligned}
    \abs{\int_{\tilde I}f-S(\Pi,(f\circ\phi)\cdot\abs{\phi'})}&\leq\abs{\int_{\tilde I}f-S(\tilde\Pi,f)}+\abs{S(\tilde\Pi,f)-S(\Pi,(f\circ\phi)\cdot\abs{\phi'})}\\
    &\leq\sum\limits_{(J,x)\in \Pi}\abs{f\circ\phi(x)}\cdot\abs{\abs{\phi(J)}-\abs{\phi'(x)}\cdot\abs J}\\
    &\leq\sum\limits_{(J,x)\in \Pi}\abs{f\circ\phi(x)}\cdot\epsilon(x)\cdot\abs{J}\\
    &\leq \epsilon\abs I
\end{aligned}$$
\chapter{Integral on the real line}
\section{Saks-Henstock's theorem}
Let I be a compact cell and $f\in KH(I)$ and $\epsilon>0$ and $\delta$ a gauge on I which is $\epsilon$-adapted to $f$. If $\Pi$ is a partial $\delta$-finite P-division of I then:
\begin{itemize}
    \item $$\abs{\sum\limits_{(J,x)\in \Pi}\left(\int_Jf\mid_J-f(x)\abs J\right)}\leq\epsilon$$
    \item $$\sum\limits_{(J,x)\in \Pi}\abs{\int_Jf\mid_J-f(x)\abs J}$$
\end{itemize}
\subsection*{Proof}
\subsubsection{$1\Rightarrow2$}
Given $\Pi$ define
$$\Pi^+=\Pi\cap\left\{(J,x)\mid\int_Jf\mid_J-f(x)\abs J\geq 0\right\}$$
$$\Pi^-=\Pi\cap\left\{(J,x)\mid\int_Jf\mid_J-f(x)\abs J<0\right\}$$
let $\pi=\Pi^+\sqcup\Pi^-$, then 
$$\sum\limits_{(J,x)\in \Pi^+}\abs{\int_Jf\mid_J-f(x)\abs J}+\abs{\sum\limits_{(J,x)\in \Pi^+}\int_Jf\mid_J-f(x)\abs J}\leq\epsilon$$
the same for $\Pi^-$
\subsubsection{prove (1)}
$\Delta_\Pi=\{J\mid(J,x)\in \Pi\}$ is a partial division of I. There exists another partial division $\Delta'$ of I s.t. $\Delta\cup\Delta_\Pi$ is a division of I.

Let $\eta>0$, $\forall K\in \Delta'$, there exists a gauge $\delta_K$ on $K$, $\eta$-adapted to $f\mid_K\in KH(K)$. Define $\tilde\delta_K(x)=\min\{\delta_K(x),\delta(x)\},x\in K$, a gauge on K. Let $\Pi_K$ be a $\delta\delta_K$-finite P-division of K. Then
$$\abs{\int_K-S(\Pi_K,f)}<\eta$$
Define $\tilde\Pi=\Pi\cup\left(\bigcup\limits_{K\in \Delta'}\Pi_K\right)$ is a P-division of I and is $\delta$-finite. Since $\delta$ is a $\epsilon$-adpated to $f$ and $\tilde\Pi$ is $\delta$-finite on I, we have:
$$\abs{\int_If-S(\tilde\Pi,f)}<\epsilon$$
$$\begin{aligned}
    S(\tilde\Pi,f)&=\sum\limits_{(J,x)\in \Pi}f(x)\abs J+\sum\limits_{K\in \Delta'}S(\Pi_K,f)\\
    \int_If&=\sum\limits_{(J,x)\in \Pi}\int_If\mid_J+\sum\limits_Kf\mid_K
\end{aligned}$$then
$$\begin{aligned}
    \abs{\sum\limits_{(J,x)\in \Pi}\int_Jf\mid_J-f(x)\abs J}&\leq\abs{\int_If-S(\tilde\Pi,f)}+\abs{\sum\limits_{K\in \Delta'}\left(\int_Kf-S(\Pi_K,f\right)}\\
    &<\epsilon+\sum\limits_{K\in \Delta'}\abs{\int_Kf-S(\Pi_K,f)}\\
    &\leq\epsilon+\eta\cdot(card\Delta')
\end{aligned}$$
\section{Hake Theorem}
\label{Hake Theorem}
Let I be a compact cell, $f:I\rightarrow\mathbb R$ and for $0<\eta<\abs I$, put $$I_\eta=[\eta+\min I,\max I]$$
TFAE
\begin{itemize}
    \item $f\in KH(I)$
    \item $\forall\eta$,$$f\mid_{I_\eta}\in KH(I_\eta)\text{ and }\lim\limits_{\eta\rightarrow0}\int_{I_\eta}f\mid_{I_\eta}\text{ exists}$$
\end{itemize}
In this case, $$\int_If=\lim\limits_{eta\rightarrow0}\int_{I_\eta}f\mid_{I_\eta}$$
\section{Corollary}
If $f\in KH(I)$, then the \textbf{indefinite integral} of $f$
$$\begin{aligned}
    F:&I&\rightarrow&\mathbb R\\
    &x&\mapsto&\begin{cases}
        \int_{[\min I,x]}f\quad&\text{if}x>\min I\\
        0&\text{if}x=\min I
    \end{cases}
\end{aligned}$$
is continuous by Hake Theorem\ref{Hake Theorem}$$\int f:=F$$
\section{Prop}
TFAE\begin{itemize}
    \item $f\in KH(I)$
    \item $\exists$ continuous function $F:I\rightarrow\mathbb R$ s.t. $\forall \epsilon>0$ $\exists$ a gauge $\delta$ on I, $\forall$ partial $\delta$finite P-division $\Pi$ of I:
    $$\sum\limits_{(J,x)\in \Pi}\abs{f(x)\abs J-F(\max J)+F(\min J)}<\epsilon$$
\end{itemize}
\section{Def: KH-integral}
A function $f:\mathbb R\rightarrow\mathbb R$ is KH-integrable if:

$\exists F:\mathbb R\rightarrow\mathbb R$ and $\lim\limits_{x\rightarrow-\infty}F(x)$ and $\lim\limits_{x\rightarrow+\infty}F(x)$ exists. $\forall \epsilon>0$ $\exists$ a gauge $\delta$ on $\mathbb R$ s.t. $\forall \Pi$ partial $\delta$-finite P-division :
$$\sum\limits_{(J,x)\in \Pi}\abs{f(x)\abs J-F(\max J)+F(\min J)}<\epsilon$$
and define
$$\int_{\mathbb R}f:=\lim\limits_{x\rightarrow+\infty}F(x)-\lim\limits_{x\rightarrow-\infty}F(x)$$
\chapter{Monotone Converges \& Lebesgue's Measure}
I be a $\textbf{closed}$ cell.
\section{Def: AKH-integrable}
$$AKH(I)=KH(I)\cap\{f\mid\abs f\in KH(I)\}$$
\section{Prop}TFAE
\begin{itemize}
    \item $f\in AKH(I)$
    \item $$f^+:=\max\{f,0\}\in KH(I)\quad f^-\min\{f,0\}\in KH(I)$$
\end{itemize}
\subsection*{Proof}
\subsubsection{$1\Rightarrow2$}
$$f^+=\frac{\abs f+f}2\quad f^-=\frac{\abs f-f}2$$
\subsubsection{$2\Rightarrow1$}
$$f=f^+-f^-\quad \abs f=f^++f^-$$
\section{Prop}Let $f\in AKH(I)$, then 
\begin{itemize}
    \item [1]$\abs{\int_If}\leq\int_I\abs f$
    \item [2]$\forall J\subseteq I$, closed cell, $$f\mid_J\in AKH(J)$$
\end{itemize}
\section{Prop}
Let $f\in KH(I)$, then $$f\in AKH(I)$$
iff
$$\sup\left\{\left.\sum\limits_{K\in \Delta}\abs{\int_Kf\mid_K}\right|\Delta\text{ is a partial division of }I\right\}<+\infty$$
\section{Theorem}
Let S be a power series with convergent radius R. Then on $D(z_0,R)$, S is holomorphic, and $$f'(z)=\sum\limits_{n=0}^\infty na_n(z-z_0)^{n-1}$$
\subsection*{Proof}
\begin{itemize}
    \item$$\limsup\limits_{n\to+\infty}\abs{a_n}^{\frac{1}n}=\limsup\limits_{n\to+\infty}\abs{na_n}^{\frac{1}{n-1}}$$ The series $f'$ exists.
    \item $\frac{\partial f}{\partial\overline z}=0$$\fpartial{f}{z}$
\end{itemize}
\end{document}