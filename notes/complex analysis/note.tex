\documentclass{book} 
\usepackage{graphicx} % Required for inserting images
\usepackage{mathrsfs}
\usepackage{amssymb}
\usepackage{amsmath}
\usepackage{indentfirst}
\usepackage{color}
\usepackage{hyperref}
\usepackage{xypic}
\usepackage{bbm}
\usepackage{xeCJK}
\usepackage{dutchcal}
\usepackage{pgfplots}
\usepackage{expl3}
\hypersetup{hidelinks,
	colorlinks=true,
	allcolors=black,
	pdfstartview=Fit,
	breaklinks=true
}
\newcommand{\abs}[1]{\left\lvert #1 \right\rvert} 
\newcommand{\norm}[1]{\left\lVert #1 \right\rVert}
\newcommand{\leftbracket}{[}
\newcommand{\rightbracket}{]}
\newcommand{\inprod}[2]{\left<#1,#2\right>}
% \newcommand{\mapping}[5][]{
% \ExplSyntaxOn
% \cs_if_exist:NTF {#1}
% {\begin{aligned}
%     #1:&#2&\rightarrow&#3\\ &#4&\mapsto&#5
% \end{aligned}}  
% {\begin{aligned}
%     &#2&\rightarrow&#3\\ &#4&\mapsto&#5
% \end{aligned}}
% \ExplSyntaxOff
% }
\begin{document}
\tableofcontents
\chapter{Preface}
\section{Ref}
\begin{itemize}
    \item Ahlfors: Complex analysis.
    \item 谭小江,伍胜健 复变函数简明教程
    \item Stein,? complex analysis.(extra exercises)
\end{itemize}
\section{A brief history of complex analysis}
Complex analysis refers studies on functions of complex variables, emerged in the 19th century. Cauchy proposed Cauchy 's integral theorem (1825) and the concept of residues. Riemann defined the Riemann Surface, which enlarge complex analysis to geometry field. Besides, he defined Riemann zeta function. And he gave Riemann mapping theorem. Weirstrass use power series to approach complex analysis.

Complex analysis also deeply connects to other filed in math.
\begin{itemize}
    \item It's essential to analysis geometry and complex geometry.
    \item Provide powerful tool to research prime numbers.
    \item In dynamics, complex dynamics is active.
    \item Deep connected with topology of 3-manifold.
    \item Deep connection with harmonic analysis(Fourier analysis).
\end{itemize}
\part{Review of learnt}
\chapter{Definition of complex numbers}
$\mathbb{R}$ denotes the real numbers. Some polynomials equation like $x^2+1=0$ has no solutions in $\mathbb{R}$. So we formally introduce the number $i$ (an imaginary number) s.t.$$i^1+1=0$$
A complex number $z=a+bi$, where $a,b\in \mathbb{R}$. Let $$\mathbb C=\{z=a+bi\mid a,b\in \mathbb R\}$$ 
$\mathbb C$ is called complex plane. The real numbers $a,b$ are called the real and imaginary part of $z$ respectively. Denoted by $\Re z,\ \Im z$

Similar with to $\mathbb R$, we can define a field structure on $\mathbb C$.\begin{itemize}
    \item[Addition]$$(a+bi)+(c+di)=(a+c)+(b+d)i$$
    \item[Multiplication]$$(a+bi)\cdot(c+di)=(ac-bd)+(ad+bc)i$$ 
\end{itemize}
To verify $\mathbb C$ a field, we need to show $\forall z\neq 0,\ \exists z^{-1}$
\section{Def: complex conjugation}Let $z\in \mathbb C$. The complex conjugation $\overline z$ of $z=a+bi$ is
$$\overline z=a-bi$$
Ones can verify are $$\overline{z+w}=\overline z+\overline w$$
$$\overline{zw}=\overline{z}\overline{w}$$
As a corollary, we consider a polynomial equation
$$a_nz^n+\cdots+a_0=0\quad a_i\in \mathbb{C}$$. If $z$ is a root, then $\overline z$ a root for:
$$\overline{a_n}z^n+\cdots+\overline{a_0}=0$$
In particular, $a_i\in\mathbb R$, then $\overline z$ is also a solution to original equation.
\section{Def:absolute value}
The absolute value of complex number $z$ is defined as:
$$\abs{z}:=\sqrt{z\cdot\overline z}=\sqrt{a^2+b^2}$$ one can verify:
$$\abs{zw}=\abs z\cdot\abs w$$
$$\abs{z+w}^2=\abs{z}^2+\abs{w}^2+2\Re(z\overline w)$$
$$\abs{z-w}^2=\abs{z}^2+\abs{w}^2-2\Re(z\overline w)$$
\section{Def: division}
Let $z_1,z_2\in \mathbb C$
$$\frac{z_1}{z_2}:=\frac{z_1\overline {z_2}}{\abs{z_2}^2}$$
In particular, if $z=a+bi$$$z^{-1}=\frac{\overline z}{\abs{z}^2}$$
\chapter{Geometry picture of complex numbers}
We can identify $\mathbb C\cong \mathbb R^2$ as $\mathbb R$-vector space, by using $z=a+bi$. We can also use the polar coordinates write $z=r(\cos\theta+i\sin\theta)$, where $r=\abs{z}$, $\theta$ is called the argument of $z$. Then conjugation flip $z$ along real axis. Addition is the same with vectors' addition. Multiplication multiplicate the length of vector and rotate the vector by the other's argument.

Consider the equation $z^n=1, \ n\geq 1$. The solution of it is called $n$-th root of unity.

\section{Some inequalities}
By the definition of absolute value
$$-\abs z\leq\Re z\leq\abs z$$
$$-\abs z\leq\Im z\leq\abs z$$
The equality $\Re z=\abs z$ iff $z$ is a non-negative real number. Since $Re(z\overline{w})\leq\abs z\abs w$ recall for $z,w \in \mathbb C$$$\abs{z+w}^2=\abs z^2+\abs w^2+2\Re(z\overline w)$$
Then we get triangle inequality:
$$\abs{z+w}\leq\abs z+\abs w$$
\subsection{Cauchy's inequality}
Let $n\geq 1$, then $$\abs{\sum\limits_{k=1}^nz_kw_k}^2\leq(\sum\limits_{k=1}^n\abs {z_k}^2)(\sum\limits_{k=1}^n\abs {w_k}^2)$$ with the equality holds iff $\exists t\in \mathbb C, \forall 1\leq k\leq n, z_k+t\overline{w_k}=0$
\subsubsection*{Proof}
Let $t\in \mathbb C$ be any complex number
$$\begin{aligned}
    0 &\leq\sum\limits_{k=1}^n\abs{z_k+t\overline {w_k}}^2=\sum\limits_{k=1}^n\abs{z_k}^2+\abs{t}^2\sum\limits_{k=1}^n\abs{w_k}^2+2\Re(\overline t\sum\limits_{k=1}^n z_kw_k)
\end{aligned}$$
choose $t=\frac{\sum\limits_{k=1}^nz_kw_k}{\sum\limits_{k=1}^n\abs{w_k}^2}$
Then we get
$$\sum\limits_{k=1}^n\abs{z_k}^2=\frac{\abs{\sum\limits_{k=1}^nz_kw_k}^2}{\sum\limits_{k=1}^n\abs{w_k}^2}\geq 0$$
The condition of equality $\Leftarrow$ the equality $0=\sum\limits_{k=1}^n\abs{z_k+t\overline {w_k}}$
\chapter{Topology and metrics on $\mathbb C$}
\section{Basic definitions}
Recall that a topology space is a set $X$ equipped with a collection of subsets of $X$ as open sets, satisfying:
\begin{itemize}
    \item $X$ and $\varnothing$ are open.
    \item Arbitrary union of open sets is open
    \item Finite intersection of open sets is open.
\end{itemize}
A closed set is by definition the complement of an open set.

A metric space is a pair $(X,d)$, where $X$ be a set and $d:X^2\rightarrow \mathbb{R}_{\geq 0}$ a mapping s.t.
\begin{itemize}
    \item $d(x,x)=0\quad\forall x\in X$
    \item $d(x,y)>0\quad\forall x\neq y\in X$
    \item $d(x,y)=d(y,x)$
    \item $d(x,y)\leq d(x,z)+d(z,y)$
\end{itemize}
let $x\in X, r>0\in \mathbb R$ the set $$\mathcal B(x,r):=\{y\in X\mid d(x,y)<r\}$$ is called an open ball. We say a subset $N\subseteq X$ is a neighborhood of $x$ if N contains an open ball centered at $x$. A subset N is open if $\forall x\in N$ N is a neighborhood of $x$
\subsection*{Remark}
For any subset $N\subseteq X$ $(N,d)$ is a metric space. The diameter of $X$:$$diam X:=\sup\limits_{x,y\in X}d(x,y)$$
X is bounded if $diam X<+\infty$. A sequence of points $x_n$ in X is called converges to $x\in X$ if $\lim\limits_n\rightarrow+\infty d(x_n,x)=0$. A sequence $(x_n)$ is called Cauchy sequence if $\forall \epsilon>0,\exists N\geq1$ s.t. $\forall n>m\geq N, d(x_n,x_m)<\epsilon$

The metric space is called complete if any Cauchy sequence converges.
\section{Notations}
$N\subseteq X$ any subset.
\begin{itemize}
    \item $\mathring N$ the interior of N, is the maximal open subset contained in N, i.e. $$\mathring N=\text{ union of all open subsets in N}$$
    \item $\overline N$ the \textbf{closure} of $N$, the minimal closed set contains $N$.
    \item $\partial N$ the \textbf{boundary} of $N$, $$\partial:=\overline N\setminus \mathring N$$let $N\subseteq X$. A point $x\in X$ is a limit point of $N$ if $x\in \overline N$ $\Leftrightarrow$ this means $\exists$ sequence $(x_n)$ in N s.t. $x_n\rightarrow x$ ($\lim\limits_{n\rightarrow+\infty}d(x_n,x)=0$)
    \item We say $x\in X$ is called an \textbf{isolated} point if $\exists$ an open ball $\mathcal B(x,r)$ s.t. $$\mathcal B(x,r)\cap X=\{x\}$$
    \item We say X is \textbf{connected} if X is not a disjoint union of non-empty open subsets.
    \item a point $x\in X$ is called a \textbf{limit point} of $N$ if $x\in \overline N$ $\Leftrightarrow$ this means $\exists$ sequence $(x_n)$ in N s.t. $x_n\rightarrow x$ ($\lim\limits_{n\rightarrow+\infty}d(x_n,x)=0$)
\end{itemize}
\chapter{Compactness}
An open cover of $X$ is a collection of open sets $\{U_\alpha\}$, $X=\bigcup\limits_\alpha U_\alpha$

X is called totally bounded if $\forall \epsilon>0, \exists$ finite open cover using $\epsilon-$radius balls. It's clear that totally bounded set is bounded.

The metric space X is called compact if every open cover of X has a finite sub-cover.
\section{Theorem}
A metric space X is compact $\Leftrightarrow$ X is complete and totally bounded.
\subsection*{Proof}
\subsubsection{$\Rightarrow$}
For completeness, assuming X isn't. Then exists a Cauchy sequence $(x_n)$ doesn't converges. Then $\forall y\in X, x_n\not\rightarrow y$. Then $\exists r>0$ s.t. $\cup_y:=\mathcal{B}(y,r)$ then $\cup_y$ contains finite many elements. Then we get an open cover $\{\cup_y\}_{y\in X}$. For X compact, we can get a finite subcover.$$X=\bigcup\limits_{y\in F}\cup_y$$where F finite.In particular $x_n\in \bigcup\limits_{y\in F}\cup_y$ so $X_n=\{x_n\mid n\in \mathbb N\}$ contains finite many elements. But finite Cauchy sequence converges. Contradiction.

For total boundence. For every $\epsilon>0, y\in X$ let $\cup_y:=\mathcal{B}(y,\epsilon)$ Then $\{\cup_y\}_{y\in X}$ is an open cover. For compactness, we get a finite subcover $X\subseteq\bigcap\limits_{y\in F}\cup_y$ so X totally bounded.
\subsubsection{$\Leftarrow$}Assume X is complete and totally bounded Assume X is not compact. Then $\exists$ open cover $\{U_\alpha\}$ s.t. $
\not\exists$ finite subcover. For totally bounded, $X=\bigcap\limits_{x\in F}\mathcal{B}(x,1)$ F finite. Then consider the index in F s.t. $\mathcal{B}(x,1)\neq\bigcup\limits_{\alpha\in E}\cup_\alpha\cap\mathcal B(x,1)$ Then exist $x_0$ s.t. $\mathcal B(x_0,1)$ can not be covered by finite many $\cup_\alpha$ So $\exists x_1\in \mathcal{B}(x_0,1)$ s.t. $\mathcal B(x_1,2^{-1})$ cannot be covered by finite many $\cup_\alpha$. Inductively, we get a sequence $(x_n)_{n\in \mathbb N}$. $$d(x_n,x_{n+1})\leq 2^{-n}$$ which means $(x_n)$ is Cauchy sequence. Moreover, $\mathcal B(x_n,2^{-n})$ can't be covered by finite many $\cup_\alpha$. By completeness, $(x_n)\rightarrow y\in X$ Let $U$ be an open set s.t. $U\in \{U_\alpha\}$ and $y\in U$. Then for $n$ large$$\mathcal B(x_n,2^{-n})\subseteq U$$contradiction.
\section{Theorem}A metric space X. Compact is equiv with Cauchy compact.
\subsection*{Proof}
\subsubsection{$\Leftarrow$}
Assume X Cauchy compact. We prove it by prove X complete and totally bounded. 

For a Cauchy sequence $(x_n)_{n\in \mathbb N}$ converges iff $\exists$ subsequence $(x_{n_i})$ converges. This means every sequence in X converges, meaning X complete.

Assume X isn't totally bounded. Then $\exists \epsilon>0$ s.t. X isn't covered by finite $\epsilon$-balls. We inductively construct a sequence $(x_n)$ as following: We choose $x_{n+1}$ s.t. $x_{n+1}\not\in \bigcup\limits_{k=1}^n\mathcal B(x_k,\epsilon)$ It doesn't have a subsequence convergent. Contradiction.
\subsubsection{$\Rightarrow$}
Assume that $\exists(x_n)_{n\in \mathbb N}$ divergent. Then $\forall y\in X,\exists r>0$ s.t. $$\cup_y:=\mathcal B(y,r)$$
$\cup_y$ contains finite points in $\{x_n\}$. Then consider the open cover $\{\cap_y\}_{y\in X}$. According to compactness, extract a finite sub-cover: $X\subseteq\bigcup\limits_{y\in F}\cup_y$. Then $\{x_n\}$ a finite set, which means $(x_n)$ has a convergent subsequence. For Cauchy sequence this implies convergence. Contradiction.
\section{Def}
Consider $X=\mathbb C$ $\forall z,w\in \mathbb C$ $$d(z,w):=\abs{z-w}$$ open balls in $\mathbb C$ is called open disks $\mathcal D(z,r)$ $$\mathbb D:=D(0,1)$$ is called unit disk.
\section{Lemma}A sequence $z_n\rightarrow z$ in $\mathbb C$$\Leftrightarrow$ \begin{itemize}
    \item $\Re z_n\rightarrow\Re z$
    \item $\Im z_n\rightarrow =\Im z$
\end{itemize}
\section{Theorem} $\mathbb C$ is complete.
\subsection*{Proof}This follows $\mathbb R$ is complete and the Lemma above.
\section{Lemma} A bounded subset in $\mathbb C$ is totally bounded.
\subsection*{Proof}
Let $K\subseteq\mathbb C$ bounded. $\exists R>0$ s.t. $K\subseteq \mathcal{D}(0,R)$. It suffice to show $\mathcal D(0,R)$ is totally bounded. It's clear, since $\mathcal{D}(0,R)$ can be covered by finitely many $\epsilon-$balls.
\section{Corollary}
A subset $K\subseteq \mathbb C$ is compact $\Leftrightarrow$ K is bounded and K is closed. 
\subsection*{Proof}K is compact $\Leftrightarrow$ K is totally bounded and complete. Since $\mathbb C$ complete, K is complete iff K is closed. Then $K$ compact $\Leftrightarrow$  K closed and bounded.
\section{Continuous mapping}
$f:X\rightarrow Y$ between metric space is continuous if $\forall$ open set $U\subseteq Y\ f^{-1}(U)$ is open

A homomorphism $f:X\rightarrow Y$ continuous and $f^{-1}$ is also continuous.
\section{Lemma}
Let $f:X\rightarrow Y$ between metric space $f$ is continuous $\Leftrightarrow$ $\forall$ sequence $(x_n)$ in X $x_n\rightarrow x\Rightarrow f(x_n)\rightarrow f(x)$
\section{Theorem}
$f:X\rightarrow Y$ continuous mapping between metric spaces. Let $K\subseteq X$ compact. Then $f(K)$ is compact in Y.
\subsection*{Proof}
Any open cover $\{V_\alpha\}$ of $f(K)$ induces an open cover $U_\alpha:=f^{-1}V_\alpha$ of K. Since K is compact, $\exists$ finite set $F$ s.t. $$K=\bigcup\limits_{\alpha\in F}U_\alpha$$
then $$f(K)=\bigcup\limits_{\alpha\in F}V_\alpha$$
so $f(K)$ is compact.
\section{Corollary}
Let X compact metric space. Let $f:X\rightarrow \mathbb R$ continuous function. Then $f(X)$ can take maximal and minimal values.
\subsection*{Proof}
$f(x)$ is compact in $\mathbb R$
\section{Theorem}$f:X\rightarrow Y$ continuous. If X is connected, then $f(X)$ is connected.
\subsection{Proof}
Assume that $f(X)=A\cup B$, with A, B non-empty and disjoint. Then $X=f^{-1}(A)\cup f^{-1}(B)$ is a union of non-empty open sets, meaning X not connected.
\section{Def}
Let $f:X\rightarrow Y$ continuous mapping $f$ is called uniformly continuous if $\forall \epsilon>0,\exists \delta>0$ s.t. if $d(x,y)<\delta$, then $d(f(x),f(y))<\epsilon$.
\section{Theorem}Let $f:X\rightarrow Y$ continuous. X compact. Then $f$ uniformly continuous.
\chapter{Path connected and homotopy}
A curve in $\mathbb C$ is a continuous mapping $\gamma:[a,b]\rightarrow \mathbb C$
\section{Def}
A subseteq $S\subseteq \mathbb C$ is called path-connected if $\forall z,w\in S,\exists \gamma:[a,b]\rightarrow S$ curve s.t. $\gamma(a)=z,\gamma(b)=w$
\section{Theorem}Let $U\subseteq \mathbb C$ open set. $U$ is connected $\Leftrightarrow$ path connected.
\subsection*{Proof}
Let $U\subseteq \mathbb C$ open $\forall z\in U$ let $$V_z:=\{\text{points} w\in U\text{ s.t.}\exists\text{curve connecting }z,w\}$$
Since every open disk is path connected, $V_z$ is open, $U\setminus V_z$ is open.
\subsubsection{$\Rightarrow$}
Assume U not path connected. Then $\exists z\in U$ s.t. $V_z\neq U$. Let $V_1:=V_z, V_2:=U\setminus V_z$. Then $V_1,V_2$ are non-empty open disjoint sets, then U not connected. Contradiction.
\subsubsection{$\Leftarrow$}
Assume U not connected. We can write $$U=V_1\cup V_2$$$V_1,V_2$ non-empty and disjoint open sets. Let $z\in V_1,w\in V_2$ and $\gamma:[a,b]\rightarrow U$ curve $\gamma(a)=z,\gamma(b)=w$. Let $$I_1:=\gamma^{-1}V_1\quad I_2:=\gamma^{-1}V_2$$ $I_1,I_2$ open non-empty and disjoint and $[a,b]\cup I_1,I_2$, telling $[a,b]$ not connected. Contradiction.


\subsection{Remark}This conclusion isn't true in general when U is not open. Consider $$S:\{i y\mid\abs y\leq 1\}\cup\{x+i\sin\frac{1}x\mid 0<x\leq 1\}$$
S closed. Try to prove:\begin{itemize}
    \item S connected
    \item S not path connected
\end{itemize}
\section{Def:homotopy}
Let $U\subseteq \mathbb C$ be an open set. Let $\gamma_0:[a,b]\rightarrow U$, $\gamma_1:[a,b]$ be two curves. A homotopy between $\gamma_0$ and $\gamma_1$ is a continuous mapping $$H:[0,1]\times[a,b]\rightarrow U$$ s.t. $$H(0,t)=\gamma_0(t)\quad H(1,t)=\gamma_1(t)$$
and $\forall s\in [0,1]$
$$H(s,a)=\gamma_0(a)\quad H(s,b)=\gamma_0(b)$$
We call $\gamma_0,\gamma_1$ are homotopic if $\exists$ such a mapping.
\section{Def}
Let $U\subseteq \mathbb C$ be a connected open set. U is called simply connected if $\forall$ two curves in U with same starting and end pts are homotopic.
\chapter{Complex value function and holomorphic function}
\section{Def:Complex valued function}
$U\subseteq \mathbb C$ a open set. \textbf{Complex value function} is a mapping $f:U\rightarrow \mathbb C$ We can view $$f=u(x,y)+i(x,y)$$ via $\mathbb C\cong\mathbb R^2, z=x+iy$. We say that $f$ is differentiable if $u,v$ are differentiable. In particular, $\exists$ partial derivatives $$\frac{\partial u}{\partial x},\frac{\partial u}{\partial y},\frac{\partial v}{\partial x},\frac{\partial v}{\partial y}$$
For $z=x+iy\in U$, define:
$$\frac{\partial f}{\partial x}(z):=\frac{\partial u}{\partial x}(z)+i\frac{\partial v}{\partial x}(z)$$
\section{Def: Differential form}
Let $U\subseteq \mathbb C$ open. A differential form is a formal sum $g\text{d}x+h+\text{d}y$, where $g,h:U\rightarrow \mathbb C$ complex valued function.

Let $f:U\rightarrow \mathbb C$ differentiable. Define $$\text{d}f:=\frac{\partial f}{\partial x}\text{d}x+\frac{\partial f}{\partial y}\text{d}y$$
\section{Prop}
Let $f,g:U\rightarrow \mathbb C$ differentiable
\begin{itemize}
    \item[Linearity] $$\text{d}(f+g)=g\text{d}f+f\text{d}g$$
    \item[Leibniz rule]$$\text{d}(fg)=\text{d}f g+g\text{d}f$$ 
\end{itemize}

$z:\mathbb C\rightarrow\mathbb C,\overline z:\mathbb C\rightarrow \mathbb C$, then 
$$\text{d}z=\text{d}x+i\text{d}y,\text{d}\overline z=\text{d}x-i\text{d}y$$
$\Rightarrow$$$\text{d}x=\frac{1}2(\text{d}z+\text{d}\overline z)\quad \text{d}y=\frac{1}{2i}(\text{d}z-\text{d}\overline z)$$
$\Rightarrow$$$\text{d}f=\frac{1}2(\frac{\partial f}{\partial x}-i\frac{\partial f}{\partial y})\text{d}z+\frac{1}2(\frac{\partial f}{\partial x}+i\frac{\partial f}{\partial y})\text{d}\overline z$$
This motivates $$\partial f:=\frac{\partial f}{\partial z}\text{d}z$$
$$\frac{\partial f}{\partial z}:=\frac{1}2(\frac{\partial f}{\partial x}-i\frac{\partial f}{\partial y})$$
Similarly$$\overline{\partial f}:=\frac{\partial f}{\partial\overline z}\text{d}\overline z$$
$$\frac{\partial f}{\partial\overline z}:=\frac{1}2(\frac{\partial f}{\partial x}+i\frac{\partial f}{\partial y})$$
\section{Def:Holomorphic functions}
Let $U\subseteq \mathbb C$ open $f:U\rightarrow \mathbb C$. Let $z\in U$ we say $f$ is complex differentiable at $z$ if$$\lim\limits_{u\rightarrow z}\frac{f(u)-f(z)}{u-z}=f'(z)$$exists.

Geometrically: in the tangent space level, $f$ acts not just like a $\mathbb R$-linear mapping, but also a $\mathbb C$-linear mapping.

If $f:\mathbb R^2\rightarrow\mathbb R^2$ is $\mathbb R$-linear, then $f$ is complex differentiable iff $\exists a\in \mathbb C$, s.t.$$f(z)=az$$
\section{Def}
Let $U\subseteq \mathbb C$ open. $f:U\rightarrow \mathbb C$ is called holomorphic if $f$ is complex differentiable at every point in U.
\section{Lemma}
$U\subseteq \mathbb C$ open. $z\in U$, $f:U\rightarrow \mathbb C$, THen $f$ is complex differentiable at $z$ iff $f$ is real differentiable and satisfies the following Cauchy-Riemann equations:$$\frac{\partial f}{\partial \overline z}(z)=0$$
\subsection*{Proof}
\subsubsection{$\Leftarrow$} We can write $f(w)-f(z)=A(w-z)_o(\abs{w-z})$, where A is a real $2\times 2$ matrix. In the coordinate $z=x+iy,f=u+iv$$$A=\begin{pmatrix}
    \frac{\partial u}{\partial x}(z)&\frac{\partial u}{\partial y}(z)\\
    \frac{\partial v}{\partial x}(z)&\frac{\partial v}{\partial y}(z)
\end{pmatrix}$$
C-R equation:
$$\frac{\partial f}{\partial \overline z}\Leftrightarrow\begin{cases}
    \frac{\partial u}{\partial x}(z)&=\frac{\partial v}{\partial y}(z)\\
    \frac{\partial u}{\partial y}(z)&=-\frac{\partial v}{\partial x}(z)
\end{cases}$$
Define $$b:=\frac{\partial u}{\partial x}(z)=\frac{\partial v}{\partial y}(z)\in \mathbb R$$
$$c:=-\frac{\partial u}{\partial y}(z)=\frac{\partial v}{\partial x}(z)\in \mathbb R$$
Let $a:=b+ci\in \mathbb C$, then $$A(z)=az$$we can write$$f(u)-f(z)=a(w-z)+o(\abs{w-z})$$$\Rightarrow$ $f'(z)$exists.
\subsubsection{$\Rightarrow$}Trivial.
\section{Corollary}
$U\subseteq \mathbb C$ open set. $f:U\rightarrow \mathbb C$ $f$ is holomorphic on U iff $f$ is real differentiable on U and the C-R equation $$\frac{\partial f}{\partial \overline z}(z)=0\quad\text{holds }\forall z\in U$$
\chapter{Conformal matrix}
Let $A\in M_{2\times 2}(\mathbb R)$ be a matrix. $A:\mathbb R^2\rightarrow \mathbb R^2$ linear mapping. The inner product $v=(x_1,y_2),w=(x_2,y_2)$$$\inprod{v}{w}:=x_1x_2+y_1y_2$$$z,w\in \mathbb C$$$\inprod{z}{w}=\Re(z\overline w)$$

Let $J:\mathbb R^2\rightarrow\mathbb R^2, z\mapsto \overline z$ be the complex conjugation matrix. $\forall z,w\in \mathbb C$$$\inprod{Jz}{Jw}=\inprod{z}w$$
reflexction w.r.t. real axis.
\section{Def}A is called a rotation matrix if $$A=\begin{pmatrix}
    \cos\theta &-\sin\theta\\\sin\theta&\cos\theta
\end{pmatrix}$$
($\Leftrightarrow$ A present inner product and $\det A>0$)
\section{Prop}
A matrix A is given by $z\mapsto az,a\in \mathbb C$ $\Leftrightarrow$ $\exists\rho\geq 0$ and a rotation matrix $R_\theta$ s.t. $A=\rho R_\theta$$$\begin{cases}
    \rho=\abs{a}\\\theta=\arg z
\end{cases}$$
\section{Polar decomposition}
Any $A\in M_{2\times 2}(\mathbb R)$ can be written as $$A= R_\theta P$$or $$A=JR_\theta P$$, where $R_\theta$ is a rotation matrix, P is positive, semi-definite matrix.
\section{Def}
$A\in M_{2\times 2}(\mathbb R)$ is called conformal if A preserves the angle between two vectors i.e. $\forall z,w\in \mathbb C$$$\frac{\inprod{Az}{Aw}}{\abs{Az}\abs{Aw}}=\frac{\inprod{z}w}{\abs{z}\abs{w}} $$
\section{Remark}Conformal matrix is invertible. A circle in $\mathbb C$ is given by $$\{z\in \mathbb C\mid \abs{z-z_0}=r\}\quad r>0,z_0\in \mathbb C$$
\section{Prop}Let $A\in M_{2\times 2}(\mathbb R)$ s.t. $\det A>0$

(such matrix are called orientation preserving)

Then the following conditions are equivalent:
\begin{itemize}
    \item [1]A is conformal
    \item [2]A is given by $z\mapsto az, a\in C$
    \item [3]A maps a circle to a circle
\end{itemize}
\subsection*{Proof}
Since $\det A>0$ by polar decomposition$$A=\begin{cases}
    R_\theta P\\JR_\theta P
\end{cases}\Rightarrow A=R_\theta P$$
$R_\theta$ rotation, P positive semi-definite.

It suffices to prove the prop for P.
$$P=R_\beta DR_\beta^{-1}$$ where $D=\begin{pmatrix}
    \lambda_1&\\ &\lambda_2
\end{pmatrix},\lambda_1,\lambda_2>0$, and $R_\beta$ rotation.

It suffice to prove the prop for D. In this case (2)$\Leftrightarrow\lambda_1=\lambda_2$ It suffices to show (1)$\Rightarrow \lambda_1=\lambda$ and (3)$\Rightarrow\lambda_1=\lambda_2$

We first show (1)$\Rightarrow\lambda_1=\lambda_2$:

$D(1)=\lambda_1,D(1+i)=\lambda_1+\lambda_2i$, $$D=\begin{pmatrix}
    \lambda_1&\\ &\lambda_2
\end{pmatrix}$$. D conformal $\Rightarrow$ $$\frac{\inprod{Dz}{Dw}}{\abs{Dz}\abs{Dw}}=\frac{\inprod{z}{w}}{\abs{z}\abs{w}}$$
Take $z=1,w=1+i$
$$\frac{\inprod{\lambda_1}{\lambda_1+\lambda_2i}}{\lambda_1\sqrt{\lambda_1^2+\lambda_2^2}}=\frac{1,1+i}{\sqrt{2}}$$
Hence
$$\frac{\lambda_1^2}{\lambda_1\sqrt{\lambda_1^2+\lambda_2^2}}=\frac{1}{\sqrt 2}$$$\Rightarrow\ \lambda_1=\lambda_2$

Next (3)$\Rightarrow\ \lambda_1=\lambda_2$. Assume D maps circle to circle. Consider $\partial D(0,\sqrt 2)$. Then the image of $\partial D(0,\sqrt 2)$ is a circle, which is central symmetry w.r.t. $D(0)=0$. So the image is a circle centred at 0. Consider pts $\sqrt 2,1+i\in \partial D(0,\sqrt 2)$. Since $$D(\sqrt 2)=\lambda_1\sqrt 2\quad D(1+i)=\lambda_1+\lambda_2i$$ we have $$\abs{\lambda_1\sqrt 2}=\abs{\lambda_1+\lambda_2i}$$
$\Rightarrow \lambda_1=\lambda_2$
\chapter{Power series}
Let $n\in \mathbb Z$ one can verify
$$\begin{aligned}
    z^n:&\mathbb C&\rightarrow&\mathbb C\\ &z&\mapsto&z^n
\end{aligned}\quad n\geq0$$
$$\begin{aligned}
    z^{-n}:&\mathbb C\setminus\{0\}&\rightarrow&\mathbb C\\ &z&\mapsto&z^{-n}
\end{aligned}\quad n\geq0$$
we have
$$\frac{\partial z^n}{\partial z}=nz^{n-1}\quad\frac{\partial z^n}{\partial\overline z}=0$$
$$\frac{\partial \overline z^{-n}}{\partial z}=0\quad\frac{\partial \overline z^n}{\partial\overline z}=nz^{n-1}$$
so $z^n$ holomorphic but $\overline z$ not holomorphic.

In following we fix $n\in \mathbb Z$, $z^n:\mathbb C\rightarrow \mathbb C$, differentiable
\section{Def}
Let $z_0\in \mathbb C$ A \textbf{power series} centered at $z_0$ is of the form $$S=\sum\limits_{n=0}^{+\infty}a_n(z-z_0)^n\quad a_n\in \mathbb C$$
Let $z\in \mathbb C$. We say that S converges at $z$ if the number series $\sum\limits_{n=0}^{+\infty}a_n(z-z_0)^n$ converges, otherwise called diverges at $z$
\section{Def}
Let $K\subseteq \mathbb C$, $z_0\in K$ and $S=\sum\limits_{n=0}^{+\infty}a_n(z-z_0)^n$ is a power series. We say $S$ is uniformly convergent on $K$ is $S(z)$ converges uniformly to a function on $K$
\section{Cauchy's criteria}
S uniformly convergent on K iff $(\forall \epsilon>0)(\exists N)(\forall n>m\geq N)(\forall z\in K)$
$$\abs{\sum\limits_{k=m}^na_k(z-z_0)^k}<\epsilon$$
\section{Corollary: Dominated convergence}
If $(\forall n)(\exists M_n\in \mathbb R_{\geq 0})(\abs{a_n(z-z_0)^n}\leq M_n)$
$$\sum\limits_{n=0}^{+\infty}a_n(z-z_0)^n<+\infty$$
then S uniformly convergent on K.
\section{Abel Theorem}
Let $S=\sum\limits_{n=0}^{+\infty}a_n(z-z_0)^n$ be a power series converges at $z\neq z_0$ Let $R:=\abs{z'-z_0}$. $\forall 0<r<R$ S uniformly converges on the closed disk $\overline D(z_0,r)$
\section{Def:convergent radius}
Let S be a power series
$$R:=\sup\left\{\abs{z-z_0}\mid S \text{ converges at }z\right\}\in [0,+\infty]$$
\section{Prop}
Let $\Omega$ be the convergent set of $S$. Then $\exists! D$ disk s.t. $\Omega\subseteq\overline D$ and $D\subseteq \Omega$
\section{Lemma}
Let S be a power series, R be the convergent radius of S. Then 
$$\frac{1}R=\limsup\limits_{n\to +\infty}\abs{a_n}^{\frac{1}n}$$
\section{Theorem}
Let S be a power series with convergent radius R. Then on $D(z_0,R)$, S is holomorphic, and $$f'(z)=\sum\limits_{n=1}^{+\infty}na_n(z-z_0)^{n-1}$$
\subsection*{Proof}
\begin{itemize}
    \item $$\limsup\limits_{n\to +\infty}\abs{a_n}^{\frac{1}n}=\limsup\limits_{n\to +\infty}\abs{na_n}^{\frac{1}{n-1}}$$The series $f'$ exists
    \item $\frac{\partial f}{\partial \overline z}=0$ complex differentiable.
    \item uniformly convergence $\Rightarrow$ $f'$ is the derivative of limit function of $f$.
\end{itemize}
\section{Prop}
Convergent radius of $(S_1+S_2)$($S_1$ and $S_2$ shares the same center)
$$R\geq\min\{R_1,R_2\}$$
\end{document}