\documentclass{book} 
\usepackage{graphicx} % Required for inserting images
\usepackage{mathrsfs}
\usepackage{amssymb}
\usepackage{amsmath}
\usepackage{indentfirst}
\usepackage{color}
\usepackage{hyperref}
\usepackage{xypic}
\usepackage{bbm}
\hypersetup{hidelinks,
	colorlinks=true,
	allcolors=black,
	pdfstartview=Fit,
	breaklinks=true
}
\newcommand{\abs}[1]{\left\lvert #1 \right\rvert} 
\newcommand{\leftbracket}{[}
\newcommand{\rightbracket}{]}
\begin{document}
\chapter{$\lambda$-terms}
\section{Def}
Let V be an infinite set (the elements of which are called variables)

We construct a set $L$ which consists of finite sequences formed with the following symbols:
\begin{itemize}
    \item elements of V
    \item left and right parenthesis()
    \item $\lambda$ (Suppose that $\lambda$(,) are distinct and do not belongs to V,e.g.$\lambda x(t)$)
\end{itemize}
in a recursive way as follows:
\begin{itemize}
    \item If $x\in V$ then $x\in L$
    \item If $t$ and $\mu$ are elements of V, then $t(\mu)\in L$
    \item If $x\in V$ $t\in L$  then $\lambda xt\in L$
\end{itemize}
e.g. $$I:=\lambda x x\in L$$

some time we omit the parenthesis$$t(\mu_1(\mu_2\cdots(\mu_n)\cdots)\cdots)$$ can be written as $$t \mu_1 \mu_2 \cdots \mu_n \cdots$$
\section{Def}
Let $\alpha\in V$ and $t\in L$ We define the free occurrences of $x$ in $t$ as
\begin{itemize}
    \item If $t=x$, the only occurrence of $x$ in $t$ is free
    \item If $t=\mu_1(\mu_2)$ the free occurrence of $x$ in $t$ are those of $x$ in $\mu_1$ and $\mu_2$
    \item If $t=\lambda y \mu,y\neq x$, the free occurrence of $x$ in $t$ are those of $x$ in $\mu$
    \item If $t=\lambda x \mu$, no occurrence of $x$ in $t$ is free
\end{itemize}
If $x$ has at least one free occurrence in $t$, we say that $x$ is a free variable of $t$

If $x$ occur in $t$ just after $\lambda$, we say that $x$ is a bound variable of $t$
\chapter{Substitutes}
\section{Def}
Let $t,t_1,\cdots,t_k$ be elements of $L$ and $x,x_1,\cdots,x_k$ be distinct variables in $V$. We define $$t<t_1/x_1,\cdots,t_k/x_k>\in L$$ as follows:
\begin{itemize}
    \item If $t=x_i$$$t<t_1/x_1,\cdots,t_k/x_k>=t_i$$
    \item If $t\in V\setminus\{x_1,\cdots,x_k\}$$$t<t_1/x_1,\cdots,t_k/x_k>=t$$
    \item If $t=\mu_1(\mu_2)$ then $$t<t_1/x_1,\cdots,t_k/x_k>=\mu_1<t_1/x_1,\cdots,t_k/x_k>(\mu_2<t_1/x_1,\cdots,t_k/x_k>)$$
    \item If $t=\lambda x_iu$$$t<t_1/x_1,\cdots,t_k/x_k>=\lambda x_i(t_1<t_1/x_1,\cdots,t_{i-1}/x_{i-1},t_{i+1}/x_{i+1},\cdots,t_k/x_k>)$$
    \item If $t=\lambda x\mu,x\not\in \{x_1,\cdots,x_k\}$$$t<t_1/x_1,\cdots,t_k/x_k>=\lambda x\mu<t_1/x_1,\cdots,t_k/x_k>$$
\end{itemize}
\underline{Reference}: Jean-Louis KrivineLambda-calculus, type and models.
\chapter{$\alpha$-equivalence}
\section{Def}
We define a binary relation $\equiv $ on $L$ in a recursive way as follows:
\begin{itemize}
    \item If $t\in V$ $t\equiv t'$ iff $t=t'$
    \item If $t=\mu_1(\mu_2)$ $t\equiv t'$ iff $\exists \mu_1'\text{ and }\mu_2'$ in $L$ such that $\mu_1\equiv \mu_1',\mu_2\equiv \mu_2'$ and $t'=\mu_1'(\mu_2')$ 
    \item if $t=\lambda x \mu$ $t\equiv t'$ iff $t'$ is of the form $t'=\lambda x'\mu'$ with $\mu<y/x>\equiv \mu'<y/x'>$ for all but finitely many $y\in V$
\end{itemize}
\section{Facts}
\begin{itemize}
    \item $\equiv$ is an equivalence relation
    \item If $t=t'$ then $t$ and $t'$ have the same length and the same free variables.
    \item Let $t,t',t_1,t_1',\cdots,t_k,t_k'$ be elements of $L$ $x_1,\cdots,x_k$ be distinct variables if $t\equiv t',t_i\equiv t_i',\forall i\in \{1,\cdots,k\}$, and no free variables of $t_1,\cdots,t_k$ is bound in $t$ and $t'$ then $$t<t_1/x_1,\cdots,t_k/x_k>\equiv t'<t_1'/x_1,\cdots,t_k'/x_k>$$
    \item $\equiv$ is $\lambda$-compatible namely\begin{itemize}
        \item if $\mu_1\equiv\mu_1',\mu_2\equiv\mu_2'$ then $\mu_1(\mu_2)\equiv\mu_1'(\mu_2')$
        \item if $t\equiv t'$ then $\lambda x t\equiv \lambda x t'$
    \end{itemize}
    Hence the constructions of $L$ induces by taking equivalence classes the following constructions on $\Lambda:=L/\equiv$\begin{itemize}
        \item For any $U_1$ and $U_2$ in $\Lambda$ with representation $\mu_1$ and $\mu_2$ respectively, we denote $U_1(U_2)$ as the equivalence class of $\mu_1(\mu_2)$
        \item $\forall x\in V,\forall T\in \Lambda$ with representation $t$, we define $\lambda xT$ as the equivalence class of $\lambda x t$
    \end{itemize}
    \item $\lambda xt\equiv\lambda yt<y/x>$ if $y$ is a variable that does not occur on $t$
    \item Let $t\in L$ and $x_1,\cdots,x_k$ be elements of $V$. $\exists t'\in L, t'\equiv t$ such that none of $x_1,\cdots,x_k$ is bound on $t'$
    \item Let $T\in \Lambda$ All elements of $T$ have the same set of free variables we call then free variables of $T$
\end{itemize}
\section{Def}
Let $T,T_1,\cdots,T_k$ be elements of $\Lambda$ $t,t_1,\cdots,t_k$ be their representations such that no bound variables of $t$ is free is $t_1,\cdots,t_k$. we define
$$T[T_1/x_1,\cdots,T_k/x_k]:=\text{ the equivalence class of } t<t_1/x_1,\cdots,t_k/x_k>$$
\subsection{Facts}
\begin{itemize}
    \item If $x_1$ is not free in $T$$$T[T_1/x_1,\cdots,T_k/x_k]=T_1[T_2/x_2,\cdots,T_k/x_k]$$
    \item Let $x_1,\cdots,x_m,y_1,\cdots,y_m$ be variables such that $x_1=y,\cdots,x_k=y_k$ and $x_1,\cdots,x_m,y_{k+1},\cdots y_n$ are distinct.
    
    Let $T,T_1,\cdots,T_m,U_1,\cdots,U_n$ be elements of $\Lambda$ $$T_i'=T_i[\mu_1/y_1,\cdots,\mu_n/y_n]$$
    Then $$T[T_1/x_1,\cdots,T_m/x_m][U_1/y_1,\cdots,U_n/y_n]=T[T_1'/x_1,\cdots,T'_m/x_m,U_{k+1}/y_{k+1},\cdots,U_n/y_n]$$ 
    \item If $i\in \{1,.\cdots,k\}$$$x_i[T_1/x_1,\cdots,T_k/x_k]=T_i$$
    
    If $x\in V\setminus\{x_1,\cdots,x_k\}$$$\ x[T_1/x_1,\cdots,T_k/x_k]=x$$(we still use $x$ to represent its equivalence class $\{x\}$) 

    If $T=\lambda xU$ $x$ is not free in $T_1,\cdots,T_k$
    $$x\not\in \{x_1,\cdots,x_k\},T[T_1/x_1,\cdots,T_k/x_k]=\lambda xU[T_1/x_1,\cdots,T_k/x_k]$$
\end{itemize}
\chapter{$\beta$-convention}
\section{Def}
We define a binary relation $\beta_0$ on $\Lambda$ as follows:
\begin{itemize}
    \item If $x\in V$ there is no $T'$ such that $x\beta _0 T'$
    \item If $T=U_1(U_2)$ $T\beta _0T'$ iff either\begin{itemize}
        \item $T'=U_1(U_2')$ with $U_2\beta _0 U_2'$
        \item $T'=U_1'(U_2)$ with $U_1\beta _0 U_1'$
        \item $U_1=\lambda xW$ $T'=W[U_2/x]$
    \end{itemize} 
\end{itemize}
We denote by $\simeq_\beta$ the smallest equivalence relation that contains $\beta_0$
\section{Def}
Let $\mathcal{T}_0$ be a set, let $\mathcal{T}$ be the free magma generated by $\mathcal{T}_0$, the composition law of which is denoted as $\rightarrow$\\
$\forall(\alpha,\beta)\in \mathcal{T}^2,\ \alpha\rightarrow\beta$ is defined as an element of $\mathcal{T}$
\begin{itemize}
    \item $\mathcal{T}$ is a set and $\rightarrow$ is a composition law of $\mathcal{T}$
    \item $\mathcal{T}_0\subseteq\mathcal{T}$ and an element of $\mathcal{T}$ is obtained by successive composition of elements of $\mathcal{T}_0$
\end{itemize}
$$\alpha_0\in \mathcal{T}_0,\beta_0\in \mathcal{T},\ \alpha_0\rightarrow\alpha_0\in \mathcal{T},\alpha_0]rightarrow\beta_0\in \mathcal{T},(\alpha_0\rightarrow\beta_0)\rightarrow\alpha_0$$
Suppose that $\mathcal{T}_n$ is constructed, we let $$\mathcal{T}_{n+1}=\mathcal{T}_n\cup\{\alpha\rightarrow\beta\mid\alpha\in \mathcal{T}_n,\beta\in \mathcal{T}_n\}$$
let $\mathcal{T}=\bigcap\limits_{n\in \mathbb{N}}\mathcal{T}_n$
\section{Def}
We call content any subset $\Gamma$ of $\Lambda\times\mathcal{T}$(if $(T,\alpha)\in \Gamma$ we write $T:\alpha\quad\Gamma\vdash T:\alpha$) that satisfies the following conditions:
\begin{itemize}
    \item If $x\in V, T\in \Lambda\quad x:\alpha,T:\beta$ then $$\lambda xT:\alpha\rightarrow\beta$$
    \item If $T\in \Lambda,U\in \Lambda$ with $T:\alpha\rightarrow\beta\ U:\alpha$ then $$T(U):\beta$$ 
\end{itemize}
\begin{table}[h]
	\centering
	\caption{comparison}
	\begin{tabular}{|c|c|c|}
		\hline
		 Type theory & Mathematic Logic & Set theory\\\hline
         $\alpha$ type & proposition & a set\\\hline
         $T:\alpha$ &proof&$t\in A$ element\\\hline
         \underline{0} \underline{1}&$\bot \ \top $&$\varnothing\ \{\varnothing\}$\\\hline
         $\alpha\rightarrow\beta$ &$A\Rightarrow B$&set of mappings from A to B\\\hline
         $(Id_\alpha=\lambda x:\alpha x)\alpha\rightarrow\alpha$&=&$\{(x,x)\mid x\in \Lambda\}$\\\hline
         $\alpha+\beta$&A or B&$A\cup B$\\\hline
         $\alpha\times\beta$&A and B&Cartesian prod $A\times B$\\\hline
         $\sum\limits_{x:\alpha}\beta(x)$&$\forall x$&disjoint sum $\coprod \limits_{x\in A}B(x)$\\\hline
        $\prod\limits_{x:\alpha}\beta(x)$&$\exists x$&$\prod\limits_{x\in A}B(x)$\\\hline
	\end{tabular}
\end{table}
\end{document}