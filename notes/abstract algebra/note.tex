\documentclass{book} 
\usepackage{graphicx} % Required for inserting images
\usepackage{mathrsfs}
\usepackage{amssymb}
\usepackage{amsmath}
\usepackage{indentfirst}
\usepackage{color}
\usepackage{hyperref}
\usepackage{xypic}
\usepackage{bbm}
\usepackage{xeCJK}
\usepackage{dutchcal}
\usepackage{pgfplots}
\hypersetup{hidelinks,
	colorlinks=true,
	allcolors=black,
	pdfstartview=Fit,
	breaklinks=true
}
\newcommand{\abs}[1]{\left\lvert #1 \right\rvert} 
\newcommand{\norm}[1]{\left\lVert #1 \right\rVert}
\newcommand{\leftbracket}{[}
\newcommand{\rightbracket}{]}
\newcommand{\inprod}[2]{\left<#1,#2\right>}
\newcommand{\zmod}[1]{\mathbb Z/#1\mathbb Z}
\newcommand{\set}[1]{\left\{#1\right\}}
\begin{document}
\tableofcontents
\chapter{preface}
\section{Aim}
\begin{itemize}
	\item abstract algebraic structures on math objects.
	\item Basic language of modern math.
\end{itemize}
\section{Ref}
\begin{itemize}
	\item Dummit \& Foote: Abstract algebra, 3rd edition.
	\item 聂灵沼\&丁石孙: 代数学引论(第二版)
\end{itemize}
\chapter{}
\section{}
for an equation: $$x^2+4x+3=0$$
\begin{itemize}
	\item [Analysis]$x^2+4x+3=0\Rightarrow(x+3)(x+1)=0\Rightarrow x=-1 $ or $ x=-3$
	\item [Algebra]Vary the coefficients, consider $ax^2+bx+c=0$ general solution is $x=\frac{-b\pm\sqrt{b^2-4ac}}{2a}$
	\item [Geometry]
	\begin{tikzpicture}
		\begin{axis}
			\addplot[color=red]{x^2+4*x+3};
		\end{axis}
	\end{tikzpicture}
\end{itemize}
For the analysis, we solve the problem itself, for algebra,, we abstract the problem (using abstract def and notations) and for geometry, we care about the graph and shapes.
\part{The integers $\mathbb{Z}$}
$$\mathbb{Z}=\{0,\pm1,\pm2,\pm3,\cdots\}$$
There are two binary operations: addition and multiplication.
\section{Addition:}
$\exists!$(exists uniquely) $0\in\mathbb{Z}$ such that $$n+0=n$$$\forall n,\exists-n\in \mathbb{Z}$ s.t. $n+(-n)=0$

and
$$n+m=m+n$$
\section{multiplication}
$\exists! 1\in \mathbb{Z}$ s.t. $$n\cdot 1=n$$
and $$m\cdot n=n \cdot m\quad \forall m,n\in \mathbb{Z}$$
Only $\pm 1$ have multiplication inverses.
\chapter{The fundamental theorem of arithmetic}
\section{Def}
For $a,b\in \mathbb{Z}$ $a$ divides $b$ (written as $a\mid b$) if
$$\exists c\text{ s.t }b=ac$$
\section{Theorem: The division algorithm}
\label{division theorem}
Let $a,b\in \mathbb{Z}$ with $b>0$. Then $\exists! (q,r)\in \mathbb{Z}^2$ such that
$$a=b\cdot q+r\text{ and }0\leq r<b$$
\subsection*{Proof}
Let $S=\{a-bk\mid k\in \mathbb{Z},a-bk\geq 0\}\subseteq \mathbb{N}$
If $0\in S$ then $b\mid a$, then $q=\frac{a}b,r=0$
Now assume $0\not\in S(\Rightarrow a\neq 0)$. Since $S\neq\varnothing$, by well ordering principle of $\mathbb{N}$, we have a smallest number, say $r=a-bq>0$. It remains to show $r<b$.
If $r\geq b$ $$a-b(q+1)=a-bq-b=r-b\geq 0$$ and$$a-b(q+1)=r-b<r$$ contradiction.

For uniqueness, assume $a=bq+r$ and $a=bq'+r'$. Suppose $r'\geq r$ then $$bq+r=a=bq'\ \Rightarrow\ b(q-q')=r'-r\geq 0$$
$\Rightarrow b\mid r'-r$ and $0\leq r'-r\leq r'<b$, thus we have
$$r'-r=0$$ so as $q=q'$
\section{Def}
\begin{itemize}
	\item $\gcd (a,b)$ is the greatest common divisor of $a$ and $b$
	\item If $\gcd (a,b)=1$ then we say $a$ and $b$ are relative prime or coprime.
\end{itemize}
\section{Corollary of \ref{division theorem}}
\label{gcd exists}
Let $a,b\in \mathbb{Z}$ no both zero, and let $c=\gcd(a,b)$. Then $\exists(xx,y)\in \mathbb{Z}^2$ such that $ax+by=c$
\subsection*{Proof}
Let $S=\{ax+by\mid(x,y)\in \mathbb{Z}^2\}\cap \mathbb{Z}_{> 0}\neq\varnothing$. Let $d=\min S$. We claim that $$d=c=\gcd(a,b)$$

First note that $c\mid a\& c\mid b$$\Rightarrow c\mid ax+by\quad \forall x,y\in \mathbb{Z}$ $\Rightarrow c\mid d$. With division algorithm, we write $$a=dq+r\quad =\leq r<d$$
Note that $r\in S$ Hence $r=0$ i.e. $d\mid a$ similarly $d\mid b\Rightarrow d\mid c$ They are positive hence $d=c$ 
\section{Def}For $a\in \mathbb{Z}\setminus\{0,\pm 1\}$\begin{itemize}
	\item $a$ is called \textbf{irreducible} in $\mathbb{Z}$, if $\forall$ factorization $a=bc$, we have $$b\in {\pm1}\text{ or }c\in {\pm1}$$
	\item $a$ is called \textbf{prime} in $\mathbb{Z}$, if $a\mid bc\Rightarrow a\mid b\text{ or }a\mid c$
\end{itemize}
\section{Euclid's Lemma}
\label{Euclid's Lemma}
In $\mathbb{Z}$, irreducible $\Leftrightarrow$ prime.
\subsection*{Proof}
\subsubsection{$\subseteq$}
Assume $a$ is irreducible and $a\mid bc$. Without loss of generality( WLOG), we assume $a>0$ and $a\not\mid b$. We show $a\mid c$ in the following way:
$$\begin{aligned}
	\left.\begin{aligned}
		a \text{irreducible}\\ a>0\\ a\not\mid b
	\end{aligned}\right\} &\Rightarrow gcd(a,b)=1\\
	&\stackrel{\text{\ref{gcd exists}}}{\Rightarrow} \exists x,y\in \mathbb{Z} s.t ax+by=1\\
	&\Rightarrow c=acx+acy=a(cx+\frac{bc}a y)\\
	&\Rightarrow a\mid c
\end{aligned}$$
\subsubsection{$\supseteq$}
Assume $a$ is prime and $a=bc$. WLOG, assume that $a\mid b$, then $$\abs{b}\stackrel{a=bc}{=}\gcd(a,b)\stackrel{a\mid b}{=}\abs{a} \Rightarrow c=\pm 1$$
\section{The fundamental theorem of arithmetic}
$\forall n\in \mathbb{Z}_{\geq 2}$ is a product of positive primes. This prime factorization is unique in the following sense:
\begin{itemize}
	\item if $n=p_1\cdots p_s$ and $n=q_1\cdots q_t$ with $p_i,q_j$ are primes. Then $s=t$ and after reordering and relabeling, $p_i=q_i\forall i$
\end{itemize}
\subsection*{Proof}
\subsubsection{}
For existence, using induction on $n$. For $n=2$, 2 is prime. Assume that the prime factorization exists for any integer $k$ that $k<n$

If $n$ is prime, done. If $n$ not a prime, using Euclid's lemma \ref{Euclid's Lemma}, $n=bc$ with $1<b<n,1<c<n$ By induction hypothesis, $n$ is also a product of primes.

For uniqueness, using induction on $l=\max\{s,t\}$ If $l=1$, $n=p_1=q_1$. If $p_s\mid q_1\cdots q_t\Rightarrow \exists i\text{ s.t. }p_s\mid q_i$ But $q_i$ is prime, so $p_s=p_i$. Reindex and we may assume $p_s=q_t$. Cancel $p_s$ with $q_t$ we get $$p_1\cdots p_{s-1}=q_1\cdots q_{t-1}$$. By induction hypothesis, $s-1=t-1$ and after reindex, $p_i=q_i\forall i$
\section{Corollary}
$\forall n\in \mathbb{Z}\setminus\{0,\pm 1\}$, $n=\pm p_1^{\alpha_1}\cdots p_s^{\alpha_s}$ with $p_i$ are primes and $\alpha_i\in \mathbb{Z}_{\geq 0}$
\chapter{Congruence in $\mathbb{Z}$}
\section{Def} Let $a,b,n\in \mathbb{Z}$ with $n>0$ $a$ is \textbf{congruent} ot $b$ \textbf{modulo} $n$, written as $$a\equiv b\mod n$$ if $n\mid a-b$
\subsection*{Remark}
\begin{itemize}
	\item It is an equivalence relation.
	\item Reflexive: $a\equiv a\mod n$
	\item Symmetric: $a\equiv b\mod n\Rightarrow b\equiv a\mod n$
	\item Transitive: $a\equiv b\mod n\& b\equiv c\mod n\Rightarrow a\equiv c\mod n$
	\item $$\begin{aligned}
		a\equiv b\mod n\\c\equiv d\mod n
	\end{aligned}\Rightarrow\begin{aligned}
		a+c\equiv b+d\mod n\\ac\equiv bd\mod n
	\end{aligned}$$
\end{itemize}

So we can have congruence class modulo $n$:
$$[a]_n:=\left\{b\in \mathbb{Z}\mid b\equiv a\mod n\right\}=a+n\mathbb{Z}$$
They are only $n$ disjoint congruence class modulo $n$:
$$[0]_n,\cdots, [n-1]_n$$
The set of congruence classes modulo $n$ is denoted as $\mathbb{Z}/n\mathbb{Z}$
\section{Lemma}
If $[a]_n=[i]_n,[b]_n=[j]_n$ thenn $$[a+b]_n=[i+j]_n\quad [ab]_n=[ij]_n\quad [a-b]_n=[i-j]_n$$
Therefore, we define the following binary operations on $\mathbb{Z}/n\mathbb{Z}$:
$$
\begin{aligned}
	&[i]_n+[j]_n:=[i+j]_n\\
	&[i]_n\cdot[j]_n:=[ij]_n
\end{aligned}
$$
We have addition and multiplication satisfying associativity law, distribution law, additive inverse.
\section{Remark}
In $\mathbb{Z}$, if $a,b$ are non-zero, then $ab\neq 0$. But in $\mathbb{Z}/n\mathbb{Z}$, $[a]_n[b]_n=[0]_n$ if $n\mid ab$.

In $\mathbb{Z}$ for $2x=1$ it have no solution. But in $\mathbb{Z}/3\mathbb{Z}$,$[2]_3x=[1]_3\Rightarrow x=[2]_3$
\section{Theorem( The structure of $\mathbb{Z}/p\mathbb{Z},\ p$ prime)}
For $p\in \mathbb{Z}_{\geq2}$. The following are equivalent(TFAE):
\begin{itemize}
	\item[1] $p$ is prime
	\item[2] $\forall  a\neq 0$ in $\mathbb{Z}/p\mathbb{Z}$, $ax=1$ has a solution in $\mathbb{Z}/p\mathbb{Z}$
	\item[3] whenever $bc=0$ in $\mathbb{Z}/p\mathbb{Z}$, $b=0$ or $c=0$
\end{itemize} 
\subsection*{Proof}
\subsubsection{$1\Rightarrow 2$}
$0\neq [a]_p\Rightarrow p\neq a$ so $\gcd(a,p)=1$ then $\exists(x,y)\in \mathbb{Z}$ s.t. $ax+py=1$. So moduloing $p$ we get $ax\equiv 1\mod p$. then $ax=1$ in $\mathbb{Z}/p\mathbb{Z}$ has a solution
\subsubsection{$2\Rightarrow 3$}
Suppose $bc=0$ in $\mathbb{Z}/p\mathbb{Z}$, WLOG, we assume $b\neq 0$ in $\mathbb{Z}/p\mathbb{Z}$, $\exists\in \mathbb{Z}/p\mathbb{Z}$ s.t. $xb=1$. $$\Rightarrow c=c\cdot 1=xbc=0$$
\subsubsection{$3\Rightarrow 1$}
$bc=0$ in $\mathbb{Z}/p\mathbb{Z}\ \Rightarrow\ p\mid bc$ Hence it follows from the define of prime.
\section{Chinese remainder theorem}
If we have $n$ and $n'$ are relative prime, $b,b'\in \mathbb Z$, then the congruence equation
$$\begin{cases}
	x\equiv b\mod n\\x\equiv b'\mod n'
\end{cases}$$
have a common solution in $\mathbb Z$, and any two solutions are congruence modulo $n\cdot n'$.
\subsection*{Proof}
$x\equiv b\mod n$$\Rightarrow$ $x=b+kn$ for some $k\in \mathbb Z$. We need to find $k$ s.t. $$b+kn=b'\mod n'$$i.e.$$kn\equiv b'-b\mod n'$$
Since $\gcd(n,n')=1$, then $\exists u,v\in \mathbb Z$ s.t. $$nu+n'v=1$$
$$b'-b=(b'-b)1=nu(b'-b)+n'v(b'-b)$$
Therefore $k=u(b'-b)$ satisfies $b+kn\equiv b'\mod n'$. If $y$ in another solution in $\mathbb Z$, then $n\mid x-y,n'\mid x-y$. We write $$x-y=nt=n't'$$ for some $t,t'\in \mathbb Z$ 
$$x-y=(x-y)1=(x-y)(nu+n'v)=nun't'+n'vnt$$
$\Rightarrow nn'\mid x-y$
\subsection*{Remark}
In other words, CRT claims that the mapping
$$\begin{aligned}
	\mathbb Z/nn'\mathbb Z&\rightarrow \mathbb Z/n\mathbb Z\times\mathbb Z/n'\mathbb Z\\
	[a]_{nn'}&\mapsto([a]_n,[a]_{n'})
\end{aligned}$$ is surjective, hence bijective.
\chapter{Rings}
\section{Def}
A \textbf{ring} is a nonempty set $R$ equipped with two binary operations (usually written as addition and multiplication) that satisfy the following:
\begin{itemize}
	\item [Addition]\begin{itemize}
		\item If $a\in R,b\in R, a+b\in R$\quad(Close for addition)
		\item $(a+b)+c=a+(b+c)$\quad(Associative)
		\item $a+b=b+a$\quad(Commutative)
		\item $\exists 0_R\in R$ s.t. $a+0_R=a$\quad(neutral element)
		\item $\forall a\in R,a+x=0_R$ has a solution in $R$\quad(additive inverse)
	\end{itemize}
	\item[Multiplication]\begin{itemize}
		\item If $a\in R,b\in R, ab\in R$\quad(Close for multiplication)
		\item $(ab)c=a(bc)$\quad(Associative)
		\item $\exists 1_R\in R$ s.t. $a\cdot 1_R=a=1\cdot a$\quad(neutral element)
		\item $a(b+c)=ab+ac$ and $(b+c)a=ba+ca$\quad(Distribution law)
	\end{itemize}
\end{itemize}
\subsection*{Warning}
\begin{itemize}
	\item A ring R cannot be empty.
	\item All rings will have the identify element.
\end{itemize}
A \textbf{commutative ring} is a ring that satisfying$$ab=ba\quad\forall a,b\in R$$
Let $S\subseteq R$ be a subset of a ring R. If S is a ring under the addition and multiplication in R, then  we say S is a \textbf{subring} of R
\subsection*{Remark}\begin{itemize}
	\item $\forall a\in R$, $a+x=0_R$ has a unique solution denoted as $-a$
	\item In R, we have $a0_R=0_R=0_Ra$
	\item $a+b=a+c\Rightarrow b=c$ and $-(a-b)=-a+b$
\end{itemize}
\section{Def}Let R be a non-trivial ring. 
\begin{itemize}
	\item An element $r\in R$ is called \textbf{unit} if $\exists s\in R$ s.t. $$rs=1_R=sr$$In this case, $s$ is called the \textbf{multiplicative inverse} of $r$
	\item We denote $R^\times$ the set of all units in R.
	\item An element $r\in R$ is called \textbf{zero-divisor} if $rs=0_R$ for some $s\neq 0\in R$ (then $0_R$ is also a zero divisor)
\end{itemize}
\subsection*{Remark}For a commutative ring R, $r\in R$ we can define $$\begin{aligned}
	\varphi_r: &R&\rightarrow&R\\ &x&\mapsto&rx
\end{aligned}$$
a mapping of sets.$$r\text{ is a unit}\Leftrightarrow\varphi_r\text{ is bijective}$$
$\Rightarrow$ $rs=1\Rightarrow \varphi_r\circ\varphi_s=Id=\varphi_s\circ\varphi_r$\\
$\Leftarrow$ $\exists s\in R$ s.t. $1=\varphi_r(s)=rs$\\
$r$ is non a zero divisor $\Leftrightarrow$ ($rs_1=rs_2\Rightarrow r(s_1-s_2)=0\Rightarrow s_1=s_2$)$\Leftrightarrow$ $\varphi_r$ is injective.
\subsection*{Example}
The only unit in $\mathbb Z$ are $\pm 1$, but ni non-zero divisor. And also 2 is neither a unit nor a zero divisor.
\begin{itemize}
	\item In $\mathbb Z/6\mathbb Z$ the zero-divisor:0,2,3,4; units: 1,5. So Any elements is either a unit or a zero-divisor in $\mathbb Z/6\mathbb Z$ (this holds for $\mathbb Z/n\mathbb Z$)
\end{itemize}
\section{Def}
\begin{itemize}
	\item A \textbf{division ring}(skew filed) is a non-trivial ring R s.t. $\forall 0_R\neq a\in R$, is a unit
	\item A non-trivial commutative ring R is an \textbf{integral domain} if it has no non-zero zero-divisor.
	\item A non-trivial commutative division ring is called a \textbf{filed}. 
\end{itemize}
\subsection*{EXample}
\begin{itemize}
	\item $\mathbb Q,\mathbb R,\mathbb C$ are fields.
	\item $\mathbb Z/p\mathbb Z$ is a filed $\Leftrightarrow$ $p$ is a prime. 
	\item $\mathbb Z$ is an integral domain, but $\mathbb Z/6\mathbb Z$ is not.
	\item Any field is an integral domain($0\neq r\in R$ $\varphi_r$ is bijective $\Rightarrow$ is injective)
	\item Real Hamilton quaternions is a division ring, but not a field.
\end{itemize}
\section{Theorem}
Every finite integral domain R is a filed.
\subsection*{Proof}
$\forall r\neq 0\in R$ we define $\varphi_r:x\mapsto rx$ is injective. But $R$ is a finite set, hence $\varphi_r$ is bijective $\Rightarrow$ $r$ is a unit.
\section{Def}Let R and S are rings. A mapping $f:R\rightarrow S$ is called a \textbf{ring homomorphism} if $$f(a+b)=f(a)+f(b)\quad f(ab)=f(a)f(b)\text{ and }f(1_R)=1_S$$ 
A ring homomorphism is called ring isomorphism if it is bijective, denoted as $\cong$
\subsection*{Remark}
\begin{itemize}
	\item We have $f(0_R)=0_S:f(0_R)=f(0_R+0_R)=f(0_R)+f(0_R)$
	\item We require $f$ sending $1_R$ to $1_S$, hence $f\equiv 0$ is not a ring morphism unless $S=0$
	\item $Id_R$ is a isomorphism 
	\item If $f:R\rightarrow S,g:S\rightarrow T$ are morphism, then $f\circ g$ also morphism.
	\item If $f:R\rightarrow S$ a ring isomorphism, so does $f^{-1}$.
	\item The image of a ring homomorphism $f:R\rightarrow S$ is a subring of S
	\item There's no morphism from $\mathbb Z$ to $\mathbb Z/n\mathbb Z$
\end{itemize}
\section{Def}
For rings R and S, we have a ring structure on the Cartesian product $R\times S$ defined coordinatewise:
$$(r_1,s_1)+(r_2,s_2)=(r_1+r_2,s_1+s_2)\quad (r_1,s_1)(r_2,s_2)=(r_1r_2,s_1s_2) $$
We have a mapping:
$$\begin{aligned}
	f:&\mathbb Z/nn'\mathbb Z&\rightarrow&\mathbb Z/n\mathbb Z\times\mathbb Z/n'\mathbb Z\\
	&[a]_{nn'}&\mapsto&([a]_n,[a]_{n'})
\end{aligned}$$It's in fact a ring morphism
$$
\begin{aligned}
	f([a]_{nn'}+[b]_{nn'})&\stackrel{\text{on }\mathbb Z/nn'\mathbb Z}{=}f([a+b]_{nn'})\\
	&\stackrel{definition}{=}([a+b]_n,[a+b]_{n'})\\
	(on \mathbb Z/n\mathbb Z and \mathbb Z/n'\mathbb Z)&=([a]_n+[b]_n,[a]_{n'}+[b]_{n'})\\
	(\text{on }\mathbb Zn\mathbb Z\times\mathbb Zn'\mathbb Z)&=([a]_n,[a]_{n'})+([b]_n,[b]_{n})\\
	&=f([a]_{nn'})+f([b]_{nn'})
\end{aligned}$$
Similarly, for $f([a]_{nn'}[b]_{nn'})=f([ab]_{nn})$. If $\gcd(n,n')=1$, by CRT, $f$ is surjective, hence bijective. We have a ring morphism:
$$\mathbb Z/nn'\mathbb Z\rightarrow \mathbb Z/n\mathbb Z\times\mathbb Z/n'\mathbb Z$$ if $\gcd(n,n')=1$
But injective is much easier:

If $f([a]_{nn'})=0\Rightarrow ([a]_n,[a]_{n'})=0\Rightarrow n\mid a,n'\mid a\Rightarrow nn'\mid a\Rightarrow [a]_{nn'}=0$
This gives an "abstract" proof of CRT.

There are 24 bijections $f:\zmod 4\rightarrow\zmod 2\times \zmod 2$. But none of theses bijections is ring morphism. For $x\in \mathbb Z/n\mathbb Z\times\mathbb Z/n'\mathbb Z$ $x+x=0$ However,$$[1]_4+[1]_4=[2]_4\neq 0\quad \text{in } \mathbb Z/4\mathbb Z$$cannot be a ring morphism.
\subsection*{Remark}
In general, if $n=p_1^{\alpha_1}\cdots p_k^{\alpha_k}$ with $p_i$ primes.
We have $$
\begin{aligned}
	\mathbb Z/n\mathbb Z&\stackrel{\cong}{\rightarrow}\mathbb Z/p_1^{\alpha_1}\mathbb Z\times\cdots\times\mathbb Z/p_s^{\alpha_s}\\
	[a]_n&\mapsto([a]_{p_1^{\alpha_1}},\cdots,[a]_{p_s^{\alpha_s}})
\end{aligned}
$$
Moreover, $[a]_n\in ^{\alpha_1}Z/n^{\alpha_1}Z$ is unit $\Leftrightarrow$ $[a]_{p_i^{\alpha_i}}$ is a unit in $\mathbb Z/p_i^{\alpha_i}\mathbb Z\ \forall i$. Therefore,
$$(\zmod n)^\times\stackrel{1=1}{\rightarrow}(\zmod{p_1^{\alpha_1}})^\times\times\cdots\times(\zmod{p_s^{\alpha_s}})^\times$$bijective.
\section{Def}
Euler $\varphi$-function is defined as:$$\varphi(n):=\#\{x<n\in \mathbb N_+\mid \gcd(x,n)=1\}$$
\section{Prop}
Euler $\varphi$-function is multiplicative:$$\varphi(n)=\stackrel{CRT}{=}\varphi(p_1^{\alpha_1})\cdots\varphi(p_s^{\alpha_s})=n(1-\frac{1}{p_1})\cdots(1-\frac{1}{p_s})$$
for $\varphi(p^\alpha)=(p^\alpha-1)-(p^{\alpha-1}-1)=p^\alpha(1-\frac{1}{p})$
\chapter{The polynomial ring}
The polynomial rings enjoy one feature that $\mathbb Z$ does not:

The notion of a root of a polynomial. $\leadsto $ The theory of equations.

\section{Def}Let R be a ring
\begin{itemize}
	\item A \textbf{polynomial with coefficients} in R is defined to be an infinite sequence $(a_n)_{n\in \mathbb N}$ such that $a_i\in R$ and only finitely many of the $a_i$ are nonzero. That's, for some index k, $a_i=0_R,\forall i>k$
	\item Two polynomial $(a_n)_{n\in \mathbb N}$ and $(b_n)_{n\in \mathbb N}$ are \textbf{equal} if they are equal as sequence; That's $a_i=b_i\forall i\geq 0$
	\item \textbf{Addition}:$$(a_n)_{n\in \mathbb N}+(b_n)_{n\in \mathbb N}=(a_n+b_n)_{n\in \mathbb N}$$
	\textbf{Multiplication}:$$(a_n)_{n\in \mathbb N}(b_n)_{n\in \mathbb N}=(\sum\limits_{i+j=n}a_ib_j)_{n\in \mathbb N}$$
\end{itemize}
\section{Theorem}
Let R be a ring, and let P be the set of all polynomials with coefficients in R. Then\begin{itemize}
	\item[1] P is a ring. If R is commutative, so as P
	\item[2] Let $\tilde{R}$ be the set of all polynomial in P of the form $(r_n)_{n\in \mathbb N}$ with $r\in R$. Then $\tilde{R}$ is a subring of P and is isomorphic to R
\end{itemize}
\subsection*{Proof}
\subsubsection{2}
Consider the mapping from R to $\tilde{R}$$$\begin{aligned}
	f:&R&\rightarrow&\tilde{R}\\ &r&\mapsto&(r,0,0,\cdots)
\end{aligned}$$
This recovers the old notion for polynomials $$(a,0,\cdots)\in \tilde R$$ will be denoted as $a$$$(0,1,0,\cdots)$$ will be denoted as $x$. Hence$$x_n=(\underbrace{0,0,\cdots}\limits_{n},1,0,\cdots)$$
\section{Theorem}
Let P be the polynomials with coefficients in R. Then P contains an isomorphic copy $\tilde R$ of R and an element $x$ s.t. \begin{itemize}
	\item [1]$ax=xa,\forall a\in \tilde R$
	\item [2]Every element in P can be written in the form $$a_0+a_1x+\cdots+a_nx^n\quad\text{for some }n\in \mathbb Z_{\geq 0}$$
	\item [3]If $a_0+a_1x+\cdots+a_nx^n=b_0+b_1x+\cdots+b_mx^m$ with $n\leq m$, then $a_i=b_i$ for $i\leq n$ and $b_i=0_R$ for $i>n$
\end{itemize}
\subsection*{Proof}
\begin{itemize}
	\item[2]If $(a_n)\in P$ then $\exists n$ s.t. $a_i=0_R$ for i>n$$(a_i)= a_0+a_1x+\cdots+a_nx^n$$
\end{itemize}
\section{Def}
\begin{itemize}
	\item Let $f(x)=a_0+a_1x+\cdots+a_nx^n\in R[x]$ with $a_n\neq 0$ Then $a_n$ is called the leading coefficients of $f(x)$ $n$ is called the degree of $f(x)$ denoted as $\deg f(x)$
	
	Remark:leading coefficients of 0 is 0, $$\deg(0_R)=-\infty$$
	\item Let $f(x),g(x)\in R[x]$ with $g(x)\neq 0_R$, we say $g(x)$ divide $f(x)$ written as $g(x)\mid f(x)$, if $f(x)=g(x)q(x)$ for some $q\in R[x]$
\end{itemize}
\section{Theorem}
Let R be a ring 
\begin{itemize}
	\item [1]For any $f(x),g(x)\in R[x]$, $$\deg(f(x)+g(x))\leq \max\{\deg f(x),\deg g(x)\}$$
	$$\deg(f(x)g(x))\leq\deg f(x)+\deg g(x)$$
	\item [2]If R is an integral domain. So as $R[x]$ and $(R[x])^\times=R^\times$
\end{itemize}
\subsection{Proof}
\begin{itemize}
	\item [1]Assume $f,g\neq 0_R$ Suppose $f(x)=a_0+a_1x+\cdots+a_nx^n,g(x)=b_0+b_1x+\cdots+b_mx^m$ with $a_n\neq 0_R,b_m\neq 0_R$ WLOG we assume $n\geq m$$$f(x)+g(x)=(a_0+b_0)+(a_1+b_1)x+\cdots+(a_n+b_m)x^n+(a_{n+1}b_{n+1})x^{n+1}+\cdots$$$\Rightarrow$$$\deg(f(x)+g(x))\leq \max\{\deg f(x),\deg g(x)\}$$ similarly for multiplication.
	\item [2]By expression of $f(x)g(x)$ the product of two nonzero polynomials in $R[x]$ is nonzero ($\deg(f(x)g(x))=\deg f+\deg g$)
	
	It's clear that $R^\times\subset(R[x])^\times$ For any $f(x)\in (R[x])^\times$, there exists $g(x)\in R[x]$, s.t. $f(x)g(x)=1_R$
	$$0\leq\deg f(x)+\deg g(x)=\deg(f(x)g(x))=\deg(1_R)=0$$ so $\deg f=\deg g=0$ meaning $$f(x)=a\in R,g(x)=b\in R$$ with $ab=1_R\Rightarrow f(x)=a\in R^\times$ 
\end{itemize}
\section{Division algorithm}
Let $k$ be a field. $f(x),g(x)\in k[x]$ with $g(x)\neq 0$ Then there exists $q(x),r(x)\in k[x]$ uniquely s.t.
$$f(x)=q(x)g(x)+r(x)$$ and $$\deg r<\deg g$$
\subsection*{Proof}
\subsubsection{Existence} If $g(x)\mid f(x)$ then $f(x)=q(x)g(x)$ for some $q(x)\in k[x]$

If $g(x)\not\mid f(x)$, then consider$$S=\{f(x)-q(x)g(x)\in k[x]\mid f(x)-q(x)g(x)\neq 0,q(x)\in k[x]\}$$
Let $r(x)=f(x)-q(x)g(x)\in S$ be a polynomial of the minimal deg. It suffices to show $$\deg r<\deg g$$
We write $$g(x)=s_0+s_1x+\cdots+s_nx^n\quad r(x)=r_0+r_1x+\cdots+r_mx^m$$with $s_n,r_m\neq 0$. 

If $\deg r\geq\deg g$, then we could define:
$$h(x)=r(x)-t_ms_n^{-1}x^{m-n}g(x)$$but $\deg h<\deg r$. By the minimal degree of $r(x)$, we have $h(x)=0$ Then $$f(x)=q(x)g(x)+t_ms_n^{-1}x^{m-n}g(x)$$$\Rightarrow\quad g(x)\mid f(x)$
\subsubsection{Uniqueness}
$f(x)=q(x)g(x)+r(x)=\tilde q(x)g(x)+\tilde r(x)$ $\Rightarrow$
$$\tilde r(x)-r(x)=(q(x)-\tilde q(x))g(x)$$$$\tilde r(x)=r(x)\quad \tilde q(x)=q(x)$$
$$\deg(\tilde r(x)-r(x))<\deg g(x)$$
$$\deg(\tilde r(x)+r(x))=\deg g+\deg(q(x)-\tilde q(x))$$
\subsubsection{Remark}The division algorithm holds for
\begin{itemize}
	\item $k$ a filed $\leadsto$ R any commutative ring
	\item $g(x)\neq 0$$\leadsto$ the leading coefficients of $g$ is unit
\end{itemize} 
\section{Corollary}
\label{Division Corollary}
Let R a commutative ring. $a\in R$ and $f(x)\in R[x]$. When we divide $f(x)$ by $x-a$, the remainder if $f(a)$
\subsection*{Proof}
We have $f(x)=q(x)(x-a)+r(x)$ with $\deg r<1$ Then $r(x)$ is a constant.

Evaluating both sides at $x=a$, we get $r=:f(a)$
\section{Def}
Let R be a commutative ring. \begin{itemize}
	\item $\forall r\in R$, the evaluation mapping is $$\begin{aligned}
		\varphi_r: &R[x]&\rightarrow &R\\
		&f(x)=\sum\limits_{i=0}^na_ix^i&\mapsto&f(r):=\sum\limits_{i=0}^na_ir^i
	\end{aligned}$$which is a ring homomorphism.
	\item If $f(r)=0_R$, then we say that $r$ is a root of $f(x)$
\end{itemize}
\section{Theorem}
Let R be an \textbf{integral domain}. Then a nonzero polynomial $f(x)\in R[x]$ of degree n has at most n roots $\in R$ counting multiplicity.
\subsection*{Proof}
If $f(a_1)=0$, then possibly applied coro\ref{Division Corollary} several times, we have$$f(x)=q_1(x)(x-a_1)^{n_1}$$ with $q_1(a_1)\neq 0$, $\deg q_1=n-n_1$. If $a_2\in R$ is another root of $f(x)$, the $$0=f(a_2)=q(a_2)(a_2-a_1)^{n_1}$$
$\Rightarrow$ $q_1(a_2)0$ so $$q_1(x)=q_2(x)(x-a_2)^{n_2}$$ with $q_2(a_2)\neq 0$ and $\deg q_2=n-n_1-n_2$.

After n application of coro\ref{Division Corollary}, the quotient becomes constant, and we write $$f(x)=c(x-a_1)^{n_1}\cdots(x-a_k)^{n_k}$$with $n=n_1+\cdots+n_k$ Since R is an integral domain, the only possible roots are $a_1,\cdots,a_k$ 

(This is a tricky proof, for $a_n$ 's existence doesn't guaranteed. But we can use induction to prove it)
\subsection*{Remark}
$f(x)=x^3$ in $\mathbb Z/8\mathbb Z[x]$ but $$f(0)=f(2)=f(4)=f(6)=0$$ in $\mathbb Z/8\mathbb Z$
\chapter{Unique factorization for the polynomial ring}
In $\mathbb Z$, we deduce the existence of the gcd and unique factorization property from the division algorithm. There results will follow identically for the polynomial ring once we have the appropriate notions of primes and gcd.
\section{Def}
Let $k$ bea non-constant polynomial $f(x)\in k[x]$ is called \textbf{irreducible} in $k[x]$ if it cannot be expressed as a product of non-constant polynomials in $k[x]$.

\section*{Remark}
The irreducibility of a polynomial depends on the field $k$. For instance, $x^2-2$ is irreducible in $\mathbb Q[x]$, but reducible in $\mathbb R[x]$.
\section{Theorem}
Given two nonzero polynomials $f(x),g(x)\in k[x]$. Let $$S:=\left\{a(x)f(x)+b(x)g(x)\in k[x]\mid a(x),b(x)\in k[x]\right\}$$
Then there's certain polynomial $d(x)\in S$ of smallest degree, and every $h(x)\in S$ is divisible by $d(x)$.\subsection*{Proof}Use well-ordering principle for $S\setminus\{0\}$
\section{Def}
Let $k$ be a field. Let $f(x),g(x)\in k[x]$ be two nonzero polynomials.
\begin{itemize}
	\item We define the greatest common divisor of $f(x),g(x)$ is $$d(x):=\gcd(f(x),g(x))$$ to be the monic polynomial in $k[x]$ satisfying:\begin{itemize}
		\item $d(x)\mid f(x)$ and $d(x)\mid g(x)$
		\item $\forall e(x)\in k[x]$ if $e(x)\mid f(x)$ and $e(x)\mid g(x)$, then $d(x)\mid e(x)$
	\end{itemize}
	\item if $$\gcd(f(x),g(x))=1$$then we say $f(x)$ and $g(x)$ are relative prime.
\end{itemize}
\section{Theorem}
Let $k$ be a field. 
\begin{itemize}
	\item Suppose that $f(x)$ is irreducible in $k[x]$ and $$f(x)\mid g(x)h(x)$$ then$$f(x)\mid g(x)\text{ or }f(x)\mid h(x)$$
	\item (unique factorization in $k[x]$) Every common non-constant polynomial $f(x)\in k[x]$ can be written as a product of irreducible polynomials in $k[x]$; the resulting expression is unique, except for rearrangement and non-zero constant factors.
\end{itemize}
\subsection*{Proof}
As before in $\mathbb Z$
\section*{Remark}
We have pointed out how similar the situation is for $\mathbb Z$ and $k[x]$. This suggests that there should be a wider class of rings, of which $\mathbb Z$ and $k[x]$ are special cases, for which much of the argumentation works.
\section{Def}\label{Def:Euclidean domain}
An integral domain R is a \textbf{Euclidean domain}( or \textit{Euclidean ring}) if there is a function $d:R^*:=R\setminus\set 0\to\mathbb Z_{\geq 0}$ s.t.\begin{itemize}
	\item[i] For $a\in R^*,b\in R^*$, have $$d(a)\leq d(ab)$$
	\item[ii] For $a\in R,b\in R^*$, there exists $q,r\in R$ s.t.$$a=qb+r$$ where $r=0$ or $d(r)<d(b)$
\end{itemize}
\section{Example}
The following are standard examples, we have seen before:
\begin{itemize}
	\item The ring of integers $\mathbb Z$ with $d(n)=\abs n$ the absolute value of $n$. Note that when we use the absolute value, then the quotient $q$ and the reminder $r$ are not unique anymore. For instance $51=6\times 8+3=6\times 9-3$
	\item The polynomial ring $k[x]$ with coefficients in a field $k$, and $d$ is given by degree.
	\item The ring of Gaussian integers $$\mathbb Z[i]=\set{a+bi\mid a,b\in \mathbb Z}$$ with $$d(a+bi)=a^2+b^2$$
\end{itemize}
\subsection*{Remark}
\subsubsection{1}
In Definition\ref{Def:Euclidean domain}, (i) is not essential. Indeed, given $(R,d)$ which only satisfies (ii) there, we can define $\tilde d$ as
$$\tilde d(a)=\min\set{d(ab)\mid b\in R^*}$$
for any $a\in R^*$. Then $(R,\tilde d)$ is a Euclidean domain defined as in Def\ref{Def:Euclidean domain}(Exercise)
\subsubsection{2}
Division in $\mathbb Z[i]$ does not have a unique quotient and a remainder. For example:
$$1+8i=(2-4i)(-1+i)-1+2i$$
$$1+8i=(2-4i)(-2+i)+1-2i$$
which both remainders have norm 5 and less then $N(2-4i)=20$
\subsubsection{3}
We have the following fact: If R is a Euclidean domain where the quotient and remainder are unique, then R is a field or $R=k[x]$ for a field $k$.
\chapter{Ideals and Quotients}
We start from a special kind of ideals.
\section{Def}
Let $f:R\to S$ be a ring homomorphism between two rings. Then kernel of $f$ is denoted as $$\ker f=\set{a\in R\mid f(a)=0_S}$$

Note that, if $a_1,a_2\in \ker f$, then $a_1\pm a_2\in \ker f$, $r_1,r_2\in \ker f$ and $r_2,r_1\in \ker f$. This set is almost to be a ring but $1_R\notin \ker f$. But this set is more special, it satisfies a absorptive properties: if $a_1\in \ker f$ and $a\in R$, then we have:
$$f(aa_1)=f(a)f(a_1)=f(a)\cdot0_S=0_S$$
hence $aa_1\in \ker f$, similarly, we also have $a_1a\in \ker f$. This type subset has its own name.
\section{Def}
Let R be a ring, let $I\subseteq R$ be a subset. I is called an ideal of R if it satisfies:
\begin{itemize}
	\item $0\in I$
	\item if $a_1,a_2\in I$, then $$a_1+a_2\in I$$
	\item if $a_1\in I$, $\forall a\in R$, then $$aa_1\in I\text{ and }a_1a\in I$$
\end{itemize}
\subsection*{Example}
\begin{itemize}
	\item If $b\in I$, then $-b=(-1_R)\cdot b\in I$. Therefore $a-b\in I$ if $a,b\in I$
	\item Let $\varphi_n:\mathbb Z\to\mathbb Z/n\mathbb Z$ be the natural ring homomorphism $a\mapsto[a]_n$. Then, $$\ker(\varphi_n)=n\mathbb Z$$is an ideal.
	\item $\forall f(x)\in R[x]$ be a polynomial with coefficients in a ring R, then $$f(x)R[x]:=\set{f(x)g(x)\mid g(x)\in R[x]}$$
	is an ideal of $R[x]$
\end{itemize}
\section{Def}
Let R be a commutative ring
\begin{itemize}
	\item $\forall a\in R$, $aR:=\set{ab\mid b\in R}$ is an ideal of R, denoted as $(a)$, called the principal ideal generated by $a$
	\item Let $a_1,\cdots, a_n\in R$. The ideal generated by $a_1,\cdots,a_n$ is denoted as $$(a_1,\cdots,a_n):=\set{r_1a_1+\cdots+r_na_n\mid r_i\in R,\forall i}$$
\end{itemize}
\section{Example}
\begin{itemize}
	\item Given $m,n\in \mathbb Z$, the ideal generated by $(m,n)$ is $$\set{mx+ny\mid x,y\in \mathbb Z}=(gcd(m,n))$$which is a principal ideal.
	\item Let R be a commutative ring, consider the evaluation map at $a\in R$:$$\begin{aligned}
		ev_a:&R[x]&\to&R\\ &f(r)&\mapsto&f(a)
	\end{aligned}$$The kernel of this mapping is given by$$\begin{aligned}
		\ker(ev_a)&=\set{f(x)\in R[x]\mid f(a)=0_R}\\ &=\set{f(x)\in R[x]\left| x-a\mid f(x)\right.}
	\end{aligned}$$
	which is just the ideal $(x-a)$. It also a principal ideal.
\end{itemize}
\section{Def}
Let R be a ring, $I\subseteq R$ be an ideal. For $a,b\in R$, we say $a$ is congruent to $b$ modulo I, written as $a\equiv b\mod I$ if $a-b\in I$
\section{Lemma}
Let R be a ring, $I\subseteq R$ be an ideal.\begin{itemize}
	\item Congruence modulo I is an equivalence relation
	\item If $a\equiv b\mod I$ and $c\equiv d\mod I$, then $a\pm c\equiv b\pm d\mod I$. That is, this congruence relation is compatible with addition and multiplication on R.
\end{itemize}

\end{document}