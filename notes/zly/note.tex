\documentclass{book}
\usepackage{graphicx} % Required for inserting images
\usepackage{mathrsfs}
\usepackage{amssymb}
\usepackage{amsmath}
\usepackage{indentfirst}
\usepackage{color}
\usepackage{hyperref}  
\hypersetup{hidelinks,
	colorlinks=true,
	allcolors=black,
	pdfstartview=Fit,
	breaklinks=true
}

\begin{document}
\tableofcontents
\part{Set}
\chapter{Ring}
\section{morphism}
    \subsection*{Def}
    \indent Let A and B be unitary rings .We call morphism of unitary rings from A to B .only mapping $A\rightarrow B$is a morphism of group from (A,+) to (B,+),and a morphism of monoid from $(A,\cdot)\ to\ (B,\cdot)$
    \subsection*{Properties}
    \begin{itemize}
        \item Let R be a unitary ting. There is a unique morphism from $\mathbb{Z}\ to\ R$
        \item 
    \end{itemize}
\subsection*{algebra}

we call k-algebra any pair(R,f),when R is a unitary ring ,and $f:k\rightarrow R$ is a morphism of unitary rings such that $\forall (b,x)\in k\times R,f(b)x=xf(b)$

Example:    For any unitary ring R,the unique morphism of unitary rings $\mathbb{Z}\rightarrow R$ define a structure of $\mathbb{Z}-algebra$ on R (extra: $\mathbb{Z}$ is commutative despite R isn't guaranteed)

Notation: Let k be a commutative unitary ring ,(A,f) be a k-algebra. If there is no ambiguity on f,for any$ (\lambda,a )\in k\times A$,we denote $f(\lambda)a\ as\ \lambda a$
\subsection*{Formal power series}
reminder: $n\in\mathbb{N}$ is possible infinite ,so $\sum\limits_{n\in \mathbb{N}}$ couldn't be executed directly.
Def:

(extended polynomial actually)
Let k be a commutative unitary ring.
Def : Let T be a formal symbol.We denote $k^\mathbb{N}$ as $k\mathbb{[}T\mathbf{]}$ If $(a_n)_{n\in\mathbb{N}}$ is an element of $k^\mathbb{N}$,when we denote $k^\mathbb{N}$ as k[T] this element is denote as $\sum_{n\in\mathbb{N}}a_nT^n$ Such element is called a formal power series over k and $a_n$ is called the Coefficient of $T^n$ of this formal power series
Notation: \begin{itemize}
    \item omit terms with coefficient O
    \item write T' as T
    \item omit Coefficient those are 1;
    \item omit $T^0$
\end{itemize}
Example $1T^0+2T^1+1T^2+0T^3+...+0T^n+...$ is written as $1+2T+T^2$

Def Remind that k[T]=$\{\sum_{n\in\mathbb{N}}a_nT^n\mid (a_n)_{n\in\mathbb{N}}\in k^\mathbb{N} \}$,define two composition laws on k[T]\\
$\forall F(T)=a_0+a_+1T+...\quad G(T)=b_0+...$\\
let $F+G=(a_0+b_0)+...$\\
$FG=\sum\limits_{n\in\mathbb{N}}\sum\limits_{i+j=n}(a_ib_j)T^n$\\
Properties:\begin{itemize}
\item $(k[T],+,\cdot)$form a commutative unitary ring.
\item $k\rightarrow k[T]\quad \lambda\mapsto\lambda T$ is a morphism 
\item $(FG)H=\left(\sum_{n\in\mathbb{N}}^{}\sum\limits_{i+j=n}(a_ib_j)T^n\right)(\sum\limits_{n\in\mathbb{N}}c_nT^n)=\sum\limits_{n\in\mathbb{N}}\left(\sum\limits_{p,q,l=n}a_pb_qc_l\right)T^n$\\is a trick applied on integral
\end{itemize}
Derivative:

let $F(T)\in k[T]$\\
\indent We denote by $F'(T)\ or\ \mathcal{D}(F(T))$ the formal power series\\
\indent$\mathcal{D}(F)=\sum\limits_{n\in\mathbb{N}}(n+1)a_{n+1}T^n$\\
Properties:\begin{itemize}
    \item $\mathcal{D}(k[T],+)\rightarrow(k[T],+)$ is a morphism of groups
    \item $\mathcal{D}(FG)=F'G+FG'$
\end{itemize}
exp

We denote $exp(T)\in k[T]$ as $\sum\limits_{n\in\mathbb{N}}{1\over n!}T^n$,which fulfil the differential equation $\mathcal{D}(exp(T))=exp(T)$(interesting)\\
Cauchy sequence: $(F_i(T))_{i\in\mathbb{N}}$ be a sequence of elements in k[T],and $F(T)\in k[T]$We say that $(F_i(T))_{i\in \mathbb{N}}$ is a Cauchy sequence if $\forall l \in \mathbb{N}$,there exists $N(l)\in \mathbb{N}$ such that $\forall(i,j)\in\mathbb{N}^2_{\geq N(l)},ord(F_i(T)-F_j(T))\geq l$
\part{Sequences}
\chapter{Supremum and infimum}
Def:

Let $(X,\le)$ be a partially ordered set A and Y be subsets of X,such that $A\subseteq Y$
\begin{itemize}
    \item If the set $\{y\in Y\mid \forall a \in A,a\leq Y\}$has a least element then we say that A has a Supremum in Y with respect to $\leq$ denoted by $sup_{(y,\leq)}A$ this least element and called it the Supremum of A in Y(this respect to $\leq$)
    \item If the set $\{y\in Y\mid\forall a\in A,y\leq a\}$ has a greatest element, we say that A has n infimum in Y with respect to $\leq$ .We denote by $inf_{(y,\leq)}A$ this greatest element and call it the infimum of A in Y 
    \item Observation: $inf_{(Y,\leq)}A=sup_{(Y,\geq)}A$
\end{itemize}
Notation:

Let $(X,\leq)$ be a partially ordered set,I be a set.
\begin{itemize}
    \item If f is a function from I to X sup f denotes the supremum of f(I) is X.$\inf f $takes the same
    \item If $(x_i)_{i\in I} is \ a\ family\ of\ element\ in\ X,then\ \sup \limits_{i\in I}x_i\ denotes\ \sup\{x_i\mid i\in I\} (in X)$
\end{itemize}

If moreover $\mathbb{P}(\cdot)$ denotes a statement depending on a parameter in I then $\sup\limits_{i\in I,\mathbb{P}(i)}x_i$ denotes $\sup\{x_i\mid i\in I,\mathbb{P}(i)\ holds\}$\\
Example:

Let $A={x\in R\mid 0\leq x<1}\subseteq \mathbb{R}$ We equip $\mathbb{R}$ with the usual order relation.$$\{y\in \mathbb{R}\mid \forall x\in A,x\leq y\}=\{y\in\mathbb{R}\mid y\geq 1\}$$
So $\sup A=1$
$$\{y\in \mathbb{R}\mid \forall x\in A,y\leq x\}=\{y\in\mathbb{R}\mid y\geq 0\}$$
Hence $\inf A=0$\\
Example: 
For $n\in \mathbb{N}$,let $x_n=(-1)^n\in R$ $$\sup\limits_{n\in \mathbb{N}}\inf\limits_{k\in \mathbb{N},k\geq n}x_k=-1$$
Proposition:

Let $(X,\leq)$ be a partially ordered set,A,Y,Z be subset of X,such that $A\subseteq Z\subseteq Y$
\begin{itemize}
    \item If max A exists,then is is also equal to $\sup_{(y,\leq)}A$
    \item If $\sup_{(y,\leq)}A$ exists and belongs to Z, then it is equal to $\sup A$
\end{itemize}

$\inf$ takes the same
Prop.

Let $X,\leq$ be a partially ordered set ,A,B,Y be subsets of X such that $A\subseteq B\subseteq Y$
\begin{itemize}
    \item If $\sup_{(y,\leq)}A$ and $\sup_{(y,\leq)}B$ exists,then $\sup_{(y,\leq)}A \leq \sup_{(y,\leq)}B$
    \item If $\inf_{(y,\leq)}A$ and $\inf_{(y,\leq)}B$ exists,then $\inf_{(y,\leq)}A \geq \inf_{(y,\leq)}B$
\end{itemize}
Prop.

Let $(X,\leq)$be a partially ordered set ,I be a set and $f,g:I\rightarrow X$ be mappings such that $\forall t\in I,f(t)\leq g(t)$
\begin{itemize}
    \item If $\inf f$ and $\inf g$ exists,then $\inf f\leq \inf g$ 
    \item If $\sup f$ and $\sup g$ exists,then $\sup f\leq \sup g$ 
\end{itemize}
\chapter{Interval}
We fix a totally ordered set $(X,\leq)$\\
Notation:

If $(a,b)\in X\times X $ such that $a\leq b $,[a,b] denotes $\{x\in X\mid a\leq x\leq b\}$\\
Def:

Let $I\subseteq X $. If $\forall(x,y)\in I\times I$ with $x\leq y$,one has $[x,y]\subseteq I$ then we say that I is a interval in X\\
Example:

Let $(a,b)\in X\times X$, such that $a\leq b$ Then the following sets are intervals
\begin{itemize}
    \item $]a,b[:=\{x\in X\mid a,x,b\}$
    \item $[a,b[:=\{x\in X\mid a,x,b\}$
    \item $]a,b]:=\{x\in X\mid a,x,b\}$
\end{itemize}
Prop. 

Let $\Lambda$ be a non-empty set and $(I_\lambda)_{\lambda\in \Lambda}$ be a family of intervals in X.
\begin{itemize}
    \item $\bigcap\limits_{\lambda\in\Lambda}I_\lambda$ is a interval in X
    \item If $\bigcap\limits_{\lambda\in\Lambda}I_\lambda\not=\varnothing $,$\bigcup\limits_{\lambda\in\Lambda}I_\lambda$ is a interval in X
\end{itemize}
We check that $[a,b]\subseteq I_\lambda\cup I_]\mu$
\begin{itemize}
    \item If $b\leq x$\quad $[a,b]\subseteq[a,x]\subseteq I_\lambda$ because $\{a,x\}\subseteq I_\lambda$
    \item If $x\leq a$\quad $[a,b]\subseteq[x,b]\subseteq I_\mu $ because $\{b,x\}\subseteq I_\mu$
    \item If a $<$ x $<$ b then $[a,b]=[a,x]\cup[x,b]\subseteq I_\lambda\cup I_\mu$
\end{itemize}
Def:

Let $(X,\leq)$ be a totally ordered set .I be a non-empty interval of X. If $\sup I$ exists in X, we call $\sup I$ the right endpoint; inf takes the similar way.\\
Prop.

Let I be an interval in X.
\begin{itemize}
    \item Suppose that $b=\sup I$exists. $\forall x\in I,[x,b[\subseteq I$
    \item Suppose that $a=\inf I$exists. $\forall x\in I,]a,x]\subseteq I$
\end{itemize}
Prop.

Let I be an interval in X. Suppose that I has supremum b and an infimum a in X.Then I is equal to one of the following sets $[a,b]\ \ [a,b[\ \ ]a,b]\ \ ]a,b[$\\
Def 

let $(X,\leq)$ be a totally ordered set .If $\forall (x,z)\in X\times X$,such that $x<z\quad \exists y\in X$ such that $x<y<z$,than we say that $(X,\leq)$ is thick\\
Prop. 

Let $(X,\leq)$ be a thick totally ordered set. $(a,b)\in X\times X,a<b$ If I is one of the following intervals $[a,b];[a,b[;]a,b];]a,b[$ Then $\inf I=a\quad \sup I=b$ (for it's thick empty set is impossible)\\
Proof:

Since X is thick,there exists $x_0\in]a,b[$ By definition,b is an upper bound of I. If b is not the supremum of I,there exists an upper bound M of I such that M<b. Since X is thick ,there is $M'\in X$ such that $x_0\leq M,M'<b$ Since $[x,b[\subseteq]a,b[\in I$ Hence M and M' belong to I,which conflicts with the uniqueness of supremum.
\chapter{Enhanced real line}
Def:

Let $+\infty$ and $-infty$ be two symbols that are different and don not belong to $\mathbb{R}$ We extend the usual total order $\leq\ on\ \mathbb{R}\ to\ \mathbb{R}\cup\{-\infty,+\infty\}$ such that $$\forall x\in \mathbb{R} ,-\infty<x<+\infty$$ 
\indent Thus $\mathbb{R} \cup\{-\infty,+\infty\}$ become a totally ordered set, and $\mathbb{R} =]-\infty,+\infty[$  Obviously,this is a thick totally ordered set.\\
We define:
\begin{itemize}
    \item $\forall x\in ]-\infty,+\infty]\quad x+(+\infty):=+\infty \quad (+\infty)+x:=+\infty$
    \item $\forall x\in [-\infty,+\infty[\quad x+(-\infty):=-\infty\quad (-\infty)+x=-\infty$
    \item $\forall x\in ]0,+\infty]\quad x(+\infty)=(+\infty)x=+\infty\quad x(-\infty)=(-\infty)x=-\infty$
    \item $\forall x\in [-\infty,0[\quad x(+\infty)=(+\infty)x=-\infty\quad x(-\infty)=(-\infty)x=+\infty$
    \item $-(+\infty)=-\infty\quad -(-\infty)=+\infty\quad (\infty)^{-1}=0$
    \item $(+\infty)+(-\infty)\quad (-\infty)+(+\infty)\quad (+\infty)0\quad 0(+\infty)\quad(-\infty)0\quad 0(-\infty)$ \\\textcolor{red}{ARE NOT DEFINED}
\end{itemize}
Def 

Let $(X,\leq)$ be a partially ordered set. If for any subset A of X,A has a supremum and an infimum in X, then we say the X is order complete\\
Example

Let $\Omega$ be a set $(\mathscr{P}(\Omega),\subseteq)$is order complete If $\mathscr{F}$is a subset of $\mathscr{P}(\Omega),\sup\mathscr{F}=\bigcup\limits_{A\in \mathscr{F}}A$

Interesting tip: $\inf \varnothing =\Omega\quad \sup \varnothing=\varnothing$\\
$\mathcal{AXION}:$\\\indent$(\mathbb{R}\cup\{-\infty,+\infty\},\leq)$ is order complete \\\indent In $\mathbb{R}\cup\{-\infty,+\infty\}\quad \sup\varnothing=-\infty\quad\inf\varnothing=+\infty$\\
Notation:
\begin{itemize}
    \item For any $A\subseteq \mathbb{R} \cup{-\infty,+\infty}\ and\ c\in \mathbb{R} $ We denote by $A+c$ the set $\{a+c\mid a\in A\}$
    \item If $\lambda\in\mathbb{R} \backslash\{0\},\lambda A$ denotes $\{\lambda a\mid a\in A\}$
    \item -A denotes (-1)A
\end{itemize}
Prop.

For any $A\subseteq\mathbb{R} \cup\{-\infty,+\infty\},\sup(-A)=-\inf A,\inf (-A)+-\sup A$
Def 

We denote by $(R,\leq)$ a field $\mathbb{R} $ equipped with a total order $\leq$, which satisfies the following condition:
\begin{itemize}
    \item $\forall (a,b)\in \mathbb{R} \times \mathbb{R} $ such that $a<b$ ,one has $\forall c\in \mathbb{R} ,a+c<b+c$
    \item $\forall(a,b)\in\mathbb{R}_{>0}\times\mathbb{R}_{>0},ab>0$
    \item $\forall A\subseteq \mathbb{R} $,if A hsa an upper bound in$\mathbb{R}$ ,then it has a supremum in$\mathbb{R} $
\end{itemize}
Prop.

Let $A\subseteq[-\infty,+\infty]$
\begin{itemize}
    \item $\forall c\in \mathbb{R} \quad \sup(A+c)=(\sup A)+c$
    \item $\forall \lambda\in \mathbb{R}_{\geq0}\quad \sup(\lambda A)=\lambda\sup(A)$
    \item $\forall\lambda\in \mathbb{R} _{\leq0}\quad sup(\lambda A)=\lambda\inf(A)$
\end{itemize}

\indent $\inf$ takes the same\\
Theorem:

Let I and J be non-empty sets\\
\indent$f:I\rightarrow[-\infty,+\infty],g:J\rightarrow[-\infty,+\infty]$\\
\indent$a=\sup\limits_{x\in I}f(x)\quad b=\sup\limits_{y\in J}g(y)\quad c=\sup\limits_{(x,y)\in I\times J,\{f(x),g(y)\}\not=\{+\infty,-\infty\}}(f(x)+g(y))$ \\
\indent If $\{a,b\}\not=\{+\infty,-\infty\}$then $c=a+b$\\
\indent $\inf$ takes the same if $(-\infty)+(+\infty)$ doesn't happen\\
Corollary:

Let I be a non-empty set,$f:I\rightarrow[-\infty,+\infty],g:J\rightarrow[-\infty,+\infty]$\\
Then $\sup\limits_{x\in I,\{f(x),g(y)\}\not=\{+\infty,-\infty\}}(f(x)+g(x))\leq(\sup\limits_{x\in I}f(x))(\sup\limits_{x\in I}g(x))$\\
\indent $\inf$ takes the similar($\leq\rightarrow\geq$) (provided when the sum are defined)
\chapter{Vector space}
In this section:\\
\indent K denotes a unitary ring.\\
\indent Let 0 be zero element of K\\
\indent 1 be the unity of K
\section{K-module}
\subsection{Def}

Let $(V,+)$ be a commutative group.We call left/right K-module structure:\\
\indent any mapping $\Phi$:$K\times V\rightarrow V$
\begin{itemize}
    \item $\forall(a,b)\in K\times K,\forall x\in V\quad \Phi(ab,x)=\Phi(a,\Phi(b,x))/\Phi(b,\Phi(a,x))$
    \item $\forall (a,b)\in K\times K,\forall x\in V,\Phi(a+b,x)=\Phi(a,x)+\Phi(b,x)$
    \item $\forall a\in K,\forall(x,y)\in V\times V,\Phi(a,x+y)=\Phi(a,x)+\Phi(a,y)$
    \item $\forall x\in V,\Phi(1,x)=x$
\end{itemize}
A commutative group (V,+) equipped with a left/right K-module structure is called a left/right K-module.
\subsection{Remark}\quad Let $K^{op}$ be the set K equipped with the following composition laws:
\begin{itemize}
    \item $K\times K\rightarrow K$
    \item $(a,b)\mapsto a+b$
    \item $K\times K\rightarrow K$
    \item $(a,b)\mapsto ba$
\end{itemize}
Then $K^{op}$ forms a unitary ring\\
Any left $K^{op}-module$ is a right K-module\\
Any right $K^{op}-module$ is a left K-module\\
$(K^{op})^{op}=K$
\subsection{Notation}

When we talk about a left/right K-module $(V,+)$,we often write its left K-module structure as $K\times V\rightarrow V\quad (a,x)\mapsto ax$\\
\indent The axioms become:\begin{align*}
    &\forall(a,b)\in K\times K,\forall x\in V\quad (ab)x=a(bx)/b(ax)\\
    &\forall(a,b)\in K\times K,\forall x\in V\quad (a+b)x=ax+bx\\
    &\forall a\in K,\forall(x,y)\in V\times V\quad a(x+y)=ax+ay\\
    &\forall x\in V\quad 1x=x
\end{align*}
\subsection{K-vector space}

If K is commutative,then $K^{op}=K$,so left K-module and right K-module structure are the same .We simply call them K-module structure. A commutative group equipped with a K-module structure is called a K-module.If K is a field,a K-module is also called a K-vector space

Let $\Phi:K\times V\rightarrow V$ be a left or right K-module structure$$\forall x\in V,\Phi(\cdot,x):K\rightarrow V\quad(a\in K)\mapsto\Phi(a,x)$$\indent is a morphism of addition groups.Hence $\Phi(0,x)=0,\Phi(-a,x)=-\Phi(a,x)$\\
$\forall a\in K,\Phi(a,\cdot):V\rightarrow V$ is a morphism of groups.Hence $\Phi(a,0)=0,\Phi(a,-x)=-\Phi(a,x)(\cdot \ is\ a\ var)$
\subsection{Association:}
$\forall x\in K$
\begin{align*}
    (f(f+g)+h)(x)=(f+g)(x)+h(x)=f(x)+g(x)+h(x)\\
    (f+(g+h))(x)=f(x)+((g+h)(x))=f(x)+g(x)+h(x)
\end{align*} 

Let $0:I\rightarrow K:x\mapsto0 \quad \forall f\in K^I\quad f+0=f$\\
Let $-f: f+(-f)=0$\\
The mapping $K\times K^I\rightarrow K^I:(a,f)\mapsto af\quad (af)(x)=af(x)$ is a left K-module structure\\
The mapping $K\times K^I\rightarrow K^I:(a\in I)\mapsto ((x\in I)\mapsto f(x)a)\quad (af)(x)=af(x)$ is a right K-module structure
\subsection{Remark:}

We can also write an element $\mu\ of\ K^I$ is the form of a family $(\mu_i)_{i\in I}$ of elements in K ($\mu_i$is the image of $i\in I\ by\ \mu$)\\
Then \begin{align*}
    &(\mu_i)_{i\in I}+(\nu_i)_{i\in I}:=(\mu_i+\nu_i)_{i\in I}\\
    &a(\mu_i)_{i\in I}:=(a\mu_i)_{i\in I}\\
    &(\mu_i)_{i\in I}a=(\mu_ia)_{i\in I}
\end{align*}
\section{sub K-module}
\subsection{Def}

Let V be a left/right K-module.If W is a subgroup of V. Such that $\forall a\in K,\forall x\in W\quad ax/xa\in W$, then we say that W is left/right sub-K-module of V.
\subsection{Example}

Let I be a set .Let $K^{\bigoplus I}$ be the subset of $K^I$ composed of mappings $f:I\rightarrow K$ such that $I_f=\{x\in I\mid f(x)\not=0\}$ is finite. It is a left and right sub-K-module of $K^I$\\
\indent In fact,$\forall (f,g)\in K^{\bigoplus I}\times K^I\quad I_{f-g}={x\in I\mid f(x)-g(x)\not=0}\subseteq I_f\cup I_g$\\
\indent Hence $f-g\in K^{\bigoplus I}$ So $K^{\bigoplus I}$ is a subgroup of $K^I$\\
\indent $\forall a\in K,\forall f\in K^{\bigoplus I}\quad I_{af}\subseteq I_f,I_{(x\mapsto f(x)a)}\subseteq I_f$
\section{morphism of K-modules}
\subsection{Def}

Let V and W be left K-module, A morphism of groups $\phi:V\rightarrow W$ is called a morphism of left K-modules if $\forall(a,x)\in K\times V,\phi(ax)=a\phi(x)$
\subsection{K-linear mapping}
\indent If K is commutative, a morphism of K-modules is also called a K-linear mapping. We denote by $\hom_{K-Mod}(V,W)$ the set of all morphism of left-K-module from V to W.This is a subgroup of $W^V$
\subsection{Theorem}

Let V be a left K-module. Let I be a set.\\
The mapping $\hom_{K-Mod}(K^{\bigoplus I},V)\rightarrow V^I:\ \phi\rightarrow (\phi(e_i))_{i\in I}$ is a bijection where $e_i:I\rightarrow K:j\mapsto\left\{\begin{aligned}
    1\quad j=i\\
    0\quad j\not=i
\end{aligned} \right.$
\subsection{Remark:column}

In the case where $I={1,2,3,...,n}\ V^I$ is denoted as $V^n,K^I$ is denoted as $K^n$\\
For any $(x_1,...,x_n)\in V^n$,by the theorem, there exists a unique morphism of left K-modules $\phi:K^n\rightarrow V$ such that $\forall i\in{1,...,n}\phi(e_i)=x_i$\\
We write this $\phi$ as a column$\begin{pmatrix}
    x_1\\
    ...\\
    x_n
\end{pmatrix}$It sends $(a_1,...,a_n)\in K^n$ to $a_1x_1+...+a_nx_n$
\section{kernel}
\subsection{Prop}

Let G and H be groups and $f:G\rightarrow H$ be a morphism of groups\begin{itemize}
    \item $I_m(f)\subseteq H $ is a subgroup of H
    \item $\ker(f)=\{x\in G\mid f(x)=e_H\}$
    \item $f$ is injection iff $\ker(f)=\{e_G\}$
\end{itemize}
\subsection{Def}

$\ker(f)$ is called the kernel of $f$
\subsection{Theorem}
$f$ is injection iff $\ker(f)=\{e_G\}$
\subsection{Proof}
Let $e_G$ and $e_H$ be neutral element of G and H respectively
\begin{itemize}
    \item [(1)]Let x and y be element of G\\$f(x)f(y)^{-1}=f(x)f(y)^{-1}=f(xy^{-1})\in Im(f).$ So $Im(f)$ is a subgroup of H
    \item [(2)]Let x and y be element of $\ker(f)$ One has $f(xy^{-1})=f(x)f(y)^{-1}=e_H\quad e_H^{-1}=e_H.$ So $xy^{-1}\in\ker(f)$ So $\ker(f)$ is a subgroup of G
    \item [(3)]Suppose that f is injection.\\ Since $f(E_G)=e_H$ one has $\ker(f)=f^{-1}(\{e_H\})=\{e_G\}$ Suppose that $\ker(f)=\{e_G\}$ If $f(x)=f(y)$then $f(xy^{-1})=f(x)f(y)^{-1}=e_H$\\Hence $xy^{-1}=e_G\ \Rightarrow\ x=y$
\end{itemize}
\subsection{Def}

Let (V,+) be a commutative group, I be a set. We define a composition law + on $V^I$ as follows$$(x_i)_{i\in I}+(y_i)_{i\in I}\ :=(x_i+y_i)_{i\in I}$$
Then $V^I$ forms a commutative group
\subsection{Remark}

Let E and F be left K-modules\\$\hom_{K=Mod}(E,F):=\{\text{morphisms of left K-modules from E to F}\}\subseteq F^E$ is a subgroup of $F^E$\\
In fact f and g are elements of $\hom_{K-Mod}(E,F)$, then $f-g$ is also a morphism of left K-module\\$(f-g)(x+y)=f(x+y)-g(x+y)=(f(x)+f(y))-(g(x)+g(y))=(f(x)-g(x))+(f(y)-g(y))=(f-g)(x)+(f-y)(x)$
\subsection{Theorem}

Let V be a left K-module,I be a set The mapping $\hom_{K-Mod}(K^{\bigoplus I},V)\rightarrow V^I\ :\ \phi\mapsto(\phi(e_i))_i\in I$ is an isomorphism of groups, where $e_i:I\rightarrow K:j\mapsto\left\{\begin{aligned}
    1\quad j=i\\
    0\quad j\not=i
\end{aligned} \right.$
\subsection{Proof:}
One has $(\phi+\psi)(e_i)=\phi(e_I)+\psi(e_i)$\\
$\forall(\phi,psi)\in \hom_{K-Mod}(K^{\bigoplus I},V)^2$\\
Hence $\Psi(\phi,\psi)=(\phi(e_i)+\psi(e_i))_{i\in I}=\Psi(\phi)+\Psi(\psi)$\\
So $\Psi$ is a morphism of groups
\begin{itemize}
    \item [injectivity] Let $\phi\in \hom_{K-Mod}(K^{\bigoplus I},V)$ Such that $\forall i\in I(\forall \phi\in\ker(\Psi))\quad\phi(e_i)=0$\\Let $a =(a_i)_{i\in I}\in K^{\bigoplus I}$ One has $a=\sum\limits_{i\in I}a_ie_i$ \\ If fact,$\forall j\in I,a_j=\sum\limits_{i\in I,a_i\not=0}a_ie_i(j)$\\Thus $\phi(a)=\sum\limits_{i\in I,a_i\not=0}a-I\phi(e_i)=0$\\Hence $\phi$ is the neutral element.
    \item [surjectivity] Let $x=(x_i)_{i\in I}\in V^I$ We define $\phi_x:K^{\bigoplus I}\rightarrow V$ such that $\forall a=(a_i)_{i\in I}\in K^{\bigoplus I},\phi_x(a)=\sum\limits_{i\in I,a_i\not=0}a_ix_i$\\This is a morphism of left K-modules\\$forall i\in I,\phi_x(e_i)=1x_i=x_i$  So $\Psi(\phi_x)=x$
\end{itemize}
{\color{blue}Suppose that K' is a unitary ring,and V is also equipped with a right K'-module structure, Then $\hom_{K-Mod}(K^{\bigoplus I},V)\subseteq V^{K^{\bigoplus I}}$ is a right sub-k'-module ,and $\Psi$ in the theorem is a right K'-module isomorphism}
\chapter{Monotone mappings}
\section{Def}

Let I and X be partially ordered sets,$f:I\rightarrow X$ be a mapping.
\begin{itemize}
    \item If $\forall(a,b)\in I\times I$ such that $a<b$. One has $f(a)\leq f(b)/f(a)<f(b)$,then we say that $f$ is increasing/strictly increasing. decreasing takes similar way.
    \item If $f$ is (strictly) increasing or decreasing, we say that $f$ is (strictly) monotone.
\end{itemize}
\section{Prop.}

Let  X,Y,Z be partially ordered sets.$f:X\rightarrow Y, g:Y\rightarrow Z$ be mappings
\begin{itemize}
    \item If $f$ and $g$ have the same monotonicity, then $g\circ f$ is increasing
    \item If $f$ and $g$ have different monotonicities, then $g\circ f$ is decreasing
\end{itemize}
strict monotonicities takes the same
\section{Def}

Let $f$ be a function from a partially ordered set I to another partially ordered set .If $f\mid_{Dom(f)}\rightarrow X$ is (strictly) increasing/decreasing then we say that $f$ is (strictly) increasing/decreasing
\section{Prop.}

Let I and X be partially ordered sets. $f$ be function from I to X.
\begin{itemize}
    \item If $f$ is increasing/decreasing and $f$ is injection, then $f$ is strictly increasing/decreasing
    \item Assume that I is totally ordered and $f$ is strictly monotone, then $f$ is injection
\end{itemize}
\section{Prop}

Let A be totally ordered set, B be a partially ordered set, $f$ be an injective function from A to B\\
If $f$ is increasing/decreasing ,then so is $f^{-1}$
\section{Def}
Let X and Y be partially ordered sets. $f:X\rightarrow Y$ be a bijection. If both $f$ and $f^{-1}$ are increasing ,then we say that $f$ is an isomorphism of partially ordered sets.\\
\indent(If X is totally, then a mapping $f:X\rightarrow Y$ is an isomorphism of partially ordered sets iff $f$ is a bijection and $f$ is increasing)
\section{Prop.}
Let I be a subset of $\mathbb{N} $ which is infinite. Then there is a unique increasing bijection $\lambda_I:\mathbb{N} \rightarrow I$
\section{Proof}
\subsection{bijection}
We construct $f:\mathbb{N} \rightarrow I$ by induction as follows.\\
Let $f(0)=\min I$ Suppose that $f(0),...,f(n)$ are constructed\\
then we take $f(n+1):=\min(I\backslash\{f(0),...,f(n)\})$\\
Since $I\backslash\{f(0),...,f(n-1)\}\supseteq I\backslash\{f(0),...,f(n)\}.$Therefore $f(n)\leq f(n+1)\\
Since f(n+1)\not\in\{f(0),...,f(n)\}$,we have $f(n)<f(n+1)$\\
Hence $f$ is strictly increasing and ths is injective\\
If $f$ is not surjective,then $I\backslash Im(f)$ has a element N. \\
Let $m=\min\{n\in\mathbb{N} \mid N\leq f(n)\}$.\\
Since $N\not\in Im(f),N<f(m)$.\\
So $m\not=0$.Hence $f(m-1)<N<f(m)=\min(I\backslash\{f(0),...,f(m-1)\})$ \\
By definition,$N\in I\backslash Im(f)\subseteq I\backslash \{f(0),...,f(m-1)\}$,\\Hence $f(m)\leq N$,causing contradiction.
\subsection{uniqueness}
exercise: Prove that $Id_\mathbb{N}$ is the only isomorphism of partially ordered sets from $\mathbb{N} $ to $\mathbb{N} $
\chapter{sequence and series}
Let $I\subseteq \mathbb{N}$ be a infinite subset
\section{Def} Let X be a set.We call sequence in X parametrized by I a mapping from I to X.
\section{Remark}
If K is a unitary ring and E is a left K-module then the set of sequence $E^I$ admits a left-K-module structure. If $x=(x_n)_{n\in I}$ is a sequence in E, we define a sequence $\sum(x):=(\sum\limits_{i\in I,i\leq n}x_i)_{n\in \mathbb{N}}$,called the series associated with the sequence x.
\section{Prop}
$\sum:E^I\rightarrow E^\mathbb{N}$ is a morphism of left-K-module
\section{proof}
Let $x=(x_i)_{i\in I}$ and $y=(y_i)_{i\in I}$ be elements of $E^I$\\
$\sum\limits_{i\in I,i\leq n}(x_i+y_i)=(\sum\limits_{i\in I,i\leq n}x_i)+(\sum\limits_{i\in I,i\leq n}y_i),\lambda\sum\limits_{i\in I,i\leq n}x_i=\sum\limits_{i\in I,i\leq n}\lambda x_i$
\section{Prop}
Let I be a totally ordered set . X be a partially ordered set,$f:I\rightarrow X$ be a mapping ,$J\in I$ Assume that J does not have any upper bound in I
\begin{itemize}
    \item If $f$ is increasing ,then $f(I)$ and $f(J)$ have the same upper bounds in X
    \item If $f$ is decreasing ,then $f(I)$ and $f(J)$ have the same lower bounds in X
\end{itemize}
\section{limit}
\subsection{Def}
Let $i\subseteq\mathbb{N}$ be a infinite subset.$\forall (x_i)_{n\in I}\in [-\infty,+\infty]^I$ where $[-\infty,+\infty]$ denotes $\mathbb{R}\cup\{-\infty,+\infty\}$,we define:
$$\limsup\limits_{n\in I,n\rightarrow +\infty}x_n:=\inf\limits_{n\in I}(\sup\limits_{i\in I,i\geq n}x_i)$$
$$\liminf\limits_{n\in I,n\rightarrow +\infty}x_n:=\sup\limits_{n\in I}(\inf\limits_{i\in I,i\geq n}x_i)$$
If $\limsup\limits_{n\in I,n\rightarrow +\infty}x_n=\liminf\limits_{n\in I,n\rightarrow +\infty}x_n=l$, we then say that $(x_n)_{n\in I}$ tends to $l$ and that $l$ is the limit of $(x_n)_{n\in I}$. If in addition $(x_n)_{n\in I}\in \mathbb{R}^I$ and $l\in \mathbb{R}$,we say that $(x_n)_{n\in I}$ converges to $l$
\subsection{Remark}
If $J\subseteq\mathbb{N}$ is an infinite subset, then: $$\limsup\limits_{n\in I,n\rightarrow +\infty}=\inf\limits_{n\in J}(\sup\limits_{i\in I,i\geq n}x_i)$$
$$\liminf\limits_{n\in I,n\rightarrow +\infty}x_n=\sup\limits_{n\in J}(\inf\limits_{i\in I,i\geq n}x_i)$$
Therefore ,if we change the values of finitely many terms in $(x_i)_{i\in I}$ the limit superior and the limit inferior do not change.\\
In fact, if we take $J=\mathbb{N}\backslash\{0,...,m\}$, then $\inf\limits_{n\in J}(...)$ and $\sup\limits_{n\in J}(...)$ only depends on the values of $x_i,i\in I,i\geq m$
\subsection{Prop}
$\forall (x_n)_{n\in I}\in [-\infty,+\infty]^I,\ \liminf\limits_{n\in I,n\rightarrow+\infty}x_n\leq\limsup\limits_{n\in I,n\rightarrow+\infty}x_n$
\subsection{Prop}
Let $(x_n)_{n\in I}\in[-\infty,+\infty]^I$
\begin{align*}
    &\forall c\in\mathbb{R} & \begin{aligned}
        \limsup\limits_{n\in I,n\rightarrow+\infty}(x_n+c)=(\limsup\limits_{n\in I,n\rightarrow+\infty}x_n)+c\\
        \liminf\limits_{n\in I,n\rightarrow+\infty}(x_n+c)=(\liminf\limits_{n\in I,n\rightarrow+\infty}x_n)+c
    \end{aligned}\\
    &\forall c\in \mathbb{R}_{>0} &\begin{aligned}
        \limsup\limits_{n\in I,n\rightarrow+\infty}(\lambda x_n)=\lambda\limsup\limits_{n\in I,n\rightarrow+\infty}x_n\\\liminf\limits_{n\in I,n\rightarrow+\infty}(\lambda x_n)=\lambda\liminf\limits_{n\in I,n\rightarrow+\infty}x_n
    \end{aligned}\\
    &\forall c\in \mathbb{R}_{<0} &\begin{aligned}
        \limsup\limits_{n\in I,n\rightarrow+\infty}(\lambda x_n)=\lambda\liminf\limits_{n\in I,n\rightarrow+\infty}x_n\\\liminf\limits_{n\in I,n\rightarrow+\infty}(\lambda x_n)=\lambda\limsup\limits_{n\in I,n\rightarrow+\infty}x_n
    \end{aligned}
\end{align*}
\subsection{Prop}
Let $(x_n)_{n\in I}$ be elements in $[-\infty,+\infty]^I$. Suppose that there exists $N_0\in \mathbb{N}$ such that $\forall n\in I,n\geq N_0$,one has $x_n\leq y_n$ Then $$\limsup\limits_{n\in I,n\rightarrow+\infty}(x_n)\leq\limsup\limits_{n\in I,n\rightarrow+\infty}y_n$$ ,$$\liminf\limits_{n\in I,n\rightarrow+\infty}(x_n)\geq\liminf\limits_{n\in I,n\rightarrow+\infty}y_n$$
\subsection{Theorem}
Let $(x_n)_{n\in I},(y_n)_{n\in I},(z_n)_{n\in I}$ be elements of $[-\infty,+\infty]^I$\\
Suppose that \begin{itemize}
    \item $\exists N-N\in\mathbb{N},\forall n\in I,n\geq N_0$ one has $x_n\leq y_n\leq z_n$
    \item $(x_n)_{n\in I}$ and $(z_n)_{n\in I}$ tend to the same limit $l$
\end{itemize}
Then $(y_n)_{n\in I}$ tends to $l$
\subsection{Def}
Let I be an infinite subset of $\mathbb{N}$, and $(x_n)_{n\in I}$ be a sequence in some set X. We call subsequence of $(x_n)_{n\in I}$ a sequence of the form $(x_n)_{n\in J}$,where J is an infinite subset of I
\subsection{Prop}
Let I and J be infinite subset of $\mathbb{N}$ such that $J\subseteq I$\\
$\forall (x_n)_{n\in I}\in[-\infty,+\infty]^I$,one has $$\liminf\limits_{n\in I,n\rightarrow+\infty}(x_n)\leq\liminf\limits_{n\in I,n\rightarrow+\infty}y_n$$$$\limsup\limits_{n\in I,n\rightarrow+\infty}(x_n)\geq\limsup\limits_{n\in I,n\rightarrow+\infty}y_n$$
In particular, if $(x_n)_{n\in I}$ tends to $l\in[-\infty,+\infty]$,then $(x_n)_{n\in J}$ tends to $l$
\subsection{Prop}
$\forall n\in \mathbb{N} $,one has $$\liminf\limits_{n\in J,n\rightarrow+\infty}(x_n)\geq\liminf\limits_{n\in I,n\rightarrow+\infty}y_n$$$$\limsup\limits_{n\in J,n\rightarrow+\infty}(x_n)\leq\limsup\limits_{n\in I,n\rightarrow+\infty}y_n$$
\subsection{Theorem}
Let $I\subseteq\mathbb{N} $ be an infinite subset and $(x_N)_{n\in I}$ be a sequence in $[-\infty,+\infty]$
\begin{itemize}
    \item If the mapping $(n\in I)\mapsto x_n$ is increasing,then $(x_N)_{i\in I}$ tends to $\sup\limits_{n\in I}x_n$
    \item If the mapping $(n\in I)\mapsto x_n$ is decreasing,then $(x_N)_{i\in I}$ tends to $\inf\limits_{n\in I}x_n$
\end{itemize}
\subsection{Notation}
If a sequence $(x_N)_{n\in I}\in [-\infty,+\infty]$ tends to some $l\in[-\infty,+\infty]$ the expression $\lim\limits_{n\in I,n\rightarrow}x_n$ denotes this limit $l$
\subsection{Corollary}
Let $(x_n)_{n\in I}$ be a sequence in $\mathbb{N} _{\geq0}$ Then the series $\sum\limits_{n\in I}x_n$(the sequence $(\sum\limits_{i\in I,i\leq n})_{n\in \mathbb{N} }$) tends to an element in $\mathbb{N} _{\geq0}\cup\{+\infty\}$ It converges in $\mathbb{R} $ iff it is bounded from above (namely has an upper bound in $\mathbb{R} $)
\subsection{Notation}
If a series $\sum\limits_{n\in I}x_n$ in $[-\infty,+\infty]$ tends to some limit, we use the expression $\sum\limits_{n\in I}x_n$ to denote the limit
\subsection{Theorem: Bolzano-Weierstrass}
Let $(x_n)_{n\in I}$ be a sequence in $[-\infty,+\infty]$ There exists a subsequence of $(x_n)_{n\in I}$ that tends to $\limsup\limits_{n\in I,n\rightarrow +\infty}x_n$ There exists a subsequence of $(x_n)_{n\in I}$ that rends to $\liminf\limits_{n\in I,n\rightarrow +\infty}x_n$
\subsection{Proof}
Let $J=\{n\in I \mid \forall m\in I,\text{if}\ m\leq n\ \text{then}\ x_m\leq x_n\}$

\indent If $J$ is infinite, the sequence $(x_N)_{n\in J}$ is decreasing so it tends to $\inf\limits_{n\in J}x_n$

\indent $\forall n\in J$ by definition $x_n=\sup\limits_{i\in I,i\geq n}x_i$ so $\limsup\limits_{n\in I,n\rightarrow+\infty}x_n=\inf\limits_{n\in J}\sup\limits_{i\in I,i\geq n}x_i=\inf\limits_{n\in J}x_n=\lim\limits_{n\in J,n\rightarrow+\infty}x_n$

\indent Assume that $J$ is finite. Let $n_0\in I$ such that $\forall n\in J,n<n_0$.Denote by $l=\sup\limits_{n\in I,n\geq n_0}$

\indent Let $N\in\mathbb{N}$ such that $N\geq n_0$. By definition $\sup\limits_{i\in  I,i\geq n_0}x_i\leq l$. If the strict inequality $\sup\limits_{i\in I,i\geq N}x_i<l$ holds, then $\sup\limits_{i\in I,i\geq N}x_i$ is NOT an upper bound of $\{x_n\mid n\in I,n_0\leq n<N\}$

\indent So there exists $n\in I$ such that $n_0\leq n<N$ such that $x_n>\sup\limits_{i\in I,i\geq N}x_i$ We may also assume that $n $ is largest among elements of {$I\cap[n_0,N[$} that satisfies this inequality.

Then $\forall m\in I$ if $m\geq n$ then $x_m\leq x_n$ Thus $n\in J$ that contradicts the maximality of $n_0$

Therefore $$l=\sup\limits_{i\in I,i\geq N}x_i$$, which leads to $$\limsup\limits_{n\in I,n\rightarrow+\infty}x_n=l$$

Moreover, if $m\in I,m\geq n_0$ then $m\not\in J$,so $x_m<l$(since otherwise $x_m=\sup\limits_{i\in I,i\geq m}x_i$ and hence $m\in J$)Hence,$\forall\ finite\ subset\ I'\ of\{m\in I\mid m\geq n_0\}$

$\max\limits_{i\in I}x_i<l$ and hence $\exists n\in I$,such that $n>\max I'$,and $\max\limits_{i\in I'}x_i<x_n$

We construct by induction an increasing sequence $(n_j)_{j\in \mathbb{N} }$ in $I$

Let $n_0$ be as above. Let $f:\mathbb{N} \rightarrow I_{\geq n_0}$ be a surjective mapping.

If $n_j$ is chosen, we choose $n_{j+1}\in I$ such that $$n_{j+1}>n_j, x_{n_{j+1}}>\max\{x_{f(j)},x_{n_j}\}$$

Hence the sequence $(x_{n_j})_{j\in \mathbb{N} }$ is increasing

And $$\sup\limits_{j\in \mathbb{N} }x_{n_j}\leq\sup\limits_{j\in\mathbb{N} }x_{f(j)}=\sup\limits_{n\in I,n\geq n_0}x_n=l$$

$$l=\sup\limits_{n\in I,n\geq n_0}$$
 
So $(x_{n_j})_{j\in\mathbb{N} }$ tends to $l$
\chapter{Cauchy sequence}
\section{Def}
Let $(x_n)_{n\in I}$ be a sequence in $\mathbb{R} $\\
If $\inf\limits_{N\in \mathbb{N} }\sup\limits_{(n,m)\in I\times I,\ n,m\geq N}\lvert x_n-x_m\rvert=\lim\limits_{N\rightarrow+\infty}\sup\limits_{(n,m)\in I\times I,\ n,m\geq N}\lvert x_n-x_m\rvert=0$ then we say that $(x_n)_{n\in I}$ is a Cauchy sequence
\section{Prop}
\begin{itemize}
    \item If $(x_n)_{i\in I}\in \mathbb{R} ^I$ converges to some $l\in \mathbb{R} $, then it is a Cauchy sequence
    \item If $(x_N)_{i\in I}$ is a Cauchy sequence, there exists $M>0$ such that $ \forall n\in I\quad \lvert x_n\rvert\leq M$
    \item If $(x_n)_{n\in I}$ is a Cauchy sequence, then $\forall J\subseteq I$ infinite,$(x_n)_{n\in I}$ is a Cauchy sequence.
    \item If $(x_n)_{n\in I}$ is a Cauchy sequence, then $\forall J\subseteq I$ infinite and $l\in \mathbb{R} $ such that $(x_n)_{n\in I}$ converges to $l$, then $(x_n)_{n\in J}$ converges to $l$ too.
\end{itemize}
\section{Theorem: Completeness of real number}
If $(x_n)_{n\in I}\in\mathbb{R} ^I$ is a Cauchy sequence,
then it converges in $\mathbb{R}$
\subsection{Proof}
Since $(x_n)_{n\in I}$ is a Cauchy sequence, $\exists M\in \mathbb{R} _{>0}$ such that $-M\leq x_n\leq M\quad \forall x\in I$ So $\limsup\limits_{n\in I,n\rightarrow+\infty}x_n\in \mathbb{R} $. By Bolzano-Weierstrass theorem. $\exists J\subseteq I$ infinite such that $(x_n)_{n\in I}$ converges to $\limsup\limits_{n\in I,n\rightarrow+\infty}x_n\in\mathbb{R} $. Therefore $(x_n)_{n\in I}$ converges to the same limit.
\section{Absolutely converge}
We say that a series $\sum\limits_{n\in I}x_n\in \mathbb{R} $ converges absolutely if $\sum\limits_{n\in I}\lvert x_n\rvert<+\infty$
\subsection{Prop}
If a series $\sum\limits_{n\in I}x_n$ converges absolutely, then it converges in $\mathbb{R}$
\chapter{Comparison and Technics of Computation}
\section{Def}
Let $(x_n)_{n\in I}$ and $(y_n)_{n\in I}$ be sequence in $\mathbb{R} $
\begin{itemize}
    \item If there exists $M\in\mathbb{R} _{>0}$ and $N\in \mathbb{N} $ such that $\forall n\in I_{\geq N},\lvert x_N\rvert\leq M\lvert y_m\rvert$ then we write $x_n=O(y_n),n\in I,n\rightarrow+\infty$
    \item If there exists $(\epsilon_n)_{n\in I}\in \mathbb{R} ^I$ and $N\in\mathbb{N} $ such that $\lim\limits_{n\in I,n\rightarrow+\infty}\epsilon_n=0$ and $\forall n\in I_{\geq N},\lvert x_N\rvert\leq \lvert \epsilon y_m\rvert$, then we write $x_n=\circ(y_n),n\in I,n\rightarrow +\infty$ \\Example:
        $$\lim\limits_{n\rightarrow+\infty}{1\over n}=0$$
\end{itemize}
\section{Prop.}
Let I and X be partially ordered sets and $f:I\rightarrow X$ be an increasing/decreasing mapping. Let J ba a subset of I. Assume that any elements of I has an upper bound in J. Then $f(I)$ and $f(J)$ have the same upper/lower bounds in X
\section{Theorem}
Let I be a totally ordered set, $f:I\rightarrow[-\infty,+\infty]$ and $g:I\rightarrow[-\infty,+\infty]$ be two mappings that are both increasing/decreasing.Then the following equalities holds,provided that the sum on the right hand side of the equality is wel defined.
$$\sup\limits_{x\in I,\{f(x),g(x)\}\not =\{-\infty,+\infty\}}=(\sup\limits_{x\in I}f(x))+(\sup\limits_{y\in I}g(y))$$
$$\inf\limits_{x\in I,\{f(x),g(x)\}\not =\{-\infty,+\infty\}}=(\inf\limits_{x\in I}f(x))+(\inf\limits_{y\in I}g(y))$$
\subsection{Proof}
We can assume $f$ and $g$ increasing. Let $a=\sup f(I)$,$b=\sup g(I)$

Let $A=\{(x,y)\in I\times I\mid\{f(x),g(x)\}\not=\{-\infty,+\infty\}\}$

We equip A with the following order relation.
$$(x,y)\leq(x',y')\ iff\ x\leq x',y\leq y'$$

Let $B=A\cap\Delta_I=\{(x,y)\in A\mid x=y\}$.\\ Consider $$h: A\rightarrow[-\infty,+\infty]\quad h(x,y)=f(x)+g(y)$$
$h$ is increasing.\\
Let $(x,y)\in A$. Assume that $x\leq y$\\
If $\{f(y),g(y)\}\not=\{-\infty,+\infty\}$ then $(y,y)\in B$ and $(x,y)\leq(y,y)$\\
If $\{f(y),g(y)\}=\{-\infty,+\infty\}$ and for $(x,y)\in A\Rightarrow f(y)=+\infty,g(y)=-\infty$. So $a=+\infty$, Hence $b>-\infty$\\
So $\exists z\in I$ such that $g(z)>-\infty$. We should have $y\leq z$ Hence $f(z)+g(z)$ is well defined,$(z,z)\in B$ and $(x,y)\leq(z,z)$ Similarly, if$x\geq y,\ (x,y)$ has also an upper bound in B. Therefore: $\sup h(A)=\sup h(B)$
\section{Prop.}
Let $I\subseteq\mathbb{N} $ be an infinite subset. Let $(x_n)_{n\in I}$ and $(y_n)_{n\in I}$ be elements of $[-\infty,+\infty]^I$ such that ,$\forall n\in I\quad \{x_n,y_n\}\not=\{-\infty,+\infty\}$. Then the following inequalities holds, provided that the sum on the right hand side is well defined.
$$\limsup\limits_{n\in I,n\rightarrow+\infty}(x_n+y_n)\leq(\limsup\limits_{n\in I,n\rightarrow+\infty}x_n)+(\limsup\limits_{n\in I,n\rightarrow+\infty}y_n)$$
$$\liminf\limits_{n\in I,n\rightarrow+\infty}(x_n+y_n)\geq(\liminf\limits_{n\in I,n\rightarrow+\infty}x_n)+(\liminf\limits_{n\in I,n\rightarrow+\infty}y_n)$$
\subsection{Proof}
$\forall n\in\mathbb{N} ,$ let $A_N=\sup\limits_{n\in I,n\geq N}x_n\quad B_N=\sup\limits_{n\in I,n\geq N}y_n$. $(A_N)_{N\in \mathbb{N} }$ and $(B_N)_{N\in \mathbb{N} }$ are decreasing, and $\limsup\limits_{n\in I,n\rightarrow+\infty}x_n=\inf\limits_{N\in \mathbb{N} }A_N\quad\limsup\limits_{n\in I,n\rightarrow+\infty}y_n=\inf\limits_{N\in \mathbb{N} }B_N$\\
By theorem:
$$\inf\limits_{N\in \mathbb{N} }A_N+\inf\limits_{N\in \mathbb{N} }B_N=\inf\limits_{N\in \mathbb{N},\{A_N,B_N\}\not=\{-\infty,+\infty\} }(A_N+B_N)$$
Let $C_N=\sup\limits_{n\in I,n\geq N}(x_n+y_n)\leq A_N+B_N$ if $A_N+B_N$ is defined.\\
Therefore $$\inf\limits_{N\in \mathbb{N} }C_N\leq\inf\limits_{N\in \mathbb{N},\{A_N,B_N\}\not=\{-\infty,+\infty\} }(A_N+B_N)=\inf\limits_{N\in \mathbb{N} }A_N+\inf\limits_{N\in \mathbb{N} }B_N$$
\section{Prop.}
Let $I\subseteq\mathbb{N} $ be an infinite subset. Let $(x_n)_{n\in I}$ and $(y_n)_{n\in I}$ be elements of $[-\infty,+\infty]^I$ such that ,$\forall n\in I\quad \{x_n,y_n\}\not=\{-\infty,+\infty\}$. Then the following inequalities holds, provided that the sum on the right hand side is well defined.
$$\limsup\limits_{n\in I,n\rightarrow+\infty}(x_n+y_n)\geq(\limsup\limits_{n\in I,n\rightarrow+\infty}x_n)+(\limsup\limits_{n\in I,n\rightarrow+\infty}y_n)$$
$$\liminf\limits_{n\in I,n\rightarrow+\infty}(x_n+y_n)\geq(\liminf\limits_{n\in I,n\rightarrow+\infty}x_n)+(\liminf\limits_{n\in I,n\rightarrow+\infty}y_n)$$
\subsection{Proof}
a tricky proof ?:
$$\limsup\limits_{n\in I,n\rightarrow}x_n=\limsup\limits_{n\in I,n\rightarrow}(x_n+y_n-y_n)\leq\limsup\limits_{n\in I,n\rightarrow}(x_n+y_n)-\liminf\limits_{n\in I,n\rightarrow}y_n$$
to have a true proof,only need to discuss conditions with $\infty$
\section{Theorem}
Let $(x_n)_{n\in I}$ and $(y_n)_{n\in I}$ be elements of $[-\infty,+\infty]^I$ . Assume that $\forall n\in I,y_n\in\mathbb{R} $ and $(y_n)_{n\in I}$ converges to some $i\in \mathbb{R} $.\\Then:
$$\limsup\limits_{n\in I,n\rightarrow+\infty}(x_n+y_n)=(\limsup\limits_{n\in I,n\rightarrow+\infty}x_n)+l$$
$$\liminf\limits_{n\in I,n\rightarrow+\infty}(x_n+y_n)=(\liminf\limits_{n\in I,n\rightarrow+\infty}x_n)+l$$
\section{Prop.}
Let $(x_n)_{n\in I}$ and $(y_n)_{n\in I}$ be elements of $[-\infty,+\infty]^I$ \\
Then:
$$\liminf\limits_{n\in I,n\rightarrow+\infty}\max\{x_n,y_n\}=\max\{\liminf\limits_{n\in I,n\rightarrow+\infty}x_n,\liminf\limits_{n\in I,n\rightarrow+\infty}y_n\}$$
$$\liminf\limits_{n\in I,n\rightarrow+\infty}\min\{x_n,y_n\}=\min\{\liminf\limits_{n\in I,n\rightarrow+\infty}x_n,\liminf\limits_{n\in I,n\rightarrow+\infty}y_n\}$$
\subsection{Proof}
About the first inequality.Since $\max\{x_n,y_n\}\geq x_n\quad \max\{x_n,y_N\}\geq y_n$

By the theorem of Bolzano-Weierstrass theorem, there exists an infinite subset $J$ of $I$ such that$$\lim\limits_{n\in J,n\rightarrow+\infty}=\limsup\limits_{n\in J,n\rightarrow+\infty}\max\{x_n,y_n\}$$
Let $J_1=\{n\in J\mid x_n\geq y_n\}\quad J_1=\{n\in J\mid x_n\leq y_n\}$\\
$J_1\cup J_2=J$ So either $J_1$ or $J_2$ is infinite\\
Suppose that $J_1$ is infinite, then
$$\lim\limits_{n\in J,n\rightarrow}\max\{x_n,y_n\}=\lim\limits_{n\in J_1,n\rightarrow}\max\{x_n,y_n\}=\lim\limits_{n\in J,n\rightarrow}x_n\leq\limsup\limits_{n\in I,n\rightarrow+\infty}x_n$$
If $J_2$ is infinite
$$\limsup\limits_{n\in I,n\rightarrow+\infty}=\lim\limits_{n\in J_2,n\rightarrow+\infty}\max\{x_n,y_n\}\leq\limsup\limits_{n\in I,n\rightarrow+\infty}y_n$$
\section{Theorem}
Let $(a_N)_{n\in I}\in \mathbb{R} ^I\ l\in\mathbb{R} $. The following statements are equivalent
\begin{itemize}
    \item $(a_N)_{n\in I}$ converges to $l$
    \item $\limsup\limits_{n\in I,n\rightarrow+\infty}\lvert a_n-l\rvert=0$
\end{itemize}
\subsection{Proof}
$$\lvert a_n-l\rvert=\max\{a_n-l,l-a_n\}$$
$$\limsup\limits_{n\in I,n\rightarrow+\infty}\lvert a_n-l\rvert=\max\{(\limsup\limits_{n\in I,n\rightarrow+\infty}a_n)-l,l-(\liminf\limits_{n\in I,n\rightarrow+\infty}a_n)\}$$
$(1)\Rightarrow(2)$:

If $(a_n)_{n\in I }$ converges to $l$, then $\limsup\limits_{n\in I,n\rightarrow+\infty}a_n=\liminf\limits_{n\in I,n\rightarrow+\infty}a_n=l$\\
$(2)\Rightarrow(1)$:

If $\limsup\limits_{n\in I,n\rightarrow+\infty}\lvert a_n-l\rvert=0$ ,then $\limsup\limits_{n\in I,n\rightarrow+\infty}a_n\leq l\leq\liminf\limits_{n\in I,n\rightarrow+\infty}a_n$

Therefore:$\limsup\limits_{n\in I,n\rightarrow+\infty}a_n=\liminf\limits_{n\in I,n\rightarrow+\infty}a_n=l$
\section{Remark}
Let $(a_n)_{n\in I }$ be a sequence in $\mathbb{R} ,\ l\in \mathbb{R} $ \\\indent The sequence $(a_n)_{n\in I }$ converges to l iff $a_n-l=o(1),n\in I,n\rightarrow+\infty$
\section{Calculates on O(),o()}
\subsection{Plus}
Let  $(a_n)_{n\in I }$ $(a_n')_{n\in I }$ and  $(b_n)_{n\in I }$ be elements in $\mathbb{R} ^I$
\begin{itemize}
    \item If $a_n=O(b_n),a_n'=O(b_n),n\in I,n\rightarrow+\infty$\\then $\forall(\lambda,\mu)\in\mathbb{R} ^2\quad\lambda a_n+\mu a_n'=O(b_n),n\in I,n\rightarrow+\infty$
    \item If $a_n=o(b_n),a_n'=o(b_n),n\in I,n\rightarrow+\infty$\\then $\forall(\lambda,\mu)\in\mathbb{R} ^2\quad\lambda a_n+\mu a_n'=o(b_n),n\in I,n\rightarrow+\infty$
\end{itemize}
\subsection{Transform}
Let $(a_n)_{n\in I }$ and $(b_n)_{n\in I }$ be two sequence in $\mathbb{R} $ If $a_n=o(b_n),n\in I,n\rightarrow+\infty$,then $a_n=O(b_n),n\in I,n\rightarrow+\infty$
\subsection{Transition}
Let $(a_n)_{n\in I},(b_n)_{n\in I}$ and $(c_n)_{n\in I}$  be elements in $\mathbb{R} ^I$
\begin{itemize}
    \item If $a_n=O(b_n)$ and $ b_n=O(c_n),n\in I,n\rightarrow+\infty$\\then$a_n=O(c_n),n\in I,n\rightarrow+\infty$
    \item If $a_n=O(b_n)$ and $ b_n=o(c_n),n\in I,n\rightarrow+\infty$\\then$a_n=o(c_n),n\in I,n\rightarrow+\infty$
    \item If $a_n=o(b_n)$ and $ b_n=O(c_n),n\in I,n\rightarrow+\infty$\\then$a_n=o(c_n),n\in I,n\rightarrow+\infty$
\end{itemize} 
\subsection{Times}
Let $(a_n)_{n\in I},(b_n)_{n\in I},(c_n)_{n\in I},(d_n)_{n\in I}$ be sequences in $\mathbb{R} $\begin{itemize}
    \item If $a-N=O(b_n),c_n=O(d_n),n\in I,n\rightarrow+\infty$\\then $a_nc_n=O(b_nd_n),n\in I,n\rightarrow+\infty$
    \item If $a-N=o(b_n),c_n=O(d_n),n\in I,n\rightarrow+\infty$\\then $a_nc_n=o(b_nd_n),n\in I,n\rightarrow+\infty$
\end{itemize}
\section{On the limit}
Let $(a_n)_{n\in I},(b_n)_{n\in I}$ be elements of $\mathbb{R}^I $ that converges to $l\in \mathbb{R} $ and $l'\in\mathbb{R} $ respectively. Then:\begin{itemize}
    \item $(a_n+b_n)_{n\in I}$ converges to $l+l'$
    \item $(a_nb_n)_{n\in I}$ converges to $ll'$
\end{itemize}
\section{Prop}Let $a\in\mathbb{R} $ THen $a^n=o(n!)\quad n\rightarrow+\infty$
\subsection{Proof}Let $N\in \mathbb{N} $ such that $\lvert a\rvert<N$For $n\in\mathbb{N} $ such that $n\geq N$
$$0\leq\frac{\lvert a^n\rvert}{n!}=\frac{\lvert a^N\rvert}{N!}\cdot\\frac{\lvert a^n-N\rvert}{n!\over N!}\leq\frac{\lvert a^N\rvert}{N!}(\frac{\lvert a\rvert}{N})^n-N$$
And $0<\frac{\lvert a\rvert}<1\Rightarrow \lim\limits_{n\rightarrow+\infty}(\frac{\lvert a\rvert}{N})^n=0$. Therefore:
$$\lim\limits_{n\rightarrow+\infty}\frac{\lvert a^n\rvert}{n!}=0$$
namely:$$a^n=o(n!)$$
\section{Prop}$n!=o(n^n)\quad n\rightarrow+\infty$
\subsection{Proof}
Let $N\in\mathbb{N} _{\geq1}$\\$0\leq\frac{n!}{n^n}\leq\frac{1}{n}\Rightarrow \lim\limits_{n\rightarrow+\infty}\frac{n!}{n^n}=0$
\section{Prop}
Let $(a_n)_{n\in I},(b_n)_{n\in I}$ be the elements of $\mathbb{R}^I$ If the series $\sum\limits_{n\in I}b_n$ converges absolutely and if $on=O(b_n)\quad n\rightarrow+\infty$Then $\sum\limits_{n\in I}a_n$ converges absolutely
\subsection{Proof}
By definition $\sum\limits_{n\in I}\lvert b_N\rvert<+\infty $ If $\lvert a_N\rvert\leq M\lvert b_N\rvert$ fro $n\in I,n\geq N$ where $N\in \mathbb{N} $ Then $$\sum\limits_{n\in I}\lvert a_n\rvert=\sum\limits_{n\in I,n<N}\lvert a_n\rvert+\sum\limits_{n\in I,n\geq N}\lvert a_n\rvert\leq\sum\limits_{n\in I,n<N}\lvert a_n\rvert+\sum\limits_{n\in I,n\geq N}\lvert b_n\rvert<+\infty$$
\section{Theorem: d'Alembert ratio test}
Let $(a_N)_{n\in \mathbb{N} }\in(\mathbb{R} \backslash\{0\})^\mathbb{N} $\begin{itemize}
    \item If $\limsup\limits_{n\rightarrow+\infty}\lvert\frac{a_{n+1}}{a_n}\rvert<1$, then $\sum\limits_{n\in \mathbb{N} }a_n$ converges absolutely
    \item If $\liminf\limits_{n\rightarrow+\infty}\lvert\frac{a_{n+1}}{a_n}\rvert>1$, then $\sum\limits_{n\in \mathbb{N} }a_n$ does not converge (diverges)
\end{itemize}
\subsection{Proof}
\subsubsection*{(1)}Let $\alpha\in\mathbb{R} $ such that $\limsup\limits_{n\rightarrow+\infty}\lvert\frac{a_{n+1}}{a_n}\rvert<\alpha<1$, $alpha$ isn't a lower bound of $(\sup\limits_{n\geq N}\lvert\frac{a_{n+1}}{a_n}\rvert)_{N\in\mathbb{N} }$\\
So $\exists N\in\mathbb{N} $ such that $\sup\limits_{n\geq N}\lvert\frac{a_{n+1}}{a_n}\rvert<\alpha$Hence for $n\geq N\quad\lvert a_n\rvert\leq\alpha^{n-N}\lvert a_N\rvert$ since $$\frac{a_n}{a_N}=\frac{a_{N+1}}{a_N}
\frac{a_{N+2}}{a_{N+1}}...\frac{a_n}{a_{n-1}}$$
Therefore $a_n=O(\alpha^n)$ since $\sum\limits_{n\in\mathbb{N} }=\frac{1}{1-\alpha}<+\infty,\sum\limits_{n\in \mathbb{N} }a_n$ converge absolutely.
\subsection{Lemma}
If a series $\sum\limits_{n\in \mathbb{N} }a_n \in\mathbb{R} $ converges, then $\lim\limits_{n\rightarrow+\infty}a_n=0$
\subsubsection{Proof}
If $(\sum\limits_{i=0}^na_i)_{n\in \mathbb{N} }$ converges to some $l\in\mathbb{R} $ ,then $(\sum\limits_{i=0}^{n-1}a_i)_{n\in \mathbb{N},n\geq 1 }$ converges to $l$, too. Hence $\left(a_n=\left(\sum\limits_{i=0}^na_i\right)-\left(\sum\limits_{i=0}^{n-1}a_i\right)\right)_{n\in\mathbb{N} }$ converges to $l-l=0$
\subsection{(2)}Let $\beta\in \mathbb{R} $ such that $1<\beta<\liminf\limits_{n\rightarrow+\infty}\lvert\frac{a_{n+1}}{a_n}\rvert=\sup\limits_{N\in\mathbb{N} }\inf\limits_{n\geq N}\lvert\frac{a_{n+1}}{a_n}\rvert$\\
So there exists $N\in\mathbb{N} $ such that $\beta< \inf\limits_{n\geq N}\lvert \frac{a_{n+1}}{a_n}\rvert$\\
$\forall n\in \mathbb{N} ,n\geq N\quad \lvert \frac{a_{n+1}}{a_n}\rvert\geq\beta$\\
Hence $(\lvert a_n\rvert)_{n\in \mathbb{N} }$ is not bounded since $\lvert a_n\rvert\geq\beta^{n-N}\lvert a_n\rvert $\\
By the lemma: $\sum\limits_{n\in\mathbb{N} }a_n$ diverges.

\section{Prop}
Let $a\in \mathbb{R} ,a>1$ Then $n=o(a^n),n\rightarrow +\infty$
\subsection{Proof}
Let $\epsilon>0$ such that $a=(1+\epsilon)^2$
$$a^n=(1+\epsilon)^{2n}=(1+\epsilon)^n(1+\epsilon)^n\geq(1+n\epsilon)(1+n\epsilon)\geq\epsilon^2n^2$$
Hence
$$n\leq\frac{a^n}{\epsilon^2n}=o(a^n)$$
\subsection{Corollary}
Let $a>1,t\in \mathbb{R} _{\geq 0}$ Then $n^t=o(a^n),n\rightarrow+\infty$
\subsubsection{Proof}
Let $d\in\mathbb{N} _{\geq1}$ such that $t\leq d$Then $n^{t-d}\leq 1$
So
$$n^t=n^dn^{t-d}=O(n^d)$$
Let $b=\sqrt[d]{a}>1$$$n^d=o((b^n)^d)=o(a^n)$$
\indent Hence $n^t=o(a^n)$
\subsection{Corollary}
There exists $M\geq1$ such that $\forall x\in \mathbb{R} ,x\geq M,\ln(x)\leq x$
\subsubsection{Proof}
Let $a\in\mathbb{R} $ such that $1<a<e$
\section{Theorem: Cauchy root test}
Let $(a_n)_{n\in\mathbb{N} }$ be a sequence in $\mathbb{R} $. Let $\alpha=\limsup\limits_{n\rightarrow+\infty}\lvert a_n\rvert^{\frac{1}{n}}$\begin{itemize}
    \item If $\alpha<1$, then $\sum\limits_{n\in \mathbb{N} }a_n$ converges absolutely.
    \item If $a>1$ then $\sum\limits_{n\in \mathbb{N} }a_n$ diverges
\end{itemize}
\subsection{Proof}
\subsubsection{(1)}
Let $\beta\in \mathbb{R} ,\alpha<\beta<1$. There exists $N\in \mathbb{N} $ such that $\lvert a_N \rvert^{\frac{1}{n}}\leq\beta$ for $n\geq N$. That means  $\lvert a_n\rvert=O(\beta ^n)$ since $0<\beta<1,\sum\limits_{n\in \mathbb{N} }a_n$ converges absolutely.
\subsubsection{(2)}
If $\alpha>1$ then $\forall N\in \mathbb{N} \quad \exists n \geq N$ such that $\lvert a_n\rvert ^{\frac{1}{n}}\geq1$, since otherwise $\exists N\in \mathbb{N} \ \forall n\geq N,\lvert a_n\rvert^\frac{1}{n}<1$ contradiction\\
Hence $(\lvert a_n\rvert)_{n\in \mathbb{N} }$ cannot converge to 0.
\part{Topology}
\chapter{Absolute value and norms}
\section{Def}
Let K be a field . By absolute value on K, we mean a mapping $\lvert \cdot\rvert:K\rightarrow\mathbb{R} _{\geq0}$ that satisfies:\begin{itemize}
    \item[(1)] $\forall a\in K\quad\lvert a\rvert=0 $ iff $a=0$
    \item[(2)] $\forall (a,b)\in K^2\quad\lvert ab\rvert=\lvert a\rvert\cdot\lvert b\rvert$
    \item[(3)] $\forall (a,b)\in K^2\quad\lvert a+b\rvert\leq\lvert a\rvert+\lvert b\rvert$(triangle inequality)
\end{itemize}
\section{Notation}
$\mathbb{Q} $
Take a prime num $p$
$\forall \alpha \in \mathbb{Q} \backslash\{0\}$ there exists a integer $ord_p(\alpha)\frac{a}{b},$ where $\begin{aligned}
    a\in \mathbb{Z} \backslash\{0\}\\
    b\in\mathbb{N} \backslash\{0\}
\end{aligned},p\nmid a,p\nmid b$
\section{Prop}
$$\lvert\cdot\rvert:\begin{aligned}
    &\mathbb{Q} \rightarrow\mathbb{R} _{\geq0}\\
    &\alpha\mapsto\left\{\begin{aligned}
        &p^{-ord_p(\alpha)}&if\ \alpha\not=0\\
        &0 &if\ \alpha=0
    \end{aligned}\right.
\end{aligned}$$ \indent is a absolute value on $\mathbb{Q} $
\subsection{proof}
\begin{itemize}
    \item [(1)] Obviously
    \item [(2)]If $\alpha=p^{ord_p(\alpha)}\frac{a}{b},\beta=p^{ord_p(\beta)}\frac{c}{d}\ \ p\nmid abcd$\\$\alpha\beta=p^{ord_p(\alpha)+ord_p(\beta)}\frac{ac}{bd}\quad p\nmid ac,p\nmid bd$
    \item [(3)]$\alpha+\beta=p^{ord_p(\alpha)}\frac{a}{b}+p^{ord_p(\beta)}\frac{c}{d}$\\Assume $ord_p(\alpha)\geq ord_p(\beta)$\\$\alpha+\beta\\=p^{ord_p(\beta)}\left(p^{ord_p(\alpha)-ord_p(\beta)}\frac{a}{b}+\frac{c}{d}\right)\\=p^{ord_p(\beta)}\frac{p^{ord_p(\alpha)-ord_p(\beta)}ad+bc}{bd}\quad p\nmid bd$\\So $$ord_p(\alpha+\beta)\geq ord(\beta)$$
    Hence $ord_p(\alpha+\beta)\geq \min\{ord_p(\alpha),ord_p(\beta)\}$\\So $\lvert\alpha+\beta\rvert _p=p^{-ord_p(\alpha+\beta)}\leq\max\{p^{-ord_p(\alpha)},p^{-ord_p(\beta)}\}=\\\max\{\lvert \alpha\rvert _p,\lvert \alpha\rvert _p\}\leq\lvert \alpha\rvert _p,\lvert \alpha\rvert _p $
\end{itemize}\
\chapter{Quotient Structure}
\section{Def}

Let X be a set and $\thicksim$  be a binary relation on X\\
If :\begin{itemize}
    \item $\forall x\in X,x\thicksim x$
    \item $\forall (x,y)\in X\times X,$ if $x\thicksim y$ then $y\thicksim x$
    \item $\forall (x,y,z)\in X^3$, if $x\thicksim y,y\thicksim z$ then $x\thicksim z$
\end{itemize}
then we say that $\thicksim$  is an equivalence relation
\section{equivalence class}
$\forall x\in X$ we denote by $[x]$ the set $\{y\in X\mid y\thicksim x\}$ and call it the equivalence class of $x$ on $X$.Let $X/\thicksim $ be the set $\{[x]\mid x\in X\}$
\section{Prop.}
Let X be a set and $\thicksim$  be an equivalence relation on X
\begin{itemize}
    \item[(1)] $\forall x\in X,y\in [x]$ on has $[x]=[y]$
    \item[(2)] If $\alpha$ and $\beta$ are elements of $X/\thicksim $ such that $\alpha\not=\beta$ then $\alpha\cap\beta =\varnothing$
    \item[(3)] $X=\bigcup\limits_{\alpha\in X/\thicksim }\alpha$
\end{itemize}
\subsection{Proof}
\begin{itemize}
    \item[(1)] Let $z\in [y]$. Then $y\thicksim z$. Since $y\in [x]$ on has $x\thicksim y$\\Therefore ,$x\thicksim z$ namely $z\in [x]$. This proves $y[]\subseteq [x]$. Moreover ,since $x\thicksim y$ , one has $x\in [y]$. Hence $[x]\subseteq[y]$. Thus we obtain $[x]=[y]$
    \item[(2)] Suppose that $\alpha\cap\beta\not=\varnothing,y\in \alpha\cap\beta$\\By (1),$\alpha=[y],\beta=[y]$, Thus leads to a contradiction.
    \item[(3)] $\forall x\in X\quad x\in [x]$ Hence $x\in \bigcup\limits_{\alpha\in X/\thicksim }\alpha$Hence $X\subseteq\bigcup\limits_{\alpha\in X/\thicksim }\alpha$.Conversely, $\forall \alpha\in X/\thicksim ,\alpha$ is a subset of $X$. Hence $\bigcup\limits_{\alpha\in X/\thicksim }\alpha\subseteq X$.Then $X=\bigcup\limits_{\alpha\in X/\thicksim }\alpha$
\end{itemize}
\section{Def}
Let G be a group and X be a set \\
We call left/right action of G on X ant mapping $G\times X\rightarrow X:(g,x)\mapsto gx/(g,x)\mapsto xg$ that satisfies:
\begin{itemize}
    \item $\forall x\in X\quad 1x=x\ /\ x1=x$
    \item $\forall(g,h)\in G^2,x\in X\quad g(hx)=(gh)x\ /\ (xg)h=x(gh)$
\end{itemize}
\section{Remark}
If we denote by $G^{op}$ the set G equipped with the composition law :$$G\times G\rightarrow G$$$$(g,h)\mapsto hg$$
The a right action of G on X is just a left action of $G^{op}$ on X.
\section{Prop}
Let G be a group and X be a set . Assume given a left action of G on X. Then the binary relation $\thicksim$ on X defined as $x\thicksim y$ iff $\exists g\in G\quad y=gx$ is an equivalence relation
\section{Notation on Equivalence Class}
We denote by $G/X$ the set $X/\thicksim $ $\forall x\in X$ the equivalence class of $x$ is denoted as $Gx/xG$ or $orb_G(x)$ call the orbit of x under the action of G

\section{Proof}
\begin{itemize}
    \item $\forall x\in X\quad x=1x$ so $x\thicksim x$
    \item $\forall (x,y)\in X^2$ if $y=gx$ for same $g\in G$ then $g^{-1}y=g^{-1}(gx)=(g^{-1}g)x=1x=x.(y\thicksim x)$
    \item $\forall(x,y,z)\in X^3,$if $\exists(g,h)\in G^2$ , such that $y=gx$ and then $z=h(gx)=(hg)x$ So $x\thicksim z$
\end{itemize}
\section{Quotient set}
Let X be a set and $\thicksim $ be an equivalence relation, the mapping $X\rightarrow X/\thicksim : (x\in X)\mapsto[x]$ is called the projection mapping.\\$X/\thicksim  $ is called the quotient set of X by equivalence relation $\thicksim $
\subsection{Example}
Let G be a group and H be a subgroup of G.Then the mapping $$H\times G\rightarrow G$$$$(h,g)\mapsto hg/(h,g)\mapsto gh$$
is a left/right action of H on G. Thus we obtain         two quotient sets $H/ G$ and $G/ H$
\section{Def}
Let G be a group and H be a subgroup of G. Ig $\forall g\in G,h\in H\quad ghg^{-1}\in H,$ Then we say that H is a normal subgroup of G
\section{Remark}
$\forall g\in G, gH=Hg$, provided that H is a normal subgroup of G. In fact $\forall h\in $,\begin{itemize}
    \item $\exists h'\in H$ such that $ghg^{-1}=h'$ Hence $gh=h'g$. This shows $gH\subseteq Hg$
    \item $\exists h''\in H$ such that $g^{-1}hg=h''$ Hence $hg=gh''$. This shows $Hg\subseteq gH$
\end{itemize}
Thus $gH=Hg$
\section{Prop}
If G is commutative, any subgroup of G is normal
\section{Theorem}
Let G be a group and H be a normal subgroup of G. Then the mapping $$G/ H\times H/ G\rightarrow G/ H$$$$(xH,Hx)\mapsto(xy)H$$
is well defined and determine a structure of group of quotient set $G/ H$ Moreover the projection mapping $$\pi:G\rightarrow G/ H$$$$ x\mapsto xH$$ is a morphism of groups.
\subsection{Proof}
\begin{itemize}
    \item If $xH=x'H,yH=y'H$ then $\exists h_1\in H,h_2\in H$ such that $x'=xh_1,y'=yh_2$ Hence $x'y'=xh_1yh_2=(xy)(y^{-1}h_1y)h_2$. For $y^{-1}h_1y,h_2\in H$ then $(x'y')H=(xy)H.$ So the mapping is well defined.
    \item $\forall (x,y,x)\in G^3\quad (xH)(yH\cdot zH)=xH((yx)H)=(x(yz)H=((xy)z)H=((xy)H)zH=(xH\cdot yH)zH)$
    \item $\forall x\in G\quad 1H\cdot xH=xH\cdot 1H=xH\quad x^{-1}HxH=xHx^{-1}H=1H$
    \item $\pi(xy)=(xy)H=xH\cdot yH=\pi(x)\pi(y)$
\end{itemize}
\section{Def}
Let K be a unitary ring and E be a left K-module. We say that a subgroup F og $(E,+)$ is a left sub-K-module of E if $\forall(a,x)\in K\times F, ax\in F$
\section{Prop}
Let K be a unitary ring , E be a left K-module and F be a sub-K-module. Then the mapping $$K\times(E/ F)\rightarrow E / F$$$$(a,[x])\mapsto[ax]$$
is well defined , and defines a left-K-module structure on $E/F$. Moreover, the projection mapping $pi:E\rightarrow E/F$ is a morphism of left-K-modules
\subsection{Proof}
Let x and x' be elements of E such that $[x]=[x']$, that meas: $x'-x\in F$ Hence $a(x'-x)=ax'-ax\in F$ So $[ax]=[ax']$\\
Let us check that $E/F$ forms a left K-module.
\begin{itemize}
    \item $a([x]+[y])=a([x+y])=[a(x+y)]=[ax+ay]=[ax]+[ay]$
    \item $(a+b)[x]=[(a+b)x]=[ax+bx]=[ax]+[bx]$
    \item $1[x]=[1x]=[x]$
    \item $a(b[x])=a[bx]=[a(bx)]=[(ab)x]=(ab)[x]$
\end{itemize}
By the provided proposition, $\pi $ is a morphism of groups. Moreover $\forall x\in E,a\in K\quad \pi(ax)=[ax]=a[x]=a\pi(x)$
\section{Def}
Let A be a unitary ring . We call two-sided ideal any subgroup I of $(A,+)$ that satisfies : $\forall x\in I,a\in A\quad \{ax,xa\}\subseteq I$() (I is a left and right sub-K-module of A)
\section{Theorem}
Let A be a unitary ring and I be a two sided ideal of A . The mapping $$(A/I)\times(A/I)\rightarrow A/I$$$$([a],[b])\mapsto[ab]$$ is well defined. Moreover , $A/I$ becomes a unitary ring under the addition and this composition law, and the projection mapping $$A\stackrel{\pi}{\longrightarrow} A/I$$ is a morphism of unitary ring (if is a morphism of additive groups and multiplicative monoids, namely $\pi(a+b)=\pi(a)+\pi(b), \pi(ab)=\pi(a)\pi(b),\pi(1)=1)$
\subsection{Proof}
If $a'\thicksim a,b'\thicksim b$ that means $a'-a\in I,b'-b\in I$ then $a'b'-ab=a'b'-a'b+a'b-ab=a'(b'-b)+(a'-a)b$. For $(a'-a),(b'-b)\in I$, then $a'b'-ab\in I$ Therefore $a'b'\thicksim ab$
\subsection{Reside Class}
Let $d\in \mathbb{Z} $ and $d\mathbb{Z} =\{n\in \mathbb{Z} \mid\exists m\in\mathbb{Z} ,n=dm\}$ $d\mathbb{Z} $ is a two sided ideal of $\mathbb{Z} $ If $m\in \mathbb{Z} $, for any $a\in \mathbb{Z} \quad adm=dma\in d\mathbb{Z}$\\
Denote by $\mathbb{Z} /d\mathbb{Z} $ the quotient ring. The class of $n\in \mathbb{Z} $ in $\mathbb{Z} /d\mathbb{Z} $ is called the reside class of n modulo d\\
If A is a commutative unitary ring, a two sided ideal of A is simply called an ideal of A
\section{Theorem}
Let $f:G\rightarrow H$ be a morphism of groups \begin{itemize}
    \item[(1)] $Im(f)$ is a subgroup of H
    \item[(2)] $\ker(f):=\{x\in G\mid f(x)=1_H\}$ is a normal subgroup of G
    \item[(3)] The mapping $$\begin{aligned}
    \widetilde{f}:&G/Ker(f)\rightarrow Im(f)\\
    &[x]\mapsto f(x)
    \end{aligned}$$
    is well defined and is an isomorphism of groups
    \item[(4)] $f$ is injective iff $\ker(f)=\{1_G\}$
\end{itemize}
\subsection{Proof}
\begin{itemize}
    \item[(1)] Let $\alpha$ and $ \beta$ be elements of $Im(f)$. Let $(x,y )\in G^2$ such that $\alpha =f(x),\beta=f(y)$ Then $\alpha\beta^{-1}=f(x)f(y)^{-1}=f(xy^{-1})\in Im(f)$ So $Im(f)$ is a subgroup
    \item[(2)] Let x and y be elements of $\ker (f)$. \\One has $f(xy^{-1}) = f(x)f(y)^{-1} = 1_H1_H^{-1} = 1_H$ \\So $xy^{-1}\in \ker f.$ Hence $\ker f$ is a subgroup of G\\Let $x\in \ker f,y\in G$. \\One has $f(yxy^{-1}) = f(y)f(x)f(y)^{-1} = f(y)f(y)^{-1} = 1_H$ Hence $yxy^{-1}\in \ker f$. So $\ker f$ is a normal subgroup
    \item[(3)] If $x\thicksim y$ then $\exists z\in \ker f $ such that $y=xz$ Hence $f(y)=f(x)f(z)=f(x)1_H=f(x)$ So $f$ is well defined. \\Moreover $\widetilde{f}([x][y])=\widetilde{f}([xy])=f(xy)=f(x)f(y)=f([x])f([y])$ Hence $\widetilde{f}$ is a morphism of groups.\\By definition $Im(\widetilde{f})=Im(f)$\quad If x and y are elements of G such that $f(x)=f(y)$ then $f(xy^{-1})=1_H$ \\Hence $xy^{-1}\in \ker f$ Since $x=(xy^{-1})y,x\sin y$ that means $[x]=[y]$ \\Therefore $\widetilde{f}$ is injective.
    \item[(4)] If $f$ is injective ,$\forall x\in \ker f\quad f(x)=1_H=f(1_G)$, so $x=1_G$. Therefore $\ker f\{1_G\}$ \\Conversely , suppose that $\ker f =\{1_G\}\quad \forall (x,y)\in G^2$ if $f(x)=f(y)$ then $f(x)f(y)^{-1}=1_H$. Hence $xy^{-1}=1_G, x=y$
\end{itemize}
\section{Theorem}
Let K be a unitary ring and $f:E\rightarrow F $ be a morphism of left K-modules. Then 
\begin{itemize}
    \item[(1)] $Im(f)$ is a left-sub-K-module of F
    \item[(2)] $\ker(f)$ is a left-sub-K-module of E
    \item[(3)] $\begin{aligned}
        \widetilde{f}:&E/\ker f\rightarrow Im (f)\\
        &[x]\mapsto f(x)
    \end{aligned}$ is a isomorphism of left K-modules
\end{itemize}
\subsection{Proof}
\begin{itemize}
    \item[(1)] $\forall x\in E,\quad f(ax)=af(x)$ So $ af(x)\in Im(f)$
    \item [(2)]
    \item [(3)]
\end{itemize}
\chapter{Topology}
\section{Def}
Let X be a set. We call topology on X any subset $\mathcal{J} $ of $\wp (x)$ that satisfies:\begin{itemize}
    \item $\varnothing\in \mathcal{J} $ and $X\in \mathcal{J} $
    \item If $(u_i)_{i\in I}$ is an arbitrary family of elements in $\mathcal{J} $, then $\bigcup\limits_{i\in I}u_i\in \mathcal{J} $
    \item If $u$ and $v$ are elements of $\mathcal{J} $, then $u\cap v\in \mathcal{J} $
\end{itemize}
\section{Remark}
If $(u_i)^n_i=1$ is a finite family of elements of $\mathcal{J} $, then $\bigcap\limits_{i=1}^{n}u_i\in \mathcal{J}$(by induction, this follows from (3))
\subsection{Example}
$\{\phi,X\}$ is a topology. call the trivial topology on $\wp (X)$ is a topology called the discrete topology.
\section{Def}
Let X be a set. We call metric on $X$ any mapping $d:\begin{aligned}
    X\times X\rightarrow\mathbb{R} _{\geq 0}\\(x,y)\mapsto d(x,y)
\end{aligned}$,that satisfies\begin{itemize}
    \item $d(x,y)=0$ iff x=y
    \item $\forall(x,y)\in X^2, d(x,y)=d(y,x)$
    \item $\forall(x,y,z)\in X^3\quad d(x,z)\leq d(x,y)+d(y,z)$(triangle inequality)
\end{itemize}
$(X,d)$ is called a metric space
\subsection{Example}
Let X be a set$$\begin{aligned}
    d:&X^2\rightarrow\mathbb{R} _{\geq0}\\ & d(x,y)=\left\{\begin{aligned}
    &1 & if\ x\not=y\\
    &0 & if\ x=y
    \end{aligned}\right.
\end{aligned}$$is a metric
\section{Def}
Let $(X,d)$ be a metric space. For any $x\in X,\epsilon \in \mathbb{R} _{\geq0}$, let $B(x,\epsilon):=\{y\in X\mid d(x,y)\leq\epsilon \}$ We call the open ball of radius $\epsilon$ centered at $x$
\subsection{Example}
Consider $(\mathbb{R} ,d)$ with $d(x,y)=\lvert x-y\rvert$,then $B(x,\epsilon)={]x-\epsilon,x+\epsilon[}$
\section{Prop.}
Let (X,d) be a metric space . let $\mathcal{J} _d$ be the set of $U\subseteq X$ such that $\forall x\in U\exists \epsilon >0\quad B(x,\epsilon)\subseteq U$ THen $\mathcal{J} _d$ is a topology on X
\subsection{Proof}
\begin{itemize}
    \item $\varnothing\in \mathcal{J} _d\quad X\in \mathcal{J} _d$
    \item Let $(u_i)_{i\in I}$ be a family of elements of $\mathcal{J} _d$ Let $U=\bigcup\limits_{i\in I}u_i$, $\forall x\in U,\exists i\in I$ such that $x\in u_i$. Since $u_i\in \mathcal{J} _d,\exists \epsilon>0$ such that $B(x,y)\subseteq u_i\subseteq U$ Hence $U\in \mathcal{J} _d$
    \item Let U and V be elements of $\mathcal{J} _d$ Let $x\in U\cap V$ $\exists a,b\in \mathbb{R} _{\geq0}$ such that $B(x,a)\subseteq U, B(x,b)\subseteq V$ Taking $\epsilon=\min\{a,b\},$ Then $B(x,\epsilon)=B(x,a)\cap B(x,b)\subseteq U\cap V$ Therefore $U\cap V\in \mathcal{J} _d$ 
\end{itemize}
\section{Def}
$\mathcal{J} _d$ is called the topology induced by the metric d
\section{Def}
We call topology space any pair $(X,\mathcal{J})$ where X is a set and $\mathcal{J}$ is a topology on X

Given a topological space $(X,\mathcal{J})$ If $U\in \mathcal{J}$ then we say that $U$ is an open subset of X. If $F\in \wp(X)$ such that $X\backslash F\in \mathcal{J}$, then we say that $F$ is closed subset of $X$\\
If there exists $d$ a metric on X such that $\mathcal{J}=\mathcal{J}_d$ then we say that $\mathcal{J}$ is metrizable
\subsection{Example}
Let X be a set . The discrete topology on X is metrizable. In fact,m if d denote the metric defined as $d(x,y)=\left\{\begin{aligned}
    &1 & if\ x\not=y\\
    &0 & if\ x=y
    \end{aligned}\right.$\\
$\forall x\in X\quad B(x,1)=\{x\}$ So $\{x\}\in \mathcal{J}_d$ Hence $\forall A\subseteq X\quad A=\bigcup\limits_{x\in A}\{x\}\in \mathcal{J}_d$ 
\section{Axiom of choice}
For any set I and any family $(A_i)_{i\in I}$ of non-empty sets , there exists a mapping $f:I\rightarrow \bigcup\limits_{i\in I}A_i$ such that $\forall i\in I, f(i)\in A_i$
\section{Def} Let $(X,\leq)$ be a partially ordered set If $\forall A\subseteq X$ A is non-empty , there exists a least element of A then we say that $(X,\leq)$ is a well ordered set.
\section{Theorem }
For any set X, there exists an order relation $\leq$ on such that $(X,\leq)$ forms a well ordered set.
\section{Zorn's lemma}
Let $(X,\leq)$ be a partially ordered set . If $\forall A\subseteq X$ that is totally ordered with respect to $\leq $, there exists an upper bound of A inside X. Then , there exists a maximal element $x_0$ of $X$($\forall y\in X, y>x_0$ does not hold)
\section{Prop.}
Let $(X,\leq)$ be a well ordered set , $y\not\in X. $ We extends $\leq$ to $X\cup\{y\}$, such that $\forall x\in X,x<y.$ Then $(X\cup\{y\},\leq)$ is well ordered.
\section{Proof}
Let $A\subseteq X\cup\{y\},A\not=\varnothing$. If $A=\{y\}$ then $Y$ is the least element of A. If $A\not=\{y\}$ then $B=A\backslash\{y\}$ is non-empty. Let $b$ be the least element of B. Since $b<y$ it's also the least element of A
\section{Def: Initial Segment}
Let $(X,\leq)$ be a well ordered set. $S\subseteq X,$ If $\forall s\in S,x\in X\quad x<s$ initial $x\in S(X_{<s}\subseteq S)$, then we say that S is an initial segment of X

If S is a initial segment such that $S=X$ then we sat that S is a proper initial segment.
\section{Example}$\forall x\in X\quad X_{<x}=\{s\in X\mid s<x\}$ Then $X_{<x}$ is a proper initial segment of X.
\section{Prop.}Let $(X,\leq)$ be a well ordered set , If $(S_i)_{i\in I}$ is a family of initial segment of X, then $\bigcup\limits_{i\in I}S_i$ is an initial segment of X
\section{Proof}
$\forall s\in \bigcup\limits_{i\in I}S_i,\exists i\in I$ such that $s\in S_i,i\in I$ Therefore $X_{<s}\subseteq S_i\subseteq\bigcup\limits_{i\in I}S_i$
\section{Prop.}
Let $(X<\leq)$ be a well erodered set.
\begin{itemize}
    \item[(1)] Let S be a proper initial segment of X, $x= \min(X\backslash S)$ Then $S=X_{<x}$
    \item[(2)] $\begin{aligned}
        &X\rightarrow\wp(X)\\ &x\mapsto X_{<x}
    \end{aligned}$
    \item[(3)] The set of all initial segments of X forms a well ordered subset of $(\wp(x),\subseteq)$
\end{itemize}
\section{Proof}
\begin{itemize}
    \item [(1)] $\forall s\in S$ if $x\leq s$ then $x\in S$ contradiction.Hence $s<x$, This shows $S\subseteq X_{<x}$ Conversely , if $t\in X, t\not\in X\backslash S$ Hence $t\in S.$ Hence $X_{<x}\subseteq S$
    \item [(2)]Let $x,y\in X,x<y$ By definition $X_{<x}\subseteq X_{<y}$ Moreover $x\in X_{<y}\backslash X_{<x}$ So $X_{<x}\subsetneqq X_{<y}$
    \item [(3)]Let $\mathcal{F}\subseteq\wp(X)$ be a set of initial segments. $\mathcal{F}\not=\varnothing$. Then there exists $A\subseteq X$ such that $\mathcal{F}\backslash \{x\}=\{X_{<x}\mid x\in A\}$ If $A=\varnothing$ then $\mathcal{F}=\{X\}$, and $\{X\}$is the least element of $\mathcal{F}$ . Otherwise $A\not=\varnothing$ and A has a least element a. Then by(2) $X_<a$ is the least element of $\mathcal{F}$
\end{itemize} 
\section{Lemma}
Let $(X,\leq)$ be a well ordered set, $f:X\rightarrow X$ be a strictly increasing mapping. Then $\forall x\in X, x\leq f(x)$
\subsection{Proof}
Let $A=\{x\in X\mid f(x)<x\}$ If $A\not=\varnothing$, let a be the least element of A. By definition $f(a)<a$. Hence $f(f(a))<f(a)$ since $f$ is strictly increasing . This shows $f(a)\in A$. But a is the least element of A, $f(a)<a$ cannot hold: contradiction.
\section{Prop}
Let $(X,\leq)$ be a well ordered set, S and T be two initial segment of X . If $f:S\rightarrow T$ is a bijection that's strictly increasing , then $S=T,f=Id_S$
\subsection{Proof}
We may assume $T\subseteq S$.Let $l:T\rightarrow S$ be the induction mapping and $g=l\circ f: S\rightarrow S$. Since $g$ is strictly increasing , by the lemma ,$ \forall s\in S,s\leq g(s)=f(s)\in T$. Since T is an initial segment, $s\in T.$ Hence $S=T$ \\Apply the lemma to $f^{-1}$ we get $\forall s\in S, s\leq f^{-1}(s)$ Hence $f(s)\leq s$ Therefore $f(s)=s$
\section{Def}
Let $(X,\leq)$ and $(Y,\leq)$ be partially ordered sets. If  $\exists f: X\rightarrow Y $ that's increasing and bijective, we say that $(X,\leq)$ and $(Y,\leq)$ are isomorphic
\section{Def}
Let $(X,\leq)$ and $(Y,\leq)$ be well ordered sets. If $(X,\leq)$ is isomorphic to an initial segment of Y. We note $X\preceq Y$ or $Y\succeq X$. If X is isomorphic to Y, we note $X\thicksim Y$. If $X\preceq Y$ but $X\not\sim Y$, we note $X\prec Y$ or $Y\prec X$
\section{Prop.}
Let X and Y be well ordered sets. Among the following condition, one and only one holds.$$X\prec Y\quad X\sim Y\quad X\succ Y$$
\subsection{Proof}
We construct a correspondence $f$ from X to Y, such that $(x,y)\in \Gamma_f$, iff $X_{<x}\sim Y_{<y}$\\By the last proposition of Oct. 11, $f$ is a function.\begin{itemize}
    \item If $a,b\in Dom(f)^2,a<b$,then $X_{<a}\subsetneqq X_{<b}$\\
By definition, $Y_{<f(b)}\sim X_{<b}\quad Y_{<f(a)}\sim X_{<a}$\\
Hence $Y_{<f(a)}$ is isomorphic to a proper initial segment of $Y{<f(b)}$. Therefore $Y_{f(a)}$ is a proper initial segment of $Y_<{f(B)}$. We then get $f(a)<f(b)$. Thus $f$ is strictly increasing.
    \item Let $a\in Dom(f)$ Let $x\in X,x<a$ Then $X_{<x}$ is a initial segment of $X_{<a}\sim Y_{<f(a)}$ Hence $\exists y\in Y\quad X_{<x}\sim Y_{<y}$ This shows that $x\in Dom(f)$. Hence $Dom(f)$ is an initial segment of X. Applying this to $f^{-1}$, we get : $Im(f)=Dom(f)$ is an initial segment of Y
    \item Either $Dom (f)=X$ or $Im(f)=Y$.\\ Assume that $x\in X\backslash Dom (f),y\in Y\backslash Im(f)$ are respectively the least elements of $X\backslash Dom (f)$ and $Y\backslash Im(f)$. \\Then we get $Dom(f)=X_{<x},Im(f)=Y_{<y}$. \\We obtain $X_{<x}\sim Y_{<y},(x,y)\in \Gamma_{f}$. Contradiction
    \item\indent \begin{itemize}
    \item [Case 1] $Dom(f)=X,Im(f)\subsetneqq Y\quad X\prec Y$
    \item [Case 2]$Dom(f)\subsetneqq X, Im(f)=Y\quad X\succ Y$
    \item [Case 3]$Dom(f)=X,Im(f)=Y\quad X\sim Y$
\end{itemize}
\end{itemize}
\section{Lemma}
Let $(X,\leq)$ be a partially ordered set . $\mathfrak{S} \subseteq\wp(X)$. Assume that\begin{itemize}
    \item $\forall A\in \mathfrak{S} , (A,\leq) $ is a well-ordered set .
    \item $\forall (A,B)\in \mathfrak{S} ^2$, either A is an initial segment of B, or B is a initial segment of A.
\end{itemize}
Let $Y=\bigcup\limits_{A\in \mathfrak{S} }A.$ Then $(Y,\leq)$ is a well ordered set, and ,$\forall A\in \mathfrak{S} ,A$ is an initial segment of Y.
\subsection{Proof}
\begin{itemize}
    \item Let $A\in \mathfrak{S} ,x\in A,y\in Y,y<x$.Since $Y=\bigcup\limits_{B\in \mathfrak{S} }B$,$\exists B\in \mathfrak{S} $, such that $y\in B$. If $y\not\in A$ then $B\not\subseteq A$. Hence A is an initial segment of B. Hence $y\in A$. Contradiction
    \item Let $Z\subseteq Y,Z\not=\varnothing$. Then $\exists A\in \mathfrak{S} , A\cap Z\not=\mathfrak{S} $. Let m be the least element of $A\cap Z.$ Let $z\in Z,B\in \mathfrak{S} $, such that $z\in B$. If $z\in A$, then $m\leq z$. If $z\not\in A$ , then A is an initial segment of B.
\end{itemize}
Since B is well ordered , if $m\not\leq z$ then $z<m$. Since $m\in A$, we het $z\in A$. Contradiction.

Therefore, m is the least element of Z.
\section{Theorem(Zorn's lemma)}
Let $(X,\leq)$ be a partially ordered set. Suppose that any well-ordered subset of X has an upper bound on X, the X has a maximal element(a maximal element m of $\{x\mid x> m\}=\varnothing)$
\subsection{Proof}
Suppose that X doesn't have any maximal element. $\forall A\in \omega.\exists f(A) $ such that $\forall a\in A,a<f(A)$\\Let $\omega=\{\text{well ordered subset of X}\}$. (guaranteed by axiom of choice) Let $f:\omega\rightarrow X$ such that $f(A)$ is an upper bound of $A\in \omega$. If $A\in \omega$ satisfies $\forall a\in A a=f(A_{<a})$ , we say that $A$ is a $f$-set\\
Let $\mathfrak{S} =\{f-sets\}$ Note that $\varnothing\in \mathfrak{S} $ If $A\in \mathfrak{S} $, $A\cap\{f(A)\}\in \mathfrak{S} $

In fact, if $a\in A$, then $A_{<a}=(A\cup\{f(A)\})_{<a}$. If $a=f(A)\not\in A$, then $(A\cup\{f(A)\})_{<a}=A$

Let A and B be elements of $\mathfrak{S}$. Let I be the union of all common initial segments of A and B. This is also a common initial segment of A and B.\\If $I\not=A$ and $I\not=B$, then $\exists (a,b)\in A\times B, I=A_{<a}=B_{<b}\quad f(I)=f(A_{<a})=f(B_{<b})$. Hence $a=b$. Then $I\cup\{a\}$ is also a common initial segment of A and B, contradiction.

By the lemma , $Y:=\bigcup\limits_{A\in\mathfrak{S}}A$ is well-ordered , and any $A\in \mathfrak{S}$ is an initial segment of Y.

$\forall a\in Y,\exists A\in\mathfrak{S}\quad a\in A$. Since A is an initial segment of Y. $A_{<a}=Y_{<a}$. Hence $f(Y_{<a})=f(A_{<a})=a$. Hence $y\in \mathfrak{S}$. Thus Y is the greatest element of $(\mathfrak{S},\subseteq)$.
However, $Y\cup\{f(Y)\}\in \mathfrak{S}$. Hence $f(y)\in Y$.

If $f(y)$ is not a maximal element of X, $\exists x\in X,f(y)<x$
\chapter{Filter}
\section{Def}
Let Xbe a set. We call filter if X any $\mathcal{F}\subseteq \wp(x)$ that satisfies:
\begin{itemize}
    \item [(1)]$\mathcal{F}\not=\varnothing,\varnothing\not\in \mathcal{F}$
    \item [(2)]$\forall A\in\mathcal{F},\forall B\in \wp(X)$, if $A\subseteq B$, then $B\in \mathcal{F}$
    \item [(3)]$\forall (A,B)\in \mathcal{F}\times\mathcal{F},A\cap B\in \mathcal{F}$
\end{itemize}
\subsection{Example}
\begin{itemize}
    \item [(1)]Let $Y\subseteq X, Y\not=\varnothing.$ $\mathcal{F}_Y:=\{A\in \wp(X)\mid Y\subseteq A\}$ is a filter, called the principal filter of Y.
    \item [(2)]Let X be an infinite set.$$\mathcal{F}_{Fr}(X):=\{A\in \wp(X)\mid X\backslash A\text{is infinite}\}$$ is a filter called the Fréchet filter of X.
    \item [(3)]Let $(X,\mathcal{J} )$ be a topological space, $x\in X$ We call neighborhood of $x$ any $V\in \wp(X)$ such that $\exists u\in\mathcal{J} ,$ satisfying $x\in U\subseteq V$. Then $\mathcal{V}=\{\text{neighborhoods of x}\}$ is a filter. 
\end{itemize}

\section{Def: Filter Basis}
Let X ba a set. $\mathscr{B}\subseteq\wp(X).$ If $\varnothing\not\in\mathscr{B}$ and $\forall (B_1,b_2)\in\mathscr{B}^2,\exists B\in\mathscr{B}$, such that $B\subseteq B_1\cap B_2$. We say that $\mathscr{B}$ is a filter basis.
\subsection{Remark}
If $\mathscr{B}$ is a filter basis, then $\mathcal{F}(\mathscr{B})=\{A\subseteq X\mid \exists B\in\mathscr{B}\quad B\subseteq A\}$ is a filter
\subsection{Proof}
$\varnothing\not\in \mathcal{F}(\mathscr{B}),\mathcal{F}(\mathscr{B})\not=\varnothing$ since $0\not=B\subseteq \mathcal{F}(\mathscr{B})$. If $A\in\mathcal{F}(\mathscr{B}),A'\in\wp(X)$such that $A\subseteq A'$, then $\exists B\in \mathscr{B}$ such that $B\subseteq A\subseteq A'$. Hence $A'\in\mathcal{F}(\mathscr{B})$ If $A_1,A_2\in\mathcal{F}(\mathscr{B})$, then $\exists(B_1,B_2)\in\mathscr{B}^2$ such that $B_1\subseteq A_1, B_2\subseteq A_2$. Since $\mathscr{B}$ is a filter basis, $\exists B\in\mathscr{B}$ such that $B\subseteq B_1\cap B_2\subseteq A_1\cap A_2$ Hence $A_1\cap A_2\in A_1\cap A_2\in \mathcal{F}(\mathscr{B})$
\subsection{Example}
\begin{itemize}
    \item Let $Y\subseteq X, Y\not=\varnothing$ \\$\mathscr{B}=\{Y\}$ is a filter basis. $\mathcal{F}(\mathscr{B})=\mathcal{F}_Y=\{A\subseteq X\mid Y\subseteq A\}$
    \item Let $(X,\mathcal{J})$ be a topological space $x\in X$. If $\mathscr{B}_x$ is a filter basis such that $\mathcal{F}(\mathscr{B_x})=\mathcal{V}_x=\{\text{neighborhood of }x\}$, then we say that $\mathscr{B}_x$ is a neighborhood basis of x
\end{itemize}
\section{Remark}
Let $\mathscr{B}_x$ is a neighborhood basis of x iff\begin{itemize}
    \item $\mathscr{B}_x\subseteq\mathcal{V}_x$
    \item $\forall V\in \mathcal{V}_x\quad \exists U\in \mathscr{B}_x$ such that $U\subseteq V$
    \item Let $(X,d)$ be a metric space , $x\in X$$\forall \epsilon>0$, Let $$B(x,\epsilon)=\{y\in X\mid d(x,y)<\epsilon\}$$$$\overline{B}(x,\epsilon)=\{y\in X\mid d(x,y)\leq\epsilon\}$$Then \begin{itemize}
    \item $\{B(x,\epsilon)\mid \epsilon>0\}$ is a neighborhood basis of x
    \item $\{\overline{B}(x,\frac{1}{n})\mid n\in \mathbb{N}_{\geq 1}\}$ is a neighborhood basis of x
    \item $\{B(x,\epsilon)\mid \epsilon>0\}$ is a neighborhood basis of x
    \item $\{\overline{B}(x,\frac{1}{n})\mid n\in \mathbb{N}_{\geq 1}\}$ is a neighborhood basis of x
    \end{itemize}
\end{itemize}
\subsection{Example}$\mathcal{V}_x\cap\mathcal{J}$ is a neighborhood basis of x
\section{Def}
$V\in \wp(X)$ is called a neighborhood of $x$ if $\exists U|in \mathcal{J}$ such that $ x\in U\subseteq V$
\section{Remark}
Let $(X,\mathcal{J})$ be a topological space, $x\in X$ and $\mathscr{B}_x$ a neighborhood basis os x. Suppose that $\mathscr{B}$ is countable . We choose a surjective mapping $(B_n)_{n\in \mathbb{N} }$ from $\mathbb{N} $ to$\mathscr{B}_x$. For any $n\in \mathbb{N} $, let $A_n=B_0\cap B_1\cap...\cap B_n\in\mathcal{V}_x$ The sequence $(A_n)_{n\in \mathbb{N} }$ is decreasing adn $\{A_n\mid n\in\mathbb{N} \}$ is a neighborhood basis of x.
\section{Extra Episode}$\wp(\mathbb{N} )$is NOT countable

Suppose that $f:\wp(\mathbb{N} )\rightarrow\mathbb{N} $ injective. Then $\exists g:\mathbb{N} \rightarrow\wp(\mathbb{N} )$ surjective. Taking $A=\{n\in \mathbb{N} \mid n\not\in g(n)\}$. Since $g$ is surjective, $\exists a\in \mathbb{N} $ such that $A=g(a)$. \begin{itemize}
    \item[If] $a\in A$, then $a\in g(a)$, hence $a\not\in A$
    \item[If] $a\not\in A$, then $a\in g(a)=A$
\end{itemize}
Contradiction
\section{Prop.}
Let Y and R be sets, $g:Y\rightarrow E$ be a mapping ,\begin{itemize}
    \item If $\mathcal{F}$ is a filter of Y, then $$G_*(\mathcal{F}):=\{A\in\wp(E)\mid g^{-1}(A)\in\mathcal{F}\}$$ is a filter on E
    \item If $\mathscr{B}$ is a filter basis of Y, then $$g(\mathscr{B}):=\{g(B)\mid B\in \mathscr{B}\}$$is a filter of E, adn $\mathcal{F}(g(\mathscr{B}))=g_*(\mathcal{F}(\mathscr{B}))$ 
\end{itemize}
\subsection{Proof}
\begin{itemize}
    \item [(1)]$E\in g_x(\mathcal{F})$ since $g^{-1}(E)=Y$\\$\varnothing \not\in g_x(\mathcal{F})$ since $g^{-1}(\varnothing)=\varnothing$
    \begin{itemize}
    \item [If] $A\in g_x(\mathcal{F})$ and $A'\supseteq A$, then $g^{-1}(A')\supseteq g^{-1}(A)\in\mathcal{J}$, so $g^{-1}(A')\in\mathcal{J},$ Hence $A'\in g_x(\mathcal{F})$
    \item [If] $A_1,A_2\in g_x(\mathcal{F})$. Then $g^{-1}(A_1)\in \mathcal{F},g^{-1}(A_2)\in \mathcal{F}$ Hence $g^{-1}(A_1\cap A_2)=g^{-1}(A_1)\cap g^{-1}(A_2)\in \mathcal{F}$. So $A_1\cap A_2\in g_x(\mathcal{F})$.
    \end{itemize}
    \item [(2)]Since $g$ is a mapping , and $\varnothing \not\in \mathscr{B}$, we get $\varnothing \not\in g(\mathscr{B})$,since $\mathscr{B}\not=\varnothing, g(\mathscr{B})\not=\varnothing$. 
    
\end{itemize}Let $B_1,B_2\in\mathscr{B}$, there exists $C\in \mathscr{B}$ such that $C\subseteq B_1\cap B_2$. Hence $g(C)\subseteq g(B_1)\cap g(B_2)$, namely $g(\mathscr{B})$ is a filter basis.
\chapter{Limit point and accumulation point}
We fix a topological space $(X,\mathcal{J})$
\section{Def}
Let $\mathcal{F}$ be a filter of X and $x\in X$\begin{itemize}
    \item If $\mathcal{V}_x\subseteq \mathcal{F}$ then we say that $x$ is an limit point of $\mathcal{F}$
    \item If $\forall (A,V)\in\mathcal{F}\times \mathcal{V}_x,A\cap V\not=\varnothing$, we say that $x$ is an accumulation point of $\mathcal{F}$
\end{itemize} 
So any limit point of $\mathcal{F}$ is necessarily a accumulation point of $mathcal{F}$ 
\section{Prop}
Let $\mathscr{B}$ be a filter basis of X, $x\in X,\mathscr{B}_x$ a neighborhood basis of x. Then $x$ is an accumulation point of $\mathcal{F}(\mathscr{B})$ iff $\forall (B,U)\in \mathscr{B}\times \mathscr{B}_x, B\cap U\not =\varnothing$
\subsection{Proof}
\subsubsection{Necessity}
Since $\mathscr{B}\subseteq \mathcal{F}(\mathscr{B}),\mathscr{B}\subseteq \mathcal{V}_x$, the necessity is true.
\subsubsection{Sufficiency}
Let $(A,V)\in \mathcal{F}(\mathscr{B})\times\mathcal{V}_x$. There exist $B\in \mathscr{B},U\in \mathscr{B}_x$, such that $B\subseteq A, U\subseteq V$. Hence $\varnothing \not=B\cap U\subseteq A\cap V$
\section{Def}
Let $Y\subseteq X, Y\not=\varnothing$. W call accumulation point of Y any accumulation point of the principal filter $\mathcal{F}=\{A\subseteq X\mid Y\subseteq A\}$. We denote by $\overline{Y}=\{\text{accumulation points of }Y\}.$ Note that $x\in\overline{Y}$ iff $\forall U\in \mathscr{B}_x,Y\cap U\not=\varnothing$\\By convention $\overline{\varnothing}=\varnothing$
\section{Prop}
Let $Y\subseteq X$. Then $\overline{Y}$ is the smallest closed subset of X containing Y.
\subsection{Proof}
$\forall x\in X\backslash \overline{Y}$, then there exists $U_x=\mathcal{V}\cap \mathcal{J}$, such that $Y\cap U_x=\varnothing$. Moreover , $\forall y\in U_x, U_x\in \mathcal{V}_y\cap\mathcal{J}$. This shows that $\forall y\in U_x,y\not\in \overline{Y}$. Therefore $X\backslash\overline{Y}=\bigcap\limits_{x\in X\backslash\overline{Y}}U_x\in \mathcal{J}$

Let $Z\subseteq X$ be a closed subset that contain Y. Suppose that $\exists y\in \overline{Y}\backslash Z.$ Then $U=X\backslash Z\in \mathcal{V}_y\cap\mathcal{J}$ and $U\cap Y\subseteq U\cap Z=\varnothing$. So $y\not\in \overline{ Y}$ contradiction. Hence $\overline{Y}\subseteq Z.$
\chapter{Limit of mappings}
\section{Def}
Let $(E,\mathcal{J}_E)$ be a topological space . $f: Y\rightarrow E$ a mapping , and $\mathcal{F}$ eb a filter of Y. If $a\in E$ is a limit point of $F_*(\mathcal{F})$ namely , $\forall \text{neighborhood} V \text{of} a,f^{-1}(V)\in \mathcal{F}$, then we say that $a$ is a limit of the filter $\mathcal{F}$ by $f$
\section{Remark}
Let $\mathscr{B}_a$ be a neighborhood basis of $a$. Then $\mathcal{V}_a\subseteq f_x(\mathcal{F})$, iff $\mathscr{B}\subseteq f_* (\mathcal{F})$ Therefore , $a$ is a limit of $\mathcal{F}$ by $f$ iff $\forall V\in\mathscr{B}_a,f^{-1}(V)\in\mathcal{F}$
\subsection{Example}
Let $(E,\mathcal{J}_E)$ be a topological space. $I\subseteq\mathbb{N} $ be an infinite subset, $x=(x_n)_{n\in I}\in E^I$. If the Fréchet filter $\mathcal{F}_{Fr}(I)$ has a limit $a\in E$ by the mapping $x:I\rightarrow E$, we say that $(x_n)_{n\in I}$ converges to a ,denote as $$a=\lim\limits_{n\in I, n\rightarrow +\infty}x_n$$
\section{Remark}
$a=\lim\limits_{n\in I, n\rightarrow +\infty}x_n$ iff, $\forall U\in \mathscr{B}_a$(where $\mathscr{B}_a$ is a neighborhood basis of $a$), $\exists N\in \mathbb{N} $ such that $x_n\in U$ for any $n\in I_{\geq N}$

Suppose that $\mathcal{J}_E$ is induced by a metric $d$.$\{B(a,\epsilon)\mid\epsilon>0\},\{\overline{B}(a,\epsilon)\mid\epsilon>0\}\{B(a,\frac{1}{n})\mid n\in \mathbb{N} _{\geq 1}\}\{\overline{B}(a,\frac{1}{n})\mid n\in \mathbb{N} _{\geq 1}\}$
are all neighborhood basis of $a$ .There fore , the following are equivalent
\begin{itemize}
    \item $a=\lim\limits_{n\in i,n\rightarrow+\infty}x_n$
    \item $\forall \epsilon>0, \exists N\in \mathbb{N} ,\forall n\in I_{\geq N},d(x_n,a)<\epsilon$
    \item $\forall \epsilon>0, \exists N\in \mathbb{N} ,\forall n\in I_{\geq N},d(x_n,a)\leq\epsilon$
    \item $\forall k\in \mathbb{N} _{\geq 1},\exists N\in \mathbb{N} ,\forall n\in I_{\geq N},d(x_n,a)<\frac{1}{n}$
    \item $\forall k\in \mathbb{N} _{\geq 1},\exists N\in \mathbb{N} ,\forall n\in I_{\geq N},d(x_n,a)\leq\frac{1}{n}$
\end{itemize}
($x^{-1}(B(a,\epsilon))=\{n\in I\mid d(x_n,a)<\epsilon\}$? unknown position )
\end{document}
