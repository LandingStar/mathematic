\documentclass{book}
\usepackage{graphicx} % Required for inserting images
\usepackage{mathrsfs}
\usepackage{amssymb}
\usepackage{amsmath}
\usepackage{indentfirst}
\usepackage{color}
\usepackage{hyperref}  
\hypersetup{hidelinks,
	colorlinks=true,
	allcolors=black,
	pdfstartview=Fit,
	breaklinks=true
}

\begin{document}
\tableofcontents
\part{Set}
\chapter{Ring}
\section{morphism}
    \subsection*{Def}
    \indent Let A and B be unitary rings .We call morphism of unitary rings from A to B .only mapping $A\rightarrow B$is a morphism of group from (A,+) to (B,+),and a morphism of monoid from $(A,\cdot)\ to\ (B,\cdot)$
    \subsection*{Properties}
    \begin{itemize}
        \item Let R be a unitary ting. There is a unique morphism from $\mathbb{Z}\ to\ R$
        \item 
    \end{itemize}
\subsection*{algebra}

we call k-algebra any pair(R,f),when R is a unitary ring ,and $f:k\rightarrow R$ is a morphism of unitary rings such that $\forall (b,x)\in k\times R,f(b)x=xf(b)$

Example:    For any unitary ring R,the unique morphism of unitary rings $\mathbb{Z}\rightarrow R$ define a structure of $\mathbb{Z}-algebra$ on R (extra: $\mathbb{Z}$ is commutative despite R isn't guaranteed)

Notation: Let k be a commutative unitary ring ,(A,f) be a k-algebra. If there is no ambiguity on f,for any$ (\lambda,a )\in k\times A$,we denote $f(\lambda)a\ as\ \lambda a$
\subsection*{Formal power series}
reminder: $n\in\mathbb{N}$ is possible infinite ,so $\sum\limits_{n\in \mathbb{N}}$ couldn't be executed directly.
Def:

(extended polynomial actually)
Let k be a commutative unitary ring.
Def : Let T be a formal symbol.We denote $k^\mathbb{N}$ as $k\mathbb{[}T\mathbf{]}$ If $(a_n)_{n\in\mathbb{N}}$ is an element of $k^\mathbb{N}$,when we denote $k^\mathbb{N}$ as k[T] this element is denote as $\sum_{n\in\mathbb{N}}a_nT^n$ Such element is called a formal power series over k and $a_n$ is called the Coefficient of $T^n$ of this formal power series
Notation: \begin{itemize}
    \item omit terms with coefficient O
    \item write T' as T
    \item omit Coefficient those are 1;
    \item omit $T^0$
\end{itemize}
Example $1T^0+2T^1+1T^2+0T^3+...+0T^n+...$ is written as $1+2T+T^2$

Def Remind that k[T]=$\{\sum_{n\in\mathbb{N}}a_nT^n\mid (a_n)_{n\in\mathbb{N}}\in k^\mathbb{N} \}$,define two composition laws on k[T]\\
$\forall F(T)=a_0+a_+1T+...\quad G(T)=b_0+...$\\
let $F+G=(a_0+b_0)+...$\\
$FG=\sum\limits_{n\in\mathbb{N}}\sum\limits_{i+j=n}(a_ib_j)T^n$\\
Properties:\begin{itemize}
\item $(k[T],+,\cdot)$form a commutative unitary ring.
\item $k\rightarrow k[T]\quad \lambda\mapsto\lambda T$ is a morphism 
\item $(FG)H=\left(\sum_{n\in\mathbb{N}}^{}\sum\limits_{i+j=n}(a_ib_j)T^n\right)(\sum\limits_{n\in\mathbb{N}}c_nT^n)=\sum\limits_{n\in\mathbb{N}}\left(\sum\limits_{p,q,l=n}a_pb_qc_l\right)T^n$\\is a trick applied on integral
\end{itemize}
Derivative:

let $F(T)\in k[T]$\\
\indent We denote by $F'(T)\ or\ \mathcal{D}(F(T))$ the formal power series\\
\indent$\mathcal{D}(F)=\sum\limits_{n\in\mathbb{N}}(n+1)a_{n+1}T^n$\\
Properties:\begin{itemize}
    \item $\mathcal{D}(k[T],+)\rightarrow(k[T],+)$ is a morphism of groups
    \item $\mathcal{D}(FG)=F'G+FG'$
\end{itemize}
exp

We denote $exp(T)\in k[T]$ as $\sum\limits_{n\in\mathbb{N}}{1\over n!}T^n$,which fulfil the differential equation $\mathcal{D}(exp(T))=exp(T)$(interesting)\\
Cauchy sequence: $(F_i(T))_{i\in\mathbb{N}}$ be a sequence of elements in k[T],and $F(T)\in k[T]$We say that $(F_i(T))_{i\in \mathbb{N}}$ is a Cauchy sequence if $\forall l \in \mathbb{N}$,there exists $N(l)\in \mathbb{N}$ such that $\forall(i,j)\in\mathbb{N}^2_{\geq N(l)},ord(F_i(T)-F_j(T))\geq l$
\part{Sequences}
\chapter{Supremum and infimum}
Def:

Let $(X,\le)$ be a partially ordered set A and Y be subsets of X,such that $A\subseteq Y$
\begin{itemize}
    \item If the set $\{y\in Y\mid \forall a \in A,a\leq Y\}$has a least element then we say that A has a Supremum in Y with respect to $\leq$ denoted by $sup_{(y,\leq)}A$ this least element and called it the Supremum of A in Y(this respect to $\leq$)
    \item If the set $\{y\in Y\mid\forall a\in A,y\leq a\}$ has a greatest element, we say that A has n infimum in Y with respect to $\leq$ .We denote by $inf_{(y,\leq)}A$ this greatest element and call it the infimum of A in Y 
    \item Observation: $inf_{(Y,\leq)}A=sup_{(Y,\geq)}A$
\end{itemize}
Notation:

Let $(X,\leq)$ be a partially ordered set,I be a set.
\begin{itemize}
    \item If f is a function from I to X sup f denotes the supremum of f(I) is X.$\inf f $takes the same
    \item If $(x_i)_{i\in I} is \ a\ family\ of\ element\ in\ X,then\ \sup \limits_{i\in I}x_i\ denotes\ \sup\{x_i\mid i\in I\} (in X)$
\end{itemize}

If moreover $\mathbb{P}(\cdot)$ denotes a statement depending on a parameter in I then $\sup\limits_{i\in I,\mathbb{P}(i)}x_i$ denotes $\sup\{x_i\mid i\in I,\mathbb{P}(i)\ holds\}$\\
Example:

Let $A={x\in R\mid 0\leq x<1}\subseteq \mathbb{R}$ We equip $\mathbb{R}$ with the usual order relation.$$\{y\in \mathbb{R}\mid \forall x\in A,x\leq y\}=\{y\in\mathbb{R}\mid y\geq 1\}$$
So $\sup A=1$
$$\{y\in \mathbb{R}\mid \forall x\in A,y\leq x\}=\{y\in\mathbb{R}\mid y\geq 0\}$$
Hence $\inf A=0$\\
Example: 
For $n\in \mathbb{N}$,let $x_n=(-1)^n\in R$ $$\sup\limits_{n\in \mathbb{N}}\inf\limits_{k\in \mathbb{N},k\geq n}x_k=-1$$
Proposition:

Let $(X,\leq)$ be a partially ordered set,A,Y,Z be subset of X,such that $A\subseteq Z\subseteq Y$
\begin{itemize}
    \item If max A exists,then is is also equal to $\sup_{(y,\leq)}A$
    \item If $\sup_{(y,\leq)}A$ exists and belongs to Z, then it is equal to $\sup A$
\end{itemize}

$\inf$ takes the same
Prop.

Let $X,\leq$ be a partially ordered set ,A,B,Y be subsets of X such that $A\subseteq B\subseteq Y$
\begin{itemize}
    \item If $\sup_{(y,\leq)}A$ and $\sup_{(y,\leq)}B$ exists,then $\sup_{(y,\leq)}A \leq \sup_{(y,\leq)}B$
    \item If $\inf_{(y,\leq)}A$ and $\inf_{(y,\leq)}B$ exists,then $\inf_{(y,\leq)}A \geq \inf_{(y,\leq)}B$
\end{itemize}
Prop.

Let $(X,\leq)$be a partially ordered set ,I be a set and $f,g:I\rightarrow X$ be mappings such that $\forall t\in I,f(t)\leq g(t)$
\begin{itemize}
    \item If $\inf f$ and $\inf g$ exists,then $\inf f\leq \inf g$ 
    \item If $\sup f$ and $\sup g$ exists,then $\sup f\leq \sup g$ 
\end{itemize}
\chapter{Interval}
We fix a totally ordered set $(X,\leq)$\\
Notation:

If $(a,b)\in X\times X $ such that $a\leq b $,[a,b] denotes $\{x\in X\mid a\leq x\leq b\}$\\
Def:

Let $I\subseteq X $. If $\forall(x,y)\in I\times I$ with $x\leq y$,one has $[x,y]\subseteq I$ then we say that I is a interval in X\\
Example:

Let $(a,b)\in X\times X$, such that $a\leq b$ Then the following sets are intervals
\begin{itemize}
    \item $]a,b[:=\{x\in X\mid a,x,b\}$
    \item $[a,b[:=\{x\in X\mid a,x,b\}$
    \item $]a,b]:=\{x\in X\mid a,x,b\}$
\end{itemize}
Prop. 

Let $\Lambda$ be a non-empty set and $(I_\lambda)_{\lambda\in \Lambda}$ be a family of intervals in X.
\begin{itemize}
    \item $\bigcap\limits_{\lambda\in\Lambda}I_\lambda$ is a interval in X
    \item If $\bigcap\limits_{\lambda\in\Lambda}I_\lambda\not=\varnothing $,$\bigcup\limits_{\lambda\in\Lambda}I_\lambda$ is a interval in X
\end{itemize}
We check that $[a,b]\subseteq I_\lambda\cup I_]\mu$
\begin{itemize}
    \item If $b\leq x$\quad $[a,b]\subseteq[a,x]\subseteq I_\lambda$ because $\{a,x\}\subseteq I_\lambda$
    \item If $x\leq a$\quad $[a,b]\subseteq[x,b]\subseteq I_\mu $ because $\{b,x\}\subseteq I_\mu$
    \item If a $<$ x $<$ b then $[a,b]=[a,x]\cup[x,b]\subseteq I_\lambda\cup I_\mu$
\end{itemize}
Def:

Let $(X,\leq)$ be a totally ordered set .I be a non-empty interval of X. If $\sup I$ exists in X, we call $\sup I$ the right endpoint; inf takes the similar way.\\
Prop.

Let I be an interval in X.
\begin{itemize}
    \item Suppose that $b=\sup I$exists. $\forall x\in I,[x,b[\subseteq I$
    \item Suppose that $a=\inf I$exists. $\forall x\in I,]a,x]\subseteq I$
\end{itemize}
Prop.

Let I be an interval in X. Suppose that I has supremum b and an infimum a in X.Then I is equal to one of the following sets $[a,b]\ \ [a,b[\ \ ]a,b]\ \ ]a,b[$\\
Def 

let $(X,\leq)$ be a totally ordered set .If $\forall (x,z)\in X\times X$,such that $x<z\quad \exists y\in X$ such that $x<y<z$,than we say that $(X,\leq)$ is thick\\
Prop. 

Let $(X,\leq)$ be a thick totally ordered set. $(a,b)\in X\times X,a<b$ If I is one of the following intervals $[a,b];[a,b[;]a,b];]a,b[$ Then $\inf I=a\quad \sup I=b$ (for it's thick empty set is impossible)\\
Proof:

Since X is thick,there exists $x_0\in]a,b[$ By definition,b is an upper bound of I. If b is not the supremum of I,there exists an upper bound M of I such that M<b. Since X is thick ,there is $M'\in X$ such that $x_0\leq M,M'<b$ Since $[x,b[\subseteq]a,b[\in I$ Hence M and M' belong to I,which conflicts with the uniqueness of supremum.
\chapter{Enhanced real line}
Def:

Let $+\infty$ and $-infty$ be two symbols that are different and don not belong to $\mathbb{R}$ We extend the usual total order $\leq\ on\ \mathbb{R}\ to\ \mathbb{R}\cup\{-\infty,+\infty\}$ such that $$\forall x\in \mathbb{R} ,-\infty<x<+\infty$$ 
\indent Thus $\mathbb{R} \cup\{-\infty,+\infty\}$ become a totally ordered set, and $\mathbb{R} =]-\infty,+\infty[$  Obviously,this is a thick totally ordered set.\\
We define:
\begin{itemize}
    \item $\forall x\in ]-\infty,+\infty]\quad x+(+\infty):=+\infty \quad (+\infty)+x:=+\infty$
    \item $\forall x\in [-\infty,+\infty[\quad x+(-\infty):=-\infty\quad (-\infty)+x=-\infty$
    \item $\forall x\in ]0,+\infty]\quad x(+\infty)=(+\infty)x=+\infty\quad x(-\infty)=(-\infty)x=-\infty$
    \item $\forall x\in [-\infty,0[\quad x(+\infty)=(+\infty)x=-\infty\quad x(-\infty)=(-\infty)x=+\infty$
    \item $-(+\infty)=-\infty\quad -(-\infty)=+\infty\quad (\infty)^{-1}=0$
    \item $(+\infty)+(-\infty)\quad (-\infty)+(+\infty)\quad (+\infty)0\quad 0(+\infty)\quad(-\infty)0\quad 0(-\infty)$ \\\textcolor{red}{ARE NOT DEFINED}
\end{itemize}
Def 

Let $(X,\leq)$ be a partially ordered set. If for any subset A of X,A has a supremum and an infimum in X, then we say the X is order complete\\
Example

Let $\Omega$ be a set $(\mathscr{P}(\Omega),\subseteq)$is order complete If $\mathscr{F}$is a subset of $\mathscr{P}(\Omega),\sup\mathscr{F}=\bigcup\limits_{A\in \mathscr{F}}A$

Interesting tip: $\inf \varnothing =\Omega\quad \sup \varnothing=\varnothing$\\
$\mathcal{AXION}:$\\\indent$(\mathbb{R}\cup\{-\infty,+\infty\},\leq)$ is order complete \\\indent In $\mathbb{R}\cup\{-\infty,+\infty\}\quad \sup\varnothing=-\infty\quad\inf\varnothing=+\infty$\\
Notation:
\begin{itemize}
    \item For any $A\subseteq \mathbb{R} \cup{-\infty,+\infty}\ and\ c\in \mathbb{R} $ We denote by $A+c$ the set $\{a+c\mid a\in A\}$
    \item If $\lambda\in\mathbb{R} \backslash\{0\},\lambda A$ denotes $\{\lambda a\mid a\in A\}$
    \item -A denotes (-1)A
\end{itemize}
Prop.

For any $A\subseteq\mathbb{R} \cup\{-\infty,+\infty\},\sup(-A)=-\inf A,\inf (-A)+-\sup A$
Def 

We denote by $(R,\leq)$ a field $\mathbb{R} $ equipped with a total order $\leq$, which satisfies the following condition:
\begin{itemize}
    \item $\forall (a,b)\in \mathbb{R} \times \mathbb{R} $ such that $a<b$ ,one has $\forall c\in \mathbb{R} ,a+c<b+c$
    \item $\forall(a,b)\in\mathbb{R}_{>0}\times\mathbb{R}_{>0},ab>0$
    \item $\forall A\subseteq \mathbb{R} $,if A hsa an upper bound in$\mathbb{R}$ ,then it has a supremum in$\mathbb{R} $
\end{itemize}
Prop.

Let $A\subseteq[-\infty,+\infty]$
\begin{itemize}
    \item $\forall c\in \mathbb{R} \quad \sup(A+c)=(\sup A)+c$
    \item $\forall \lambda\in \mathbb{R}_{\geq0}\quad \sup(\lambda A)=\lambda\sup(A)$
    \item $\forall\lambda\in \mathbb{R} _{\leq0}\quad sup(\lambda A)=\lambda\inf(A)$
\end{itemize}

\indent $\inf$ takes the same\\
Theorem:

Let I and J be non-empty sets\\
\indent$f:I\rightarrow[-\infty,+\infty],g:J\rightarrow[-\infty,+\infty]$\\
\indent$a=\sup\limits_{x\in I}f(x)\quad b=\sup\limits_{y\in J}g(y)\quad c=\sup\limits_{(x,y)\in I\times J,\{f(x),g(y)\}\not=\{+\infty,-\infty\}}(f(x)+g(y))$ \\
\indent If $\{a,b\}\not=\{+\infty,-\infty\}$then $c=a+b$\\
\indent $\inf$ takes the same if $(-\infty)+(+\infty)$ doesn't happen\\
Corollary:

Let I be a non-empty set,$f:I\rightarrow[-\infty,+\infty],g:J\rightarrow[-\infty,+\infty]$\\
Then $\sup\limits_{x\in I,\{f(x),g(y)\}\not=\{+\infty,-\infty\}}(f(x)+g(x))\leq(\sup\limits_{x\in I}f(x))(\sup\limits_{x\in I}g(x))$\\
\indent $\inf$ takes the similar($\leq\rightarrow\geq$) (provided when the sum are defined)
\chapter{Vector space}
In this section:\\
\indent K denotes a unitary ring.\\
\indent Let 0 be zero element of K\\
\indent 1 be the unity of K
\section{K-module}
\subsection{Def}

Let $(V,+)$ be a commutative group.We call left/right K-module structure:\\
\indent any mapping $\Phi$:$K\times V\rightarrow V$
\begin{itemize}
    \item $\forall(a,b)\in K\times K,\forall x\in V\quad \Phi(ab,x)=\Phi(a,\Phi(b,x))/\Phi(b,\Phi(a,x))$
    \item $\forall (a,b)\in K\times K,\forall x\in V,\Phi(a+b,x)=\Phi(a,x)+\Phi(b,x)$
    \item $\forall a\in K,\forall(x,y)\in V\times V,\Phi(a,x+y)=\Phi(a,x)+\Phi(a,y)$
    \item $\forall x\in V,\Phi(1,x)=x$
\end{itemize}
A commutative group (V,+) equipped with a left/right K-module structure is called a left/right K-module.
\subsection{Remark}\quad Let $K^{op}$ be the set K equipped with the following composition laws:
\begin{itemize}
    \item $K\times K\rightarrow K$
    \item $(a,b)\mapsto a+b$
    \item $K\times K\rightarrow K$
    \item $(a,b)\mapsto ba$
\end{itemize}
Then $K^{op}$ forms a unitary ring\\
Any left $K^{op}-module$ is a right K-module\\
Any right $K^{op}-module$ is a left K-module\\
$(K^{op})^{op}=K$
\subsection{Notation}

When we talk about a left/right K-module $(V,+)$,we often write its left K-module structure as $K\times V\rightarrow V\quad (a,x)\mapsto ax$\\
\indent The axioms become:\begin{align*}
    &\forall(a,b)\in K\times K,\forall x\in V\quad (ab)x=a(bx)/b(ax)\\
    &\forall(a,b)\in K\times K,\forall x\in V\quad (a+b)x=ax+bx\\
    &\forall a\in K,\forall(x,y)\in V\times V\quad a(x+y)=ax+ay\\
    &\forall x\in V\quad 1x=x
\end{align*}
\subsection{K-vector space}

If K is commutative,then $K^{op}=K$,so left K-module and right K-module structure are the same .We simply call them K-module structure. A commutative group equipped with a K-module structure is called a K-module.If K is a field,a K-module is also called a K-vector space

Let $\Phi:K\times V\rightarrow V$ be a left or right K-module structure$$\forall x\in V,\Phi(\cdot,x):K\rightarrow V\quad(a\in K)\mapsto\Phi(a,x)$$\indent is a morphism of addition groups.Hence $\Phi(0,x)=0,\Phi(-a,x)=-\Phi(a,x)$\\
$\forall a\in K,\Phi(a,\cdot):V\rightarrow V$ is a morphism of groups.Hence $\Phi(a,0)=0,\Phi(a,-x)=-\Phi(a,x)(\cdot \ is\ a\ var)$
\subsection{Association:}
$\forall x\in K$
\begin{align*}
    (f(f+g)+h)(x)=(f+g)(x)+h(x)=f(x)+g(x)+h(x)\\
    (f+(g+h))(x)=f(x)+((g+h)(x))=f(x)+g(x)+h(x)
\end{align*} 

Let $0:I\rightarrow K:x\mapsto0 \quad \forall f\in K^I\quad f+0=f$\\
Let $-f: f+(-f)=0$\\
The mapping $K\times K^I\rightarrow K^I:(a,f)\mapsto af\quad (af)(x)=af(x)$ is a left K-module structure\\
The mapping $K\times K^I\rightarrow K^I:(a\in I)\mapsto ((x\in I)\mapsto f(x)a)\quad (af)(x)=af(x)$ is a right K-module structure
\subsection{Remark:}

We can also write an element $\mu\ of\ K^I$ is the form of a family $(\mu_i)_{i\in I}$ of elements in K ($\mu_i$is the image of $i\in I\ by\ \mu$)\\
Then \begin{align*}
    &(\mu_i)_{i\in I}+(\nu_i)_{i\in I}:=(\mu_i+\nu_i)_{i\in I}\\
    &a(\mu_i)_{i\in I}:=(a\mu_i)_{i\in I}\\
    &(\mu_i)_{i\in I}a=(\mu_ia)_{i\in I}
\end{align*}
\section{sub K-module}
\subsection{Def}

Let V be a left/right K-module.If W is a subgroup of V. Such that $\forall a\in K,\forall x\in W\quad ax/xa\in W$, then we say that W is left/right sub-K-module of V.
\subsection{Example}

Let I be a set .Let $K^{\bigoplus I}$ be the subset of $K^I$ composed of mappings $f:I\rightarrow K$ such that $I_f=\{x\in I\mid f(x)\not=0\}$ is finite. It is a left and right sub-K-module of $K^I$\\
\indent In fact,$\forall (f,g)\in K^{\bigoplus I}\times K^I\quad I_{f-g}={x\in I\mid f(x)-g(x)\not=0}\subseteq I_f\cup I_g$\\
\indent Hence $f-g\in K^{\bigoplus I}$ So $K^{\bigoplus I}$ is a subgroup of $K^I$\\
\indent $\forall a\in K,\forall f\in K^{\bigoplus I}\quad I_{af}\subseteq I_f,I_{(x\mapsto f(x)a)}\subseteq I_f$
\section{morphism of K-modules}
\subsection{Def}

Let V and W be left K-module, A morphism of groups $\phi:V\rightarrow W$ is called a morphism of left K-modules if $\forall(a,x)\in K\times V,\phi(ax)=a\phi(x)$
\subsection{K-linear mapping}
\indent If K is commutative, a morphism of K-modules is also called a K-linear mapping. We denote by $\hom_{K-Mod}(V,W)$ the set of all morphism of left-K-module from V to W.This is a subgroup of $W^V$
\subsection{Theorem}

Let V be a left K-module. Let I be a set.\\
The mapping $\hom_{K-Mod}(K^{\bigoplus I},V)\rightarrow V^I:\ \phi\rightarrow (\phi(e_i))_{i\in I}$ is a bijection where $e_i:I\rightarrow K:j\mapsto\left\{\begin{aligned}
    1\quad j=i\\
    0\quad j\not=i
\end{aligned} \right.$
\subsection{Remark:column}

In the case where $I={1,2,3,...,n}\ V^I$ is denoted as $V^n,K^I$ is denoted as $K^n$\\
For any $(x_1,...,x_n)\in V^n$,by the theorem, there exists a unique morphism of left K-modules $\phi:K^n\rightarrow V$ such that $\forall i\in{1,...,n}\phi(e_i)=x_i$\\
We write this $\phi$ as a column$\begin{pmatrix}
    x_1\\
    ...\\
    x_n
\end{pmatrix}$It sends $(a_1,...,a_n)\in K^n$ to $a_1x_1+...+a_nx_n$
\section{kernel}
\subsection{Prop}

Let G and H be groups and $f:G\rightarrow H$ be a morphism of groups\begin{itemize}
    \item $I_m(f)\subseteq H $ is a subgroup of H
    \item $\ker(f)=\{x\in G\mid f(x)=e_H\}$
    \item $f$ is injection iff $\ker(f)=\{e_G\}$
\end{itemize}
\subsection{Def}

$\ker(f)$ is called the kernel of $f$
\subsection{Theorem}
$f$ is injection iff $\ker(f)=\{e_G\}$
\subsection{Proof}
Let $e_G$ and $e_H$ be neutral element of G and H respectively
\begin{itemize}
    \item [(1)]Let x and y be element of G\\$f(x)f(y)^{-1}=f(x)f(y)^{-1}=f(xy^{-1})\in Im(f).$ So $Im(f)$ is a subgroup of H
    \item [(2)]Let x and y be element of $\ker(f)$ One has $f(xy^{-1})=f(x)f(y)^{-1}=e_H\quad e_H^{-1}=e_H.$ So $xy^{-1}\in\ker(f)$ So $\ker(f)$ is a subgroup of G
    \item [(3)]Suppose that f is injection.\\ Since $f(E_G)=e_H$ one has $\ker(f)=f^{-1}(\{e_H\})=\{e_G\}$ Suppose that $\ker(f)=\{e_G\}$ If $f(x)=f(y)$then $f(xy^{-1})=f(x)f(y)^{-1}=e_H$\\Hence $xy^{-1}=e_G\ \Rightarrow\ x=y$
\end{itemize}
\subsection{Def}

Let (V,+) be a commutative group, I be a set. We define a composition law + on $V^I$ as follows$$(x_i)_{i\in I}+(y_i)_{i\in I}\ :=(x_i+y_i)_{i\in I}$$
Then $V^I$ forms a commutative group
\subsection{Remark}

Let E and F be left K-modules\\$\hom_{K=Mod}(E,F):=\{\text{morphisms of left K-modules from E to F}\}\subseteq F^E$ is a subgroup of $F^E$\\
In fact f and g are elements of $\hom_{K-Mod}(E,F)$, then $f-g$ is also a morphism of left K-module\\$(f-g)(x+y)=f(x+y)-g(x+y)=(f(x)+f(y))-(g(x)+g(y))=(f(x)-g(x))+(f(y)-g(y))=(f-g)(x)+(f-y)(x)$
\subsection{Theorem}

Let V be a left K-module,I be a set The mapping $\hom_{K-Mod}(K^{\bigoplus I},V)\rightarrow V^I\ :\ \phi\mapsto(\phi(e_i))_i\in I$ is an isomorphism of groups, where $e_i:I\rightarrow K:j\mapsto\left\{\begin{aligned}
    1\quad j=i\\
    0\quad j\not=i
\end{aligned} \right.$
\subsection{Proof:}
One has $(\phi+\psi)(e_i)=\phi(e_I)+\psi(e_i)$\\
$\forall(\phi,psi)\in \hom_{K-Mod}(K^{\bigoplus I},V)^2$\\
Hence $\Psi(\phi,\psi)=(\phi(e_i)+\psi(e_i))_{i\in I}=\Psi(\phi)+\Psi(\psi)$\\
So $\Psi$ is a morphism of groups
\begin{itemize}
    \item [injectivity] Let $\phi\in \hom_{K-Mod}(K^{\bigoplus I},V)$ Such that $\forall i\in I(\forall \phi\in\ker(\Psi))\quad\phi(e_i)=0$\\Let $a =(a_i)_{i\in I}\in K^{\bigoplus I}$ One has $a=\sum\limits_{i\in I}a_ie_i$ \\ If fact,$\forall j\in I,a_j=\sum\limits_{i\in I,a_i\not=0}a_ie_i(j)$\\Thus $\phi(a)=\sum\limits_{i\in I,a_i\not=0}a-I\phi(e_i)=0$\\Hence $\phi$ is the neutral element.
    \item [surjectivity] Let $x=(x_i)_{i\in I}\in V^I$ We define $\phi_x:K^{\bigoplus I}\rightarrow V$ such that $\forall a=(a_i)_{i\in I}\in K^{\bigoplus I},\phi_x(a)=\sum\limits_{i\in I,a_i\not=0}a_ix_i$\\This is a morphism of left K-modules\\$forall i\in I,\phi_x(e_i)=1x_i=x_i$  So $\Psi(\phi_x)=x$
\end{itemize}
{\color{blue}Suppose that K' is a unitary ring,and V is also equipped with a right K'-module structure, Then $\hom_{K-Mod}(K^{\bigoplus I},V)\subseteq V^{K^{\bigoplus I}}$ is a right sub-k'-module ,and $\Psi$ in the theorem is a right K'-module isomorphism}
\chapter{Monotone mappings}
\section{Def}

Let I and X be partially ordered sets,$f:I\rightarrow X$ be a mapping.
\begin{itemize}
    \item If $\forall(a,b)\in I\times I$ such that $a<b$. One has $f(a)\leq f(b)/f(a)<f(b)$,then we say that $f$ is increasing/strictly increasing. decreasing takes similar way.
    \item If $f$ is (strictly) increasing or decreasing, we say that $f$ is (strictly) monotone.
\end{itemize}
\section{Prop.}

Let  X,Y,Z be partially ordered sets.$f:X\rightarrow Y, g:Y\rightarrow Z$ be mappings
\begin{itemize}
    \item If $f$ and $g$ have the same monotonicity, then $g\circ f$ is increasing
    \item If $f$ and $g$ have different monotonicities, then $g\circ f$ is decreasing
\end{itemize}
strict monotonicities takes the same
\section{Def}

Let $f$ be a function from a partially ordered set I to another partially ordered set .If $f\mid_{Dom(f)}\rightarrow X$ is (strictly) increasing/decreasing then we say that $f$ is (strictly) increasing/decreasing
\section{Prop.}

Let I and X be partially ordered sets. $f$ be function from I to X.
\begin{itemize}
    \item If $f$ is increasing/decreasing and $f$ is injection, then $f$ is strictly increasing/decreasing
    \item Assume that I is totally ordered and $f$ is strictly monotone, then $f$ is injection
\end{itemize}
\section{Prop}

Let A be totally ordered set, B be a partially ordered set, $f$ be an injective function from A to B\\
If $f$ is increasing/decreasing ,then so is $f^{-1}$
\section{Def}
Let X and Y be partially ordered sets. $f:X\rightarrow Y$ be a bijection. If both $f$ and $f^{-1}$ are increasing ,then we say that $f$ is an isomorphism of partially ordered sets.\\
\indent(If X is totally, then a mapping $f:X\rightarrow Y$ is an isomorphism of partially ordered sets iff $f$ is a bijection and $f$ is increasing)
\section{Prop.}
Let I be a subset of $\mathbb{N} $ which is infinite. Then there is a unique increasing bijection $\lambda_I:\mathbb{N} \rightarrow I$
\section{Proof}
\subsection{bijection}
We construct $f:\mathbb{N} \rightarrow I$ by induction as follows.\\
Let $f(0)=\min I$ Suppose that $f(0),...,f(n)$ are constructed\\
then we take $f(n+1):=\min(I\backslash\{f(0),...,f(n)\})$\\
Since $I\backslash\{f(0),...,f(n-1)\}\supseteq I\backslash\{f(0),...,f(n)\}.$Therefore $f(n)\leq f(n+1)\\
Since f(n+1)\not\in\{f(0),...,f(n)\}$,we have $f(n)<f(n+1)$\\
Hence $f$ is strictly increasing and ths is injective\\
If $f$ is not surjective,then $I\backslash Im(f)$ has a element N. \\
Let $m=\min\{n\in\mathbb{N} \mid N\leq f(n)\}$.\\
Since $N\not\in Im(f),N<f(m)$.\\
So $m\not=0$.Hence $f(m-1)<N<f(m)=\min(I\backslash\{f(0),...,f(m-1)\})$ \\
By definition,$N\in I\backslash Im(f)\subseteq I\backslash \{f(0),...,f(m-1)\}$,\\Hence $f(m)\leq N$,causing contradiction.
\subsection{uniqueness}
exercise: Prove that $Id_\mathbb{N}$ is the only isomorphism of partially ordered sets from $\mathbb{N} $ to $\mathbb{N} $
\chapter{sequence and series}
Let $I\subseteq \mathbb{N}$ be a infinite subset
\section{Def} Let X be a set.We call sequence in X parametrized by I a mapping from I to X.
\section{Remark}
If K is a unitary ring and E is a left K-module then the set of sequence $E^I$ admits a left-K-module structure. If $x=(x_n)_{n\in I}$ is a sequence in E, we define a sequence $\sum(x):=(\sum\limits_{i\in I,i\leq n}x_i)_{n\in \mathbb{N}}$,called the series associated with the sequence x.
\section{Prop}
$\sum:E^I\rightarrow E^\mathbb{N}$ is a morphism of left-K-module
\section{proof}
Let $x=(x_i)_{i\in I}$ and $y=(y_i)_{i\in I}$ be elements of $E^I$\\
$\sum\limits_{i\in I,i\leq n}(x_i+y_i)=(\sum\limits_{i\in I,i\leq n}x_i)+(\sum\limits_{i\in I,i\leq n}y_i),\lambda\sum\limits_{i\in I,i\leq n}x_i=\sum\limits_{i\in I,i\leq n}\lambda x_i$
\section{Prop}
Let I be a totally ordered set . X be a partially ordered set,$f:I\rightarrow X$ be a mapping ,$J\in I$ Assume that J does not have any upper bound in I
\begin{itemize}
    \item If $f$ is increasing ,then $f(I)$ and $f(J)$ have the same upper bounds in X
    \item If $f$ is decreasing ,then $f(I)$ and $f(J)$ have the same lower bounds in X
\end{itemize}
\section{limit}
\subsection{Def}
Let $i\subseteq\mathbb{N}$ be a infinite subset.$\forall (x_i)_{n\in I}\in [-\infty,+\infty]^I$ where $[-\infty,+\infty]$ denotes $\mathbb{R}\cup\{-\infty,+\infty\}$,we define:
$$\limsup\limits_{n\in I,n\rightarrow +\infty}x_n:=\inf\limits_{n\in I}(\sup\limits_{i\in I,i\geq n}x_i)$$
$$\liminf\limits_{n\in I,n\rightarrow +\infty}x_n:=\sup\limits_{n\in I}(\inf\limits_{i\in I,i\geq n}x_i)$$
If $\limsup\limits_{n\in I,n\rightarrow +\infty}x_n=\liminf\limits_{n\in I,n\rightarrow +\infty}x_n=l$, we then say that $(x_n)_{n\in I}$ tends to $l$ and that $l$ is the limit of $(x_n)_{n\in I}$. If in addition $(x_n)_{n\in I}\in \mathbb{R}^I$ and $l\in \mathbb{R}$,we say that $(x_n)_{n\in I}$ converges to $l$
\subsection{Remark}
If $J\subseteq\mathbb{N}$ is an infinite subset, then: $$\limsup\limits_{n\in I,n\rightarrow +\infty}=\inf\limits_{n\in J}(\sup\limits_{i\in I,i\geq n}x_i)$$
$$\liminf\limits_{n\in I,n\rightarrow +\infty}x_n=\sup\limits_{n\in J}(\inf\limits_{i\in I,i\geq n}x_i)$$
Therefore ,if we change the values of finitely many terms in $(x_i)_{i\in I}$ the limit superior and the limit inferior do not change.\\
In fact, if we take $J=\mathbb{N}\backslash\{0,...,m\}$, then $\inf\limits_{n\in J}(...)$ and $\sup\limits_{n\in J}(...)$ only depends on the values of $x_i,i\in I,i\geq m$
\subsection{Prop}
$\forall (x_n)_{n\in I}\in [-\infty,+\infty]^I,\ \liminf\limits_{n\in I,n\rightarrow+\infty}x_n\leq\limsup\limits_{n\in I,n\rightarrow+\infty}x_n$
\subsection{Prop}
Let $(x_n)_{n\in I}\in[-\infty,+\infty]^I$
\begin{align*}
    &\forall c\in\mathbb{R} & \begin{aligned}
        \limsup\limits_{n\in I,n\rightarrow+\infty}(x_n+c)=(\limsup\limits_{n\in I,n\rightarrow+\infty}x_n)+c\\
        \liminf\limits_{n\in I,n\rightarrow+\infty}(x_n+c)=(\liminf\limits_{n\in I,n\rightarrow+\infty}x_n)+c
    \end{aligned}\\
    &\forall c\in \mathbb{R}_{>0} &\begin{aligned}
        \limsup\limits_{n\in I,n\rightarrow+\infty}(\lambda x_n)=\lambda\limsup\limits_{n\in I,n\rightarrow+\infty}x_n\\\liminf\limits_{n\in I,n\rightarrow+\infty}(\lambda x_n)=\lambda\liminf\limits_{n\in I,n\rightarrow+\infty}x_n
    \end{aligned}\\
    &\forall c\in \mathbb{R}_{<0} &\begin{aligned}
        \limsup\limits_{n\in I,n\rightarrow+\infty}(\lambda x_n)=\lambda\liminf\limits_{n\in I,n\rightarrow+\infty}x_n\\\liminf\limits_{n\in I,n\rightarrow+\infty}(\lambda x_n)=\lambda\limsup\limits_{n\in I,n\rightarrow+\infty}x_n
    \end{aligned}
\end{align*}
\subsection{Prop}
Let $(x_n)_{n\in I}$ be elements in $[-\infty,+\infty]^I$. Suppose that there exists $N_0\in \mathbb{N}$ such that $\forall n\in I,n\geq N_0$,one has $x_n\leq y_n$ Then $$\limsup\limits_{n\in I,n\rightarrow+\infty}(x_n)\leq\limsup\limits_{n\in I,n\rightarrow+\infty}y_n$$ ,$$\liminf\limits_{n\in I,n\rightarrow+\infty}(x_n)\geq\liminf\limits_{n\in I,n\rightarrow+\infty}y_n$$
\subsection{Theorem}
Let $(x_n)_{n\in I},(y_n)_{n\in I},(z_n)_{n\in I}$ be elements of $[-\infty,+\infty]^I$\\
Suppose that \begin{itemize}
    \item $\exists N-N\in\mathbb{N},\forall n\in I,n\geq N_0$ one has $x_n\leq y_n\leq z_n$
    \item $(x_n)_{n\in I}$ and $(z_n)_{n\in I}$ tend to the same limit $l$
\end{itemize}
Then $(y_n)_{n\in I}$ tends to $l$
\subsection{Def}
Let I be an infinite subset of $\mathbb{N}$, and $(x_n)_{n\in I}$ be a sequence in some set X. We call subsequence of $(x_n)_{n\in I}$ a sequence of the form $(x_n)_{n\in J}$,where J is an infinite subset of I
\subsection{Prop}
Let I and J be infinite subset of $\mathbb{N}$ such that $J\subseteq I$\\
$\forall (x_n)_{n\in I}\in[-\infty,+\infty]^I$,one has $$\liminf\limits_{n\in I,n\rightarrow+\infty}(x_n)\leq\liminf\limits_{n\in I,n\rightarrow+\infty}y_n$$$$\limsup\limits_{n\in I,n\rightarrow+\infty}(x_n)\geq\limsup\limits_{n\in I,n\rightarrow+\infty}y_n$$
In particular, if $(x_n)_{n\in I}$ tends to $l\in[-\infty,+\infty]$,then $(x_n)_{n\in J}$ tends to $l$
\subsection{Prop}
$\forall n\in \mathbb{N} $,one has $$\liminf\limits_{n\in J,n\rightarrow+\infty}(x_n)\geq\liminf\limits_{n\in I,n\rightarrow+\infty}y_n$$$$\limsup\limits_{n\in J,n\rightarrow+\infty}(x_n)\leq\limsup\limits_{n\in I,n\rightarrow+\infty}y_n$$
\subsection{Theorem}
Let $I\subseteq\mathbb{N} $ be an infinite subset and $(x_N)_{n\in I}$ be a sequence in $[-\infty,+\infty]$
\begin{itemize}
    \item If the mapping $(n\in I)\mapsto x_n$ is increasing,then $(x_N)_{i\in I}$ tends to $\sup\limits_{n\in I}x_n$
    \item If the mapping $(n\in I)\mapsto x_n$ is decreasing,then $(x_N)_{i\in I}$ tends to $\inf\limits_{n\in I}x_n$
\end{itemize}
\subsection{Notation}
If a sequence $(x_N)_{n\in I}\in [-\infty,+\infty]$ tends to some $l\in[-\infty,+\infty]$ the expression $\lim\limits_{n\in I,n\rightarrow}x_n$ denotes this limit $l$
\subsection{Corollary}
Let $(x_n)_{n\in I}$ be a sequence in $\mathbb{N} _{\geq0}$ Then the series $\sum\limits_{n\in I}x_n$(the sequence $(\sum\limits_{i\in I,i\leq n})_{n\in \mathbb{N} }$) tends to an element in $\mathbb{N} _{\geq0}\cup\{+\infty\}$ It converges in $\mathbb{R} $ iff it is bounded from above (namely has an upper bound in $\mathbb{R} $)
\subsection{Notation}
If a series $\sum\limits_{n\in I}x_n$ in $[-\infty,+\infty]$ tends to some limit, we use the expression $\sum\limits_{n\in I}x_n$ to denote the limit
\subsection{Theorem: Bolzano-Weierstrass}
Let $(x_n)_{n\in I}$ be a sequence in $[-\infty,+\infty]$ There exists a subsequence of $(x_n)_{n\in I}$ that tends to $\limsup\limits_{n\in I,n\rightarrow +\infty}x_n$ There exists a subsequence of $(x_n)_{n\in I}$ that rends to $\liminf\limits_{n\in I,n\rightarrow +\infty}x_n$
\subsection{Proof}
Let $J=\{n\in I\mid\forall m\in I,\ $if$\ m\leq n\ $then$\ x_m\leq x_n\}$\\
\indent If $J$ is infinite, the sequence $(x_N)_{n\in J}$ is decreasing so it tends to $\inf\limits_{n\in J}x_n$\\
\indent $\forall n\in J$ by definition $x_n=\sup\limits_{i\in I,i\geq n}x_i$ so $\limsup\limits_{n\in I,n\rightarrow+\infty}x_n=\inf\limits_{n\in J}\sup\limits_{i\in I,i\geq n}x_i=\inf\limits_{n\in J}x_n=\lim\limits_{n\in J,n\rightarrow+\infty}x_n$\\
\indent Assume that $J$ is finite. Let $n_0\in I$ such that $\forall n\in J,n<n_0$.Denote by $l=\sup\limits_{n\in I,n\geq n_0}$\\\indent 
Let $N\in\mathbb{N}$ such that $N\geq n_0$. By definition $\sup\limits_{i\in  I,i\geq n_0}x_i\leq l$. If the strict inequality $\sup\limits_{i\in I,i\geq N}x_i<l$ holds, then $\sup\limits_{i\in I,i\geq N}x_i$ is NOT an upper bound of $\{x_n\mid n\in I,n_0\leq n<N\}$\\\indent 
So there exists $n\in I$ such that $n_0\leq n<N$ such that $x_n>\sup\limits_{i\in I,i\geq N}x_i$ We may also assume that $n $ is largest among elements of {$I\cap[n_0,N[$} that satisfies this inequality. Then $\forall m\in I$ if $m\geq n$ then $x_m\leq x_n$ Thus $n\in J$ that contradicts the maximality of $n_0$ Therefore $l=\sup\limits_{i\in I,i\geq N}x_i$, which leads to $\limsup\limits_{n\in I,n\rightarrow+\infty}x_n=l$\\\indent 
Moreover, if $m\in I,m\geq n_0$ then $m\not\in J$,so $x_m<l$(since otherwise $x_m=\sup\limits_{i\in I,i\geq m}x_i$ and hence $m\in J$)Hence,$\forall\ finite\ subset\ I'\ of\{m\in I\mid m\geq n_0\}$\\\indent 
$\max\limits_{i\in I}x_i<l$ and hence $\exists n\in I$,such that $n>\max I'$,and $\max\limits_{i\in I'}x_i<x_n$\\
We construct by induction an increasing sequence $(n_j)_{j\in \mathbb{N} }$ in $I$\\\indent 
Let $n_0$ be as above. Let $f:\mathbb{N} \rightarrow I_{\geq n_0}$ be a surjective mapping.\\\indent 
If $n_j$ is chosen, we choose $n_{j+1}\in I$ such that $n_{j+1}>n_j, x_{n_{j+1}}>\max\{x_{f(j)},x_{n_j}\}$ Hence the sequence $(x_{n_j})_{j\in \mathbb{N} }$ is increasing, and $\sup\limits_{j\in \mathbb{N} }x_{n_j}\leq\sup\limits_{j\in\mathbb{N} }x_{f(j)}=\sup\limits_{n\in I,n\geq n_0}x_n=l$\\\indent 
$l=\sup\limits_{n\in I,n\geq n_0}$\\\indent 
So $(x_{n_j})_{j\in\mathbb{N} }$ tends to $l$
\chapter{Cauchy sequence}
\section{Def}
Let $(x_n)_{n\in I}$ be a sequence in $\mathbb{R} $\\
If $\inf\limits_{N\in \mathbb{N} }\sup\limits_{(n,m)\in I\times I,\ n,m\geq N}\lvert x_n-x_m\rvert=\lim\limits_{N\rightarrow+\infty}\sup\limits_{(n,m)\in I\times I,\ n,m\geq N}\lvert x_n-x_m\rvert=0$ then we say that $(x_n)_{n\in I}$ is a Cauchy sequence
\section{Prop}
\begin{itemize}
    \item If $(x_n)_{i\in I}\in \mathbb{R} ^I$ converges to some $l\in \mathbb{R} $, then it is a Cauchy sequence
    \item If $(x_N)_{i\in I}$ is a Cauchy sequence, there exists $M>0$ such that $ \forall n\in I\quad \lvert x_n\rvert\leq M$
    \item If $(x_n)_{n\in I}$ is a Cauchy sequence, then $\forall J\subseteq I$ infinite,$(x_n)_{n\in I}$ is a Cauchy sequence.
    \item If $(x_n)_{n\in I}$ is a Cauchy sequence, then $\forall J\subseteq I$ infinite and $l\in \mathbb{R} $ such that $(x_n)_{n\in I}$ converges to $l$, then $(x_n)_{n\in J}$ converges to $l $ too.
\end{itemize}
\section{Theorem: Completeness of real number}
If $(x_n)_{n\in I}\in\mathbb{R} ^I$ is a Cauchy sequence,
then it converges in $\mathbb{R} $
\subsection{Proof}
Since $(x_n)_{n\in I}$ is a Cauchy sequence, $\exists M\in \mathbb{R} _{>0}$ such that $-M\leq x_n\leq M\quad \forall x\in I$ So $\limsup\limits_{n\in I,n\rightarrow+\infty}x_n\in \mathbb{R} $. By Bolzano-Weierstrass theorem. $\exists J\subseteq I$ infinite such that $(x_n)_{n\in I}$ converges to $\limsup\limits_{n\in I,n\rightarrow+\infty}x_n\in\mathbb{R} $. Therefore $(x_n)_{n\in I}$ converges to the same limit.
\section{Absolutely converge}
We say that a series $\sum\limits_{n\in I}x_n\in \mathbb{R} $ converges absolutely if $\sum\limits_{n\in I}\lvert x_n\rvert<+\infty$
\subsection{Prop}
If a series $\sum\limits_{n\in I}x_n$ converges absolutely, then it converges in $\mathbb{R} $
\chapter{Comparison and Technics of Computation}
\section{Def}
Let $(x_n)_{n\in I}$ and $(y_n)_{n\in I}$ be sequence in $\mathbb{R} $
\begin{itemize}
    \item If there exists $M\in\mathbb{R} _{>0}$ and $N\in \mathbb{N} $ such that $\forall n\in I_{\geq N},\lvert x_N\rvert\leq M\lvert y_m\rvert$ then we write $x_n=O(y_n),n\in I,n\rightarrow+\infty$
    \item If there exists $(\epsilon_n)_{n\in I}\in \mathbb{R} ^I$ and $N\in\mathbb{N} $ such that $\lim\limits_{n\in I,n\rightarrow+\infty}\epsilon_n=0$ and $\forall n\in I_{\geq N},\lvert x_N\rvert\leq \lvert \epsilon y_m\rvert$, then we write $x_n=\circ(y_n),n\in I,n\rightarrow +\infty$ \\Example:
        $$\lim\limits_{n\rightarrow+\infty}{1\over n}=0$$
\end{itemize}
\end{document}
